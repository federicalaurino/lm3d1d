\section{Saddle-point problem analysis}
Let $a: X \times X \rightarrow \mathbb{R}$ and $b: X\times Q \rightarrow \mathbb{R}$ be bounded bilinear forms. Let us consider the general saddle point problem of the form: find $u\in X$, $\lambda\in Q$ s.t.
\begin{eqnarray}\label{eq:saddle-point}
\begin{cases}
a(u,v)+b(v,\lambda)=c(v)\quad &\forall v\in X\\
b(u,\mu)=d(\mu) \quad &\forall \mu\in Q.
\end{cases}
\end{eqnarray}
We denote with $A$ and $B$ the operators associated to the bilinear forms $a$ and $b$, 
namely $A: X \longrightarrow X'$ with $\langle Au,v\rangle _{X',X} = a(u,v)$ and $B: X \longrightarrow Q'$ with $\langle Bv,\mu\rangle_{Q',Q} = b(v,\mu)$. Problem \eqref{eq:saddle-point} embraces problems 1 and 2 described before. For the analysis of such problems we apply the following general abstract theorem.
\begin{theorem}{\cite[Theorem 2.34]{MR2050138}}\label{th:bnb}
Problem \eqref{eq:saddle-point} is well posed iff 
\begin{eqnarray}\label{BNB1}
\begin{cases}
\exists \alpha >0 :\, \inf_{u\in ker(B)}\sup_{v\in ker(B)} \frac{a(u,v)}{\|u\|_{X}\|v\|_{X}}\geq \alpha\\
\forall v \in ker(B), \, \left( \forall u \in ker(B),\, a(u,v)=0 \right)\implies v=0.
\end{cases}
\end{eqnarray}
and 
\begin{equation}\label{eq:infsup}
\exists \beta >0:\,\inf_{\mu\in Q}\sup_{v\in X} \frac{b(v,\mu)}{\|v\|_{X}\|\mu\|_{Q}}\geq \beta .
\end{equation}
\end{theorem} 
Notice that if $a$ is coercive on $ker(B)$, \eqref{BNB1} is clearly fulfilled. \\

% >>>>>>>>>>>>>>>>>>>>>>>>>>>>>>>>>>>>>>>>>>>>>>>>>>>>>>>>>>>>>>>>>>>>>>>>>>>>>>>>>>>>>>>>>>>>>>>>>>>>
\subsection{Problem 1}
It consists to find $u \in H^1_0(\Omega),\ \ud \in H_0^1(\Lambda), \ \lambda \in H^{-\frac12}(\Gamma )$, %such that
%\begin{subequations}\label{eq:red_dirneu}
%\begin{align}
%&(u,v)_{H^1(\Omega)} + |{\cal D}| (\ud,\vd)_{H^1(\Lambda)} 
%+ \langle \Pi_1 v  - \Pi_2 \vd, L \rangle_\Gamma 
%\\
%\nonumber
%&\qquad\qquad= (f,v)_{L^2(\Omega)} + |{\cal D}|(\avrd{g},\vd)_{L^2(\Lambda)}
%\quad \forall v \in H^1_0(\Omega), \ \vd \in H^1_0(\Lambda)
%\\
%&   \langle \Pi_1 u - \Pi_2 \ud , M \rangle_\Gamma = 0
%\quad \forall M \in H^{-\frac12}(\Gamma)\,,
%\end{align}
%\end{subequations}

solutions of \eqref{eq:saddle-point}, where
\begin{equation*}
a([u, \ud], [v, \vd])= (u,v)_{H^1(\Omega)} + (\ud,\vd)_{H^1(\Lambda),|\D|}
\end{equation*} 
\begin{equation*}
b([v, \vd], \mu)= \langle \trace v - \ext \vd , \mu \rangle_\Gamma
\end{equation*} 
\begin{equation*}
c([v,\vd])= (f,v)_{L^2(\Omega)} + (\avrd{g},\vd)_{L^2(\Lambda),|\D|}
\end{equation*}
\begin{equation*}
d(\mu)=0
\end{equation*}

%Here, $\Pi_1: H^1_0(\Omega) \rightarrow H^{\frac12}_{00}(\Gamma)$ is the trace operator 
%while $\Pi_2$ is the uniform extension from $H^1_0(\Lambda)$ to  $H^{\frac12}_{00}(\Gamma)$. 
We prove that the hypothesis of \ref{th:bnb} are fullfilled choosing 
$X=H^1_0(\Omega) \times H^1_0(\Lambda)$, $Q=H^{-\frac 12}(\Gamma)$, where $X$  is equipped with the norm $\vertiii{[u,\ud ]}^2=\|u\|^2_{H^1(\Omega)} + \|\ud\|^2_{H^1(\Lambda),|\D|}$ and $Q$ equipped with the norm
\begin{equation*}
\|\mu \|_{H^{-\frac 12}(\Gamma)} := \sup_{q\in H^{\frac 12}_{00}(\Gamma)}\frac{\langle q, \mu\rangle_\Gamma}{\|q\|_{H^{\frac 12}(\Gamma)}}
\end{equation*}. 
\begin{lemma}\label{lemma:prob1_boundedness} 
The bilinear forms $a(\cdot \ , \ \cdot)$ and $b(\cdot \ , \ \cdot)$ are bounded.
\end{lemma}
\begin{proof}
The bilinear form $a(\cdot \ , \ \cdot)$ is clearly bounded since
\begin{equation*}
a([u, \ud], [v, \vd])\leq \|u\|_{H^1(\Omega)}\|v\|_{H^1(\Omega)} + \|\ud\|_{H^1(\Lambda), |\D|}\|\vd\|_{H^1(\Lambda), |\D|} \leq 2 \vertiii{[u, \ud]}\vertiii{[v, \vd]}.
\end{equation*}
Concerning the bilinear form $b(\cdot \ , \ \cdot)$ we have
\begin{multline*}
b([v, \vd], \mu)= \langle \trace v - \ext \vd , \mu \rangle_\Gamma 
\leq \|\trace v - \ext \vd\|_{H^{\frac 12}(\Gamma)}\|\mu\|_{H^{-\frac 12}(\Gamma)}\\
\leq \left(\|\trace v\|_{H^{\frac 12}(\Gamma)} + \|\ext \vd\|_{H^{\frac 12}(\Gamma)}\right)\|\mu\|_{H^{-\frac 12}(\Gamma)}
\leq \left(C_T \|v\|_{H^1(\Omega)} + \|\ext \vd\|_{H^1(\Gamma)}\right)\|\mu\|_{H^{-\frac 12}(\Gamma)}\\
\leq \left(C_T \|v\|_{H^1(\Omega)} + \left(\frac{\max |\DD|}{\min |\D|}\right)^{\frac 12} \|\vd\|_{H^1(\Lambda),|\D|}\right)\|\mu\|_{H^{-\frac 12}(\Gamma)}\\
\leq \left( C_T + \left(\frac{\max |\DD|}{\min |\D|}\right)^{\frac 12}\right) \vertiii{[v,\vd]}\|\mu\|_{H^{-\frac 12}(\Gamma)}
\end{multline*}
\end{proof}

\begin{lemma}\label{lemma:prob1_coercivity}
The bilinear form $a(\cdot \ , \ \cdot)$ is coercive .
\end{lemma}
\begin{proof}
 Indeed, we have,
\begin{equation*}
a([u,\ud], [u,\ud])= (u,u)_{H^1(\Omega)} + |{\cal D}|(\ud,\ud)_{H^1(\Lambda)} = \vertiii{[u,\ud]}^2\,.
\end{equation*}
\end{proof}
\begin{lemma}
The inf-sup inequality \eqref{eq:infsup} is fulfilled, namely $\exists \beta_1 >0$ such that $\forall \mu \in H^{-\frac 12}(\Gamma)$:
\begin{equation*}
\sup _{\substack{v\in H^1_0(\Omega),\\ \vd \in H^1_0(\Lambda)}} \frac{ \langle \trace v  - \ext \vd, \mu \rangle_\Gamma}{\vertiii{[v, \vd]}}
\geq \beta_1 \sup_{q\in H^{\frac 12}_{00}(\Gamma)}\frac{\langle q, \mu\rangle_{\Gamma}}{\|q\|_{H^{\frac 12}(\Gamma)}}.
\end{equation*}
\end{lemma} 
\begin{proof}
We choose $\vd \in H^1_0(\Lambda)$ such that $\ext\vd =0$. Therefore,
\begin{equation*}
\sup _{\substack{v\in H^1_0(\Omega),\\ \vd \in H^1_0(\Lambda)}} \frac{ \langle \trace v  - \ext \vd, \mu\rangle_\Gamma}{\vertiii{[v, \vd]}} 
\geq \sup _{v\in H^1_0(\Omega)} \frac{ \langle \trace v, \mu \rangle_\Gamma}{\|v\|_{H^1(\Omega)}}.
\end{equation*}

We notice that the trace operator is surjective from $H^1_0(\Omega)$ to $H^{\frac12}_{00}(\Gamma)$. Indeed, $\forall \xi \in H^{\frac 12}_{00}(\Gamma)$, we  can find $v$ solution of
\begin{eqnarray*}
-\Delta v&=0 \quad &\text{in }\Omega\\
v&=0 &\text{on }\partial \Omega\\
v&=\xi &\text{on } \Gamma. 
\end{eqnarray*}
We denote with $\mathcal{E}_\Omega$ the harmonic extension operator defined above.
The boundedness/stability of this operator ensures that there exists $\| \mathcal{E}_\Omega \| \in \mathbb{R}$ such that
$v=\mathcal{E}_\Omega(\xi) $ and $\|v \|_{H^1(\Omega)}\leq \|\mathcal{E}_\Omega\| \|\xi \|_{H^{\frac 12}(\Gamma)}$. 
Substituting in the previous inequalities we obtain
\begin{equation}\label{infsup_traceop}
\sup _{v\in H^1_0(\Omega)} \frac{ \langle \trace v, \mu \rangle_\Gamma}{\|v\|_{H^1(\Omega)}}
\geq  \sup _{\xi \in H^{\frac 12}_{00}(\Gamma )} \frac{ \langle \xi , \mu \rangle_\Gamma}{\|\mathcal{E}_\Omega\| \|\xi\|_{H^{\frac 12}(\Gamma)}}
= \|\mathcal{E}_\Omega\|^{-1} \|\mu\|_{H^{-\frac 12}(\Gamma)},
\end{equation}
where in the last inequality we exploited the fact that $H^{-\frac 12}(\Gamma)=(H^{\frac 12 }_{00}(\Gamma))^*$. 
\end{proof}


% >>>>>>>>>>>>>>>>>>>>>>>>>>>>>>>>>>>>>>>>>>>>>>>>>>>>>>>>>>>>>>>>>>>>>>>>>>>>>>>>>>>>>>>>>>>>>>>>>>>>
\subsection{Problem 2}
This problem requires to find $u \in H^1_0(\Omega),\ \ud \in H^1_0(\Lambda), \ \ld \in H^{-\frac12}(\Lambda)$, %such that
%\begin{subequations}\label{eq:red_dirneu}
%\begin{align}
%&(u,v)_{H^1(\Omega)} + |{\cal D}|(\ud,\vd)_{H^1(\Lambda)} 
%+ |\partial {\cal D}| \langle  \Pi_1 \vd - \Pi_2 v, L \rangle_\Lambda 
%\\
%\nonumber
%&\qquad\qquad= (f,v)_{L^2(\Omega)} + |{\cal D}| (\avrd{g},V)_{L^2(\Lambda)}
%\quad \forall v \in H^1_0(\Omega), \ \vd \in H^1(\Lambda)
%\\
%&  |\partial {\cal D}| \langle \Pi_1 \ud - \Pi_2 u, M \rangle_\Lambda = 0
%\quad \forall M \in H^{-\frac12}(\Lambda)\,.
%\end{align}
%\end{subequations}

solution of \eqref{eq:saddle-point} with
\begin{equation*}
a([u, \ud], [v, \vd])= (u,v)_{H^1(\Omega)} + (\ud,\vd)_{H^1(\Lambda),|\D|}
\end{equation*} 
\begin{equation*}
b([v, \vd], \md)=  \langle  \mtrace v - \vd, \md \rangle_{\Lambda, |\DD|} 
\end{equation*} 
\begin{equation*}
c([v,\vd])= (f,v)_{L^2(\Omega)} + (\avrd{g},\vd)_{L^2(\Lambda),|\D|}
\end{equation*}
\begin{equation*}
d(\md)=0
\end{equation*}

%Here, $\Pi_1: H^1_0(\Lambda)\rightarrow H^{\frac 12}_{00}(\Lambda)$ is the immersion operator and $\Pi_2: H^1_0(\Omega)\rightarrow H^{\frac 12}_{00}(\Lambda)$ is defined as the composition of the trace operator $T_{\Gamma}: H^1_0(\Omega) \rightarrow H^{\frac 12}_{00}(\Gamma)$ and the average operator $\bar{(\,)}:H^{\frac 12}_{00}(\Gamma) \rightarrow H^{\frac 12}_{00}(\Lambda)$, namely $\Pi_2= \bar{(\,)}\circ T_{\Gamma}$. 
%First of all we prove that if $u\in H^1_0(\Omega)$, than $\avrc{T u} \in H^{\frac 12}_{00}(\Lambda)$. In particular, from standard trace theory, we have that $T u\in H^{\frac 12}_{00}(\Gamma)$, therefore we have to prove that if $u \in H^{\frac 12 }_{00}(\Gamma)$ then $\avrc{u}\in H^{\frac 12}_{00}(\Lambda)$. 


%%%%%Analysis with a non-weighted norm for the multiplier: the constants depend on the min and max \DD
%We prove that the hypotesis of  Theorem \ref{th:bnb} are fulfilled with the following spaces $X=H^1_0(\Omega) \times H^1_0(\Lambda)$, $Q=H^{-\frac 12}(\Lambda)$.
%Let us consider $X$ equipped again with the norm $\vertiii{[\cdot,\cdot ]}$ and  
%$Q$ equipped with the norm $\|\cdot \|_{H^{-\frac 12}}$.
%Then, we have the following lemmas.
%\begin{lemma}
%The bilinear forms $a(\cdot \ , \ \cdot)$ and $b(\cdot \ , \ \cdot)$ are bounded.
%\end{lemma}
%\begin{proof}
%The boundedness of $a(\cdot \ , \ \cdot)$ can be proved as in Lemma \ref{lemma:prob1_boundedness}. Concerning $b(\cdot \ , \ \cdot)$, we have
%\begin{multline*}
%b([v, \vd], \md)=  \langle  \avrc{Tv} - \vd, \md \rangle_{\Lambda, |\DD|} 
%\leq \|\avrc{Tv} - \vd\|_{H^{\frac 12 }(\Lambda), |\DD|}\|\md\|_{H^{-\frac 12}(\Lambda)}\\
%\leq \left(\|\avrc{Tv}\|_{H^{\frac 12 }(\Lambda), |\DD|}+\|\vd\|_{H^{\frac 12 }(\Lambda), |\DD|}\right)\|\md\|_{H^{-\frac 12}(\Lambda)} \\
%\leq \left(\|Tv\|_{H^{\frac 12 }(\Gamma)}+ \|\vd\|_{H^1(\Lambda), |\DD|}\right)\|\md\|_{H^{-\frac 12}(\Lambda)} \\
%\leq \left(C_T\|v\|_{H^1(\Omega)}+ \left(\frac{\max |\D|}{\min |\D|}\right)^{\frac 12} \|\vd\|_{H^1(\Lambda), |\D|}\right) \|\md\|_{H^{-\frac 12}(\Lambda)} \\
%\lesssim \vertiii{[v,\vd]}\|\md\|_{H^{-\frac 12}(\Lambda)} 
%\end{multline*}
%{\color{red} check $\||\DD| \avrc{Tv}\|_{H^{\frac 12}(\Lambda)}\leq \|Tv\|_{H^{\frac 12}(\Gamma)}$} 
%\end{proof}
%
%\begin{lemma}\label{lemma:prob2_coercivity}
%The bilinear form $a(\cdot \ , \ \cdot)$ is coercive.
%\end{lemma}
%
%\begin{lemma}
%The inf-sup inequality \eqref{eq:infsup} holds, namely $ \exists \beta_2 >0$ such that $\forall \md \in H^{-\frac 12}(\Lambda),\,$:
%\begin{equation*}
%\sup _{\substack{v\in H^1_0(\Omega),\\ \vd \in H^1_0(\Lambda)}} \frac{ \langle \avrc{T v} - \vd, \md \rangle_{\Lambda,|\DD|}}{\vertiii{[v,\vd]}}
%\geq \beta_2 \|\md\|_{H^{\frac 12}(\Lambda)}.
%\end{equation*}
%\end{lemma} 
%We choose $\vd=0$ and we obtain
%\begin{equation*}
%\sup _{\substack{v\in H^1_0(\Omega),\\ \vd \in H^1_0(\Lambda)}} \frac{ \langle \avrc{T v} - \vd, \md \rangle_{\Lambda,|\DD|}}{\vertiii{[v,\vd]}}
%\geq \sup _{v\in H^1_0(\Omega)} \frac{ \langle \avrc{T v}, \md \rangle_{\Lambda,|\DD|}}{\|v\|_{H^1(\Omega)}}. 
%\end{equation*}
%
%For any $q \in H^{\frac 12}_{00}(\Lambda)$, we consider its uniform extension to $\Gamma$ $\ext q$
%and then we consider the harmonic extension $v=\mathcal{E}(\ext q)\in H^1_0(\Omega)$. It follows that $\avrc{T v}=q$. Therefore, 
%\begin{equation*}
%\sup _{v\in H^1_0(\Omega)}  \langle \avrc{T v}, \md \rangle_{\Lambda,|\DD|} \gtrsim \sup_{q \in H^{\frac 12}_{00}(\Lambda)} \langle q, \md  \rangle_\Lambda\,.
%\end{equation*}
%Moreover, using Lemma \ref{lemma:H12norm} we obtain
%\begin{equation*}
%\|v\|_{H^1_0(\Omega)}\leq \|\mathcal{E}\| \|\ext q\|_{H^{\frac 12}(\Gamma)}  \lesssim \|\mathcal{E}\| \|q\|_{H^{\frac 12}(\Lambda)}.
%\end{equation*}
% Therefore,
%\begin{multline*}
%\sup _{v\in H^1_0(\Omega)} \frac{ \langle \avrc{T v}, \md \rangle_{\Lambda,|\DD|}}{\|v\|_{H^1(\Omega)}}
%\gtrsim \sup _{q\in H^{\frac 12}_{00}(\Lambda)} \frac{ \langle q, \md \rangle_\Lambda}{\|v\|_{H^1(\Omega)}}
%\gtrsim \|\mathcal{E}\|^{-1} \sup _{q\in H^{\frac 12}_{00}(\Lambda)} \frac{ \langle q, \md \rangle_\Lambda}{\|q\|_{H^{\frac 12}(\Lambda)}} 
%\\= \|\mathcal{E}\|^{-1} \|\md\|_{H^{-\frac 12}(\Lambda)}
%\end{multline*}
%and the constants in the inequalities depend on $\min _{s \in (0,S)} |\DD(s)|$ and $\max _{s \in (0,S)} |\DD(s)|$ which we suppose to be strictly positive.\\

%%%%Analysis with a weighted norm for the multiplier
We prove that the hypotesis of  Theorem \ref{th:bnb} are fulfilled with the following spaces $X=H^1_0(\Omega) \times H^1_0(\Lambda)$, $Q=H^{-\frac 12}(\Lambda)$.
Let us consider $X$ equipped again with the norm $\vertiii{[\cdot,\cdot ]}$ and  
$Q$ equipped with the norm $\|\cdot \|_{H^{-\frac 12}(\Lambda),|\DD|}$, defined as
\begin{equation*}
\|\md \|_{H^{-\frac 12}(\Lambda),|\DD|}:= \sup_{q\in H^{\frac 12}_{00}(\Lambda)}\frac{\langle q, \md\rangle_{\Lambda, |\DD|}}{\|q\|_{H^{\frac 12}(\Lambda),|\DD|}}
\end{equation*}
Then, we have the following lemmas.
\begin{lemma}
The bilinear forms $a(\cdot \ , \ \cdot)$ and $b(\cdot \ , \ \cdot)$ are bounded.
\end{lemma}
\begin{proof}
The boundedness of $a(\cdot \ , \ \cdot)$ can be proved as in Lemma \ref{lemma:prob1_boundedness}. Concerning $b(\cdot \ , \ \cdot)$, we have
\begin{multline*}
b([v, \vd], \md)=  \langle  \mtrace v - \vd, \md \rangle_{\Lambda, |\DD|} 
\leq \|\mtrace v - \vd\|_{H^{\frac 12 }(\Lambda), |\DD|}\|\md\|_{H^{-\frac 12}(\Lambda),|\DD|}\\
\leq \left(\|\mtrace v\|_{H^{\frac 12 }(\Lambda), |\DD|}+\|\vd\|_{H^{\frac 12 }(\Lambda), |\DD|}\right)\|\md\|_{H^{-\frac 12}(\Lambda),|\DD|} \\
\leq \left(C_\Gamma \|\trace v\|_{H^{\frac 12 }(\Gamma)}+ \|\vd\|_{H^1(\Lambda), |\DD|}\right)\|\md\|_{H^{-\frac 12}(\Lambda),|\DD|} \\
\leq \left(C_\Gamma C_T\|v\|_{H^1(\Omega)}+ \left(\frac{\max |\DD|}{\min |\D|}\right)^{\frac 12} \|\vd\|_{H^1(\Lambda), |\D|}\right) \|\md\|_{H^{-\frac 12}(\Lambda),|\DD|} \\
\leq  \left( C_\Gamma C_T + \left(\frac{\max |\DD|}{\min |\D|}\right)^{\frac 12} \right) \vertiii{[v,\vd]}\|\md\|_{H^{-\frac 12}(\Lambda),|\DD|}.
\end{multline*} 
\end{proof}

\begin{lemma}\label{lemma:prob2_coercivity}
The bilinear form $a(\cdot \ , \ \cdot)$ is coercive.
\end{lemma}

\begin{lemma}
The inf-sup inequality \eqref{eq:infsup} holds, namely $ \exists \beta_2 >0$ such that $\forall \md \in H^{-\frac 12}(\Lambda),\,$:
\begin{equation*}
\sup _{\substack{v\in H^1_0(\Omega),\\ \vd \in H^1_0(\Lambda)}} \frac{ \langle \mtrace v - \vd, \md \rangle_{\Lambda,|\DD|}}{\vertiii{[v,\vd]}}
\geq \beta_2 \|\md\|_{H^{\frac 12}(\Lambda)}.
\end{equation*}
\end{lemma} 
We choose $\vd=0$ and we obtain
\begin{equation*}
\sup _{\substack{v\in H^1_0(\Omega),\\ \vd \in H^1_0(\Lambda)}} \frac{ \langle \mtrace v - \vd, \md \rangle_{\Lambda,|\DD|}}{\vertiii{[v,\vd]}}
\geq \sup _{v\in H^1_0(\Omega)} \frac{ \langle \mtrace v, \md \rangle_{\Lambda,|\DD|}}{\|v\|_{H^1(\Omega)}}. 
\end{equation*}

For any $q \in H^{\frac 12}_{00}(\Lambda)$, we consider its uniform extension to $\Gamma$ named as $\ext q$
and then we consider the harmonic extension $v=\mathcal{E}_\Omega \ext q\in H^1_0(\Omega)$. It follows that $\mtrace v=q$. Therefore, 
\begin{equation*}
\sup _{v\in H^1_0(\Omega)}  \langle \mtrace v, \md \rangle_{\Lambda,|\DD|} \geq\sup_{q \in H^{\frac 12}_{00}(\Lambda)} \langle q, \md  \rangle_{\Lambda,|\DD|}\,.
\end{equation*}
Moreover, using Lemma \ref{lemma:H12norm} we obtain
\begin{equation*}
\|v\|_{H^1_0(\Omega)}\leq \|\mathcal{E}_\Omega\| \|\ext q\|_{H^{\frac 12}(\Gamma)}  = \|\mathcal{E}_\Omega\| \|q\|_{H^{\frac 12}(\Lambda),|\DD|}.
\end{equation*}
 Therefore,
\begin{multline*}
\sup _{v\in H^1_0(\Omega)} \frac{ \langle \mtrace v, \md \rangle_{\Lambda,|\DD|}}{\|v\|_{H^1(\Omega)}}
\geq \sup _{q\in H^{\frac 12}_{00}(\Lambda)} \frac{ \langle q, \md \rangle_{\Lambda,|\DD|}}{\|v\|_{H^1(\Omega)}}
\geq \|\mathcal{E}_\Omega\|^{-1} \sup _{q\in H^{\frac 12}_{00}(\Lambda)} \frac{ \langle q, \md \rangle_{\Lambda,|\DD|}}{\|q\|_{H^{\frac 12}(\Lambda),|\DD|}} 
\\= \|\mathcal{E}_\Omega\|^{-1} \|\md\|_{H^{-\frac 12}(\Lambda),|\DD|}.
\end{multline*}



