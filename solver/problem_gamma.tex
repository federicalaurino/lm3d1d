Let $V=H^1{(\Gamma)}$ We consider the minimization problem
\[
\min_{u\in V}\quad\norm{u}^2_1 - \int_{\Gamma}f u\,\mathrm{d}x\quad\mbox{subject to}\quad
u=g\mbox{ in }\Gamma.
\]
Denoting $p$ the Lagrange multiplier for the constraint the extremal points
of the related Lagrangian are found as a solution to
%
\begin{equation}\label{eq:A1}
\begin{pmatrix}f\\g\end{pmatrix}=
\mathcal{A}_1\begin{pmatrix}u\\p\end{pmatrix}=
%
\begin{pmatrix}
  -\Delta + I & I\\
  I           &  \\
\end{pmatrix}
\begin{pmatrix}u\\p\end{pmatrix}.
\end{equation}
%
Setting $Q=H^{-1}(\Gamma)$, the operator $\mathcal{A}_1:W\rightarrow W^{\prime}$
can be seen to be an isomorphism with $W=V\times Q$. In particular the inf-sup
condition holds with constant 1 (by definition of the $H^{-1}$ norm).

While it can be easily seen that P1-P1 discretization is inf-sup stable for
the problem the pair which we are interested in is P1-P0 as piece-wise constant
elements were used in the previous section. We recall that stable P1-P1 discretization
for $\mathcal{A}_2$ (using different meshes for $\Omega$, $\Gamma$ is also possible,
cf. \cite{burman2009interior}), however, it is not clear how to extend this result
to 3$d$-1$d$ problem \eqref{eq:problem2}.

To illustrate that P1-P0 is not inf-sup stable for $\mathcal{A}_1$ consider
a uniform mesh $\Gamma_h$ and for simplicity assume homogeneous Dirichlet
boundary conditions on $V$. Let $\text{mid}_{K}$ be the midpoint of $K\in\Gamma_h$.
Then 
\[
\int_{\Gamma} p v = \sum_{K\in{\Gamma}_h}p(\text{mid}_{K})v(\text{mid}_{K})
\]
and it can be seen that a ``checker-board'' function $p$ renders the integral
zero, that is, the inf-sup condition does not hold.

Let $C$ be the constant from the inf-sup condition of $\mathcal{A}_2$. 
Since $\mathcal{A}_1$ is only a sub-problem of \eqref{eq:operator2d} the
inf-sup condition for the operator of the coupled problem can be obtained as
\[
\sup_{(v,\hat{v})\in V\times \hat{V}}\frac{\int_{\Gamma}(v-\hat{v}) p}{\norm{(v,\hat{v})}_{V\times \hat{V}}}
\geq
\sup_{v\in V}\frac{\int_{\Gamma}v p}{\norm{v}_{V}}
\geq
C\norm{p}_Q.
\]
If we apply similar argument to the discrete inf-sup condition it seems
that the unstable discretization of $\mathcal{A}_1$ does not present issue.
We shall now give an example which demonstrates that for \emph{robust} preconditioning
the inf-sup condition should be established in the (intersection) space in which \emph{both}
subproblems are inf-sup stable.

\begin{example}[Stokes problem]\label{ex:stokes}
  Consider the problem of finding $u_1$, $u_2$, $p$ satisfying
  \[
  \begin{aligned}
  -\Delta u_1 -\epsilon\nabla p &= f_1\mbox{ in }\Omega,\\
  -\Delta u_2 + \nabla p &= f_2\mbox{ in }\Omega,\\
  \nabla\cdot(\epsilon u_1 - u_2) &= g\mbox{ in }\Omega.
  \end{aligned}
  \]
  We shall assume that the $u_1=0$, $u_2=0$ on $\Gamma_D\subset\partial\Omega$
  and $\semi{\Gamma_D}\neq\semi{\partial\Omega}$. Let now $V_1=H^{1}_{0, \Gamma_D}(\Omega)$,
  $V_2=H^{1}_{0, \Gamma_D}(\Omega)$ and $Q=L^2(\Omega)$. Then the operator defined 
  by the weak form of the problem
  \[
  \mathcal{A}=\begin{pmatrix}
  -\Delta & & -\epsilon\nabla\\
  & -\Delta & \nabla\\
  \epsilon\nabla\cdot & -\nabla\cdot & \\
  \end{pmatrix}: V_1\times V_2\times Q \rightarrow (V_1\times V_2\times Q)^{\prime}
  \]
  is an isomorphism. However, with these spaces the Riesz map preconditioner
  is not parameter robust. Instead $Q=\epsilon L^2(\Omega)\cap L^2{(\Omega)}$
  shall be used.

  To illustrate the issue with unstable discretization
  let $V_1$ be discretized by P2 elements while $V_2$ uses P1 elements. As
  P1 elements are used for $Q$ the first Stokes problem is stable while the
  latter one is not. We shall first consider
  \[
  \mathcal{B}=\text{diag}(-\Delta, -\Delta, (\epsilon^2+1)I)^{-1}
  \]
  as a preconditioner for $\mathcal{A}$.

  Table \ref{tab:stokes} shows that
  iterations are stable when $\epsilon\gg 1$ while they are unbounded for
  $\epsilon<1$. A heuristic explanation for
  the observation can be based on Schur complement reasoning. Indeed the
  Schur complement of $\mathcal{A}_h$ consists of two parts corresponding
  to the two Stokes problems
  \[
  S = \epsilon^2\nabla_h\cdot(-\Delta_h)^{-1}\nabla_h + \nabla_h\cdot(-\Delta_h)^{-1}\nabla_h.
  \]
  Owing to the discrete inf-sup condition the spectrum of the first part is
  bounded by $\epsilon^2 I_h$ for all $h$. However, for the second component
  a similar bound does not hold (as the corresponding discretization is unstable)
  and this part can become dominant for small $\epsilon$.

  To obtain parameter robustness we consider a stabilized formulation
  of the Stokes problem in which $\gamma \Delta p = 0$ is added to the
  divergence constraint (with Dirichlet boundary conditions for $p$ on
  $\partial\Omega\setminus \Gamma_D$) and $\gamma$ depends on the square
  of the mesh size, see also \cite[\S 7]{mardal2011preconditioning}. As
  a preconditioner for the stabilized operator we then consider
  \[
  \mathcal{B}_s=\text{diag}(-\Delta, -\Delta, (\epsilon^2+1)I-\beta\Delta)^{-1}.
  \]
  Table \ref{tab:stokes} confirms robustness of the preconditioner. We remark
  that the stabilizing term has little effect on approximation properties
  of the formulation.

  \begin{table}
      \scriptsize{
    \begin{minipage}{0.49\textwidth}
  \begin{center}
    \begin{tabular}{c|ccccc}
      \hline
      \multirow{2}{*}{$\epsilon$} & \multicolumn{5}{c}{$h$}\\
      \cline{2-6}
 &   $2^{-3}$  & $2^{-4}$ & $2^{-5}$ & $2^{-6}$ & $2^{-7}$  \\  
\hline
$10^{6}$ & 54 & 57 & 59 & 59 & 96\\
$10^{4}$ & 121 & 116 & 113 & 107 & 109\\
$10^{2}$ & 100 & 86 & 69 & 121 & 172\\
1 & 177 & 244 & 287 & 311 & 329\\
$2^{-1}$ & 231 & 303 & 372 & 415 & 500\\
$2^{-2}$ & 350 & 453 & 500 & 500 & 500\\
$10^{-1}$ & 500 & 500 & 500 & 500 & 500\\
\hline
  \end{tabular}
  \end{center}
  \end{minipage}
      }
            \scriptsize{
    \begin{minipage}{0.49\textwidth}
  \begin{center}
    \begin{tabular}{c|cccccc}
      \hline
      \multirow{2}{*}{$\epsilon$} & \multicolumn{5}{c}{$h$}\\
      \cline{2-6}
 &   $2^{-3}$  & $2^{-4}$ & $2^{-5}$ & $2^{-6}$ & $2^{-7}$  \\  
\hline
$10^{6}$ & 51 & 54 & 55 & 57 & 58\\
$10^{4}$ & 51 & 54 & 55 & 57 & 58\\
$10^{2}$ & 51 & 54 & 56 & 57 & 58\\
1 & 36 & 42 & 47 & 49 & 50\\
$10^{-2}$ & 32 & 38 & 44 & 47 & 49\\
$10^{-4}$ & 32 & 38 & 44 & 47 & 49\\
$10^{-6}$ & 32 & 38 & 44 & 47 & 49\\
\hline
  \end{tabular}
  \end{center}
  \end{minipage}
  }
    \caption{Iteration counts for the Stokes problem. (left) Formulation without
      stabilization using preconditioner $\mathcal{B}$. (right) Stabilized
      formulation with preconditioner $\mathcal{B}_s$. Maximum number of
    iterations allowed is 500.}
  \label{tab:stokes}
  \end{table}
\end{example}

Example \ref{ex:stokes} motivates the need to find a stabilizing term
for the P1-P0 discretization of $\mathcal{A}_1$. In addition, the stabilization
should be consistent so as to not ruin the approximation properties of the
formulation.

\subsection{Stabilized $\mathcal{A}_1$} We construct the stabilizing term
based on two observations. (i) In \S\ref{sec:problem_omega} the stabilization
for $p\in Q_h\subset H^{-1/2}$ takes the form of the facet integral with integrand $h^{2}\jump{p}\jump{q}$.
(ii) For P0 elements $-\Delta_h$ is spectrally equivalent to operator involving
facet integrals with integrand $h^{-1}\jump{p}\jump{q}$. In turn we assume
that the stabilization for $H^s$ norms takes the form $h^{p}\jump{p}\jump{q}$
and by interpolation from (i) and (ii) $p=-2s+1$. Thus the stabilizing term for $H^{-1}$ shall
involve the cube of the mesh size and we consider a discrete formulation of \eqref{eq:A1}
in terms of the operator
%
\[
\langle\mathcal{A}_{1, h}\begin{pmatrix}u\\p\end{pmatrix},
  \begin{pmatrix}v\\q\end{pmatrix}
    \rangle
    =
    \int_{\Gamma}\nabla u\cdot\nabla v+uv\,\mathrm{d}x+\int_{\Gamma}uq + vp\,\mathrm{d}x-
    \sum_{F\in\mathcal{F}}h^3\jump{p}\jump{q}\,\mathrm{d}s.
\]
The preconditioner for $\mathcal{A}_{1, h}$ is then the Riesz mapping which
includes the negative stabilized term
\[
\langle\mathcal{B}^{-1}_{1, h}\begin{pmatrix}u\\p\end{pmatrix},
  \begin{pmatrix}v\\q\end{pmatrix}
    \rangle
    =
    \int_{\Gamma}\nabla u\cdot\nabla v+uv\,\mathrm{d}x+
    \langle(-\Delta+I)^{-1}p, q\rangle+
    \sum_{F\in\mathcal{F}}h^3\jump{p}\jump{q}\,\mathrm{d}s.
\]

Table \ref{tab:A1} shows error convergence of the proposed formulation
together with the condition number of the preconditioned problem. We remark
that the manufactured solution on the unit interval has $g=0$ so that $p=f=\cos(\pi x)$
and $u=0$ is the solution. It can be seen that the formulation is stable and
yields converging solutions.

\begin{table}
  \footnotesize{
    \begin{center}
      \begin{tabular}{c|cc|c}
        \hline
        $h$ & $\norm{u-u_h}_1$ & $\norm{p-p_h}_0$ & $\kappa$ \\
        \hline
% 2.50E-01& 6.4083E-02(--)  & 2.4243E-01(--)  & 2.65607  \\
% 1.25E-01& 5.5055E-03(3.54)& 8.0279E-02(1.59)& 2.66149  \\
% 6.25E-02& 3.5103E-04(3.97)& 4.0044E-02(1.00)& 2.66364  \\
% 3.12E-02& 2.1949E-05(4.00)& 2.0040E-02(1.00)& 2.66582  \\
 1.56E-02& 1.3715E-06(4.00)& 1.0020E-02(1.00)& 2.66592  \\
 7.81E-03& 8.5716E-08(4.00)& 5.0100E-03(1.00)& 2.66596  \\
 3.91E-03& 5.3572E-09(4.00)& 2.5050E-03(1.00)& 2.66599  \\
 1.95E-03& 3.3482E-10(4.00)& 1.2525E-03(1.00)& 2.66599  \\
 9.77E-04& 2.0926E-11(4.00)& 6.2625E-04(1.00)& 2.66599  \\
 4.88E-04& 1.3079E-12(4.00)& 3.1312E-04(1.00)& 2.66575  \\
        \hline
      \end{tabular}
    \end{center}
  }
  \caption{Conditioning and approximation properties of stabilized
    formulation of \eqref{eq:A1} using P1-P0 elements.
  }
  \label{tab:A1}
\end{table}

