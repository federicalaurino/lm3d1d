\section{Finite element approximation (Different meshes for solution and Lagrange multiplier)}
Let us introduce an admissibile triangulation $\mathcal{T}^{\Omega}_h$ of $\Omega$ and an admissible partition $\mathcal{T}^{\Lambda}_{h}$ of $\Lambda$. We denote by $X^0_h(\Omega)\subset H^1_0(\Omega)$ the conforming finite element space of continuous piecewise linear functions defined on $\Omega$ and by $X_{h}(\Lambda)\subset H^1_0(\Lambda)$ the space of continuous piecewise linear functions defined on $\Lambda$. Moreover, $Q_H$ denotes a suitable trial space for the lagrange multiplier $L_H$, defined on a different triangulation of $\Gamma$ with mesh size $H$. In particular, $Q_H \subset H^{-\frac12}(\Gamma )$ in the case of Problem 1 and $Q_H \subset H^{-\frac12}(\Lambda)$ in the case of Problem 2. 

\subsection{Problem 1}

It consists to find $u_h \in X_h(\Omega) ,\ {\ud}_h \in X_h(\Lambda) , \ L_H \in Q_H(\Gamma) \subset H^{-\frac12}(\Gamma )$, such that
\begin{subequations}\label{eq:red_dirneu}
\begin{align}
&(u_h,v_h)_{H^1(\Omega)} + |{\cal D}| ({\ud}_h,{\vd}_h)_{H^1(\Lambda)} 
+ \langle \Pi_1 v_h  - \Pi_2 {\vd}_h, L_H \rangle_\Gamma 
\\
\nonumber
&\qquad\qquad= (f,v_h)_{L^2(\Omega)} + |{\cal D}|(\avrd{g},{\vd}_h)_{L^2(\Lambda)}
\quad \forall v_h \in X_h(\Omega), \ {\vd}_h \in X_h(\Lambda)
\\
&   \langle \Pi_1 u_h - \Pi_2 {\ud}_h , M_H \rangle_\Gamma = 0
\quad \forall M_H \in Q_H(\Gamma)\,,
\end{align}
\end{subequations}

\begin{theorem}
$\exists \gamma _1 >0$ s.t.
\begin{equation}\label{inf_sup_discrete_prob1}
\inf_{M_H \in Q_H(\Gamma)} 
\sup_{\substack{v_h \in X_h(\Omega),\\ {\vd}_h \in X_h(\Lambda)}} \frac{ \langle \Pi_1 v_h - \Pi_2 {\vd}_h, M_H \rangle _{\Gamma}} {\vertiii{[v_h, {\vd}_h]} \|M_H\|_{H^{-\frac 12 }(\Gamma)}} 
\geq \gamma _1. 
\end{equation}
\end{theorem}

\begin{proof}
Let $M_H \in Q_H(\Gamma)$. As in the continuos case, let us choose ${\vd}_h =0$, therefore 
\begin{equation*}
\sup_{\substack{v_h \in X_h(\Omega),\\ {\vd}_h \in X_h(\Lambda)}} \frac{ \langle \Pi_1 v_h - \Pi_2 {\vd}_h, M_H \rangle _{\Gamma}} {\vertiii{[v_h, {\vd}_h]}}
\geq \sup_{v_h \in X_h(\Omega)} \frac{ \langle \Pi_1 v_h, M_H \rangle _{\Gamma} } {\|v_h\|_{H^1(\Omega)}}.
\end{equation*}
Following \cite[Theorem 11.5]{steinbach2007numerical}, it can be shown under the following assumptions 
\begin{itemize}
\item the mesh size $h$ of the trial space $X_h(\Omega)$ is sufficiently small compared to the mesh size $H$ of $Q_H(\Gamma)$, i.e. $h \leq c_0 H$ with $c_0 < 1$, and
\item a global inverse inequality for the trial space $Q_H(\Gamma)$ holds,
\end{itemize}
that exists a positive constant $c_S$
\begin{equation}\label{infsup_tracespace}
c_S \|M_H\|_{H^{-\frac 12}(\Gamma)} \leq 
\sup_{w_h \in W_h(\Gamma)} \frac{ \langle w_h, M_H \rangle _{\Gamma} } {\|w_h\|_{H^{\frac 12}(\Gamma)}} \qquad \forall M_H \in Q_H(\Gamma),
\end{equation}
being $W_h(\Gamma)$ the trace space of the functions in $X_h(\Omega)$, namely the space of the restrictions of the functions in $X_h(\Omega)$ to $\Gamma$. 
Using the boundedness of the extension operator $E$ from $H^{\frac 12}_0(\Gamma)$ to $H^1_0(\Omega)$ introduced in the previous section, we have
\begin{equation*}
\sup_{w_h \in W_h(\Gamma)} \frac{ \langle w_h, M_H \rangle _{\Gamma} } {\|w_h\|_{H^{\frac 12}(\Gamma)}} 
\leq 
\|E\| \sup_{w_h \in W_h(\Gamma)} \frac{ \langle w_h, M_H \rangle _{\Gamma} } {\|E w_h\|_{H^1(\Omega)}}.
\end{equation*}
Let $R_h: H^1_0(\Omega) \rightarrow X_h(\Omega)$ be a quasi interpolation operator satisfying 
\begin{equation*}
\|R_h v\|_{H^1(\Omega)} \leq C_R \|v\|_{H^1(\Omega)} \qquad \forall v \in H^1_0(\Omega).
\end{equation*}
Therefore, we obtain 
\begin{equation*}
\|E\| \sup_{w_h \in W_h(\Gamma)} \frac{ \langle w_h, M_H \rangle _{\Gamma} } {\|E w_h\|_{H^1(\Omega)}}
\leq
\|E\| C_R \sup_{w_h \in W_h(\Gamma)} \frac{ \langle w_h, M_H \rangle_{\Gamma} } {\|R_hE w_h\|_{H^1(\Omega)}}
\end{equation*}
and using \eqref{infsup_tracespace}, we have
\begin{multline}
c_S \|M_H\|_{H^{-\frac 12}(\Gamma)} 
\leq 
\sup_{w_h \in W_h(\Gamma)} \frac{ \langle w_h, M_H \rangle_{\Gamma} } {\|w_h\|_{H^{\frac 12}(\Gamma)}} 
\leq
\|E\| C_R \sup_{w_h \in W_h(\Gamma)} \frac{ \langle w_h, M_H \rangle_{\Gamma} } {\|E w_h\|_{H^1(\Gamma)}}
\\
=
\|E\| C_R \sup_{w_h \in W_h(\Gamma)} \frac{ \langle \Pi_1 \ R_h E w_h, M_H \rangle_{\Gamma} } {\|R_h E w_h\|_{H^1(\Omega)}} 
\leq \|E\| C_R \sup_{v_h \in X_h(\Omega)} \frac{ \langle \Pi_1 v_h, M_H \rangle_{\Gamma} } {\|v_h\|_{H^1(\Omega)}}. 
\end{multline}
Therefore the inf-sup condition $\eqref{inf_sup_discrete_prob1}$ holds with $\gamma_1 = c_S\|E\|^{-1} C_R^{-1}$.
\end{proof}


\subsection{Problem 2}
This problem requires to find  $u_h \in X_h(\Omega) ,\ {\ud}_h \in X_h(\Lambda), \ L_H \in Q_H(\Lambda) \subset H^{-\frac12}(\Lambda)$, such that
\begin{subequations}
\begin{align*}
&(u_h,v_h)_{H^1(\Omega)} + |{\cal D}|({\ud}_h,{\vd}_h)_{H^1(\Lambda)} 
+ |\partial {\cal D}| \langle  \Pi_1 {\vd}_h - \Pi_2 v_h, L_H \rangle_\Lambda 
\\
\nonumber
&\qquad\qquad= (f,v_h)_{L^2(\Omega)} + |{\cal D}| (\avrd{g},{\vd}_h)_{L^2(\Lambda)}
\quad \forall v_h \in X_h(\Omega), \ {\vd}_h \in X_h(\Lambda)
\\
&  |\partial {\cal D}| \langle \Pi_1 {\ud}_h - \Pi_2 u_h, M_H \rangle_\Lambda = 0
\quad \forall M_H \in Q_H(\Lambda)\,.
\end{align*}
\end{subequations}

\begin{theorem}
$\exists \gamma _2 >0$ s.t.
\begin{equation}\label{inf_sup_discrete_prob2}
\inf_{M_H \in Q_H(\Lambda)} 
\sup_{\substack{vh \in X_h(\Omega),\\ {\vd}_h \in X_h(\Lambda)} }\frac{\langle \Pi_1 v_h - \Pi_2 {\vd}_h, M_H \rangle _{\Lambda} } {\vertiii{[v_h, {\vd}_h]} \|M_H\|_{H^{-\frac 12 }(\Lambda)}  } 
\geq \gamma _2. 
\end{equation}
\end{theorem}
\begin{proof}
Let $M_H$ be arbitrarly chosen in $Q_H(\Lambda)$. Again, we choose ${\vd}_h =0$, so that \eqref{inf_sup_discrete_prob2} reduces to prove
\begin{equation*}
\gamma _2 \|M_H\|_{H^{\frac 12}(\Lambda)}
\leq 
\sup_{v_h \in X_h(\Omega)} \frac{ \langle \Pi_2 v_h , M_H \rangle _{\Lambda} } {\|v_h\|_{H^1(\Omega)} } \qquad \forall M_H \in Q_H(\Lambda).
\end{equation*}
Assume that
\begin{itemize}
\item the mesh size $h$ of the trial space $X_h(\Omega)$ is sufficiently small compared to the mesh size $H$ of $Q_H(\Lambda)$, i.e. $h \leq c_1 H$ with $c_1 < 1$,
\item a global inverse inequality for the trial space $Q_H(\Lambda)$ holds and
\item the space $W_h(\Lambda)$, defined as the space of the restrictions on $\Gamma$ of the functions in $X_h(\Omega)$ averaged on the cross section, has the approximation property, namely if $Q_h^{\sigma}$ denotes the projection from $H^\sigma (\Lambda)$ to $W_h(\Lambda)$, we have
\begin{equation*}
\|w-Q_h^{\sigma} w\|_{H^{\sigma} (\Lambda)} \leq c_A h^{s-\sigma} |w|_{H^s(\Lambda)} \qquad \forall w \in H^s(\Lambda).
\end{equation*}
(Actually $W_h(\Lambda)$ concides with the space of piecewise linear continuous polinomials on $\Lambda$).
\end{itemize}
Under the previous assumptions, an inequality similar to \eqref{infsup_tracespace} holds also for $W_h(\Lambda)$. In particular, following the same proof in Steinbach, we can prove 
\begin{equation}
c_{S_2} \|M_H\|_{H^{-\frac 12}(\Lambda)} \leq 
\sup_{w_h \in W_h(\Lambda)} \frac{ \langle w_h, M_H \rangle _{\Lambda} } {\|w_h\|_{H^{\frac 12}(\Lambda)}} \qquad \forall M_H \in Q_H(\Lambda).
\end{equation}
If we denote with $\mathcal{U}_E$ the uniform extension operator from $\Lambda$ to $\Gamma$, using Lemma \ref{lemma:H12norm_avrc}, we easily have for any $w \in H^{\frac 12}(\Lambda)$,
\begin{equation*}
\|\mathcal{U}_E w\|_{H^{\frac 12}(\Gamma)}=|\DD| \|w\|_{H^{\frac 12}(\Lambda)}.
\end{equation*}
Consequently, using again the extension operator $E$ from $H^{\frac 12}_0(\Omega)$ to $H^1_0(\Omega)$ and the quasi interpolation operator $R_h$ from $H^1_0(\Omega)$ to $X_h(\Omega)$, we obtain
\begin{multline}
c_{S_2} \|M_H\|_{H^{-\frac 12}(\Lambda)} \leq 
\sup_{w_h \in W_h(\Lambda)} \frac{ \langle w_h, M_H \rangle_{\Lambda} } {\|w_h\|_{H^{\frac 12}(\Lambda)}} 
\\
\leq |\DD| \sup_{w_h \in W_h(\Lambda)} \frac{ \langle w_h, M_H \rangle _{\Lambda}} {\|\mathcal{U}_E w_h\|_{H^{\frac 12}(\Gamma)}} 
\leq |\DD| \|E\| \sup_{w_h \in W_h(\Lambda)} \frac{ \langle w_h, M_H \rangle _{\Lambda} } {\|E \mathcal{U}_E w_h\|_{H^1(\Omega)}} 
\\
\leq |\DD|\|E\| C_R \sup_{w_h \in W_h(\Lambda)} \frac{ \langle w_h, M_H \rangle _{\Lambda} } {\|R_h E \mathcal{U}_E w_h\|_{H^1(\Omega)}}
\\ 
=  |\DD| \|E\| C_R \sup_{w_h \in W_h(\Lambda)} \frac{ \langle \Pi _1  R_h E \mathcal{U}_E w_h, M_H \rangle _{\Lambda}} {\|R_h E \mathcal{U}_E w_h\|_{H^1(\Omega)}}
\\
\leq |\DD| \|E\| C_R \sup_{v_h \in X_h(\Omega)} \frac{ \langle \Pi _2  v_h, M_H \rangle _{\Lambda}} {\|v_h\|_{H^1(\Omega)}}. 
\end{multline}
Therefore, \eqref{inf_sup_discrete_prob2} holds with $\gamma_2=  c_{S_2} |\DD|^{-1} \|E\|^{-1} C_R^{-1}$
\end{proof}

\section{Finite element approximation (Same mesh for solution and Lagrange multiplier)}
Let us introduce a shape-regular triangulation $\mathcal{T}^{\Omega}_h$ of $\Omega$ and an admissible partition $\mathcal{T}^{\Lambda}_{h}$ of $\Lambda$. We denote by $X_{h,k}^0(\Omega)\subset H^1_0(\Omega)$ the conforming finite element space of continuous piecewise polynomials of degree $k$ defined on $\Omega$ satysfing homogeneous Dirichlet conditions on the boundary and by $X_{h,k}^0(\Lambda)\subset H^1_0(\Lambda)$ the space of continuous piecewise polynomials of degree $k$ defined on $\Lambda$, satisfying homogeneous Dirichlet conditions on $\Lambda \cap \partial \Omega$. Moreover, $Q_{h}$ denotes a suitable trial space for the lagrange multiplier $L_h$. In particular, $Q_h \subset H^{-\frac12}(\Gamma )$ in the case of Problem 1 and $Q_h \subset H^{-\frac12}(\Lambda)$ in the case of Problem 2.
 
\subsection{Problem 1}It consists to find $u_h \in X_{h,k}^0 (\Omega) ,\ {\ud}_h \in X_{h,k}^0(\Lambda) , \ L_h \in Q_h(\Gamma) \subset H^{-\frac12}(\Gamma )$, such that
\begin{subequations}
\begin{align}
&(u_h,v_h)_{H^1(\Omega)} + |{\cal D}| ({\ud}_h,{\vd}_h)_{H^1(\Lambda)} 
+ \langle \Pi_1 v_h  - \Pi_2 {\vd}_h, L_h \rangle_\Gamma 
\\
\nonumber
&\qquad\qquad= (f,v_h)_{L^2(\Omega)} + |{\cal D}|(\avrd{g},{\vd}_h)_{L^2(\Lambda)}
\quad \forall v_h \in X_{h,k}^0(\Omega), \ {\vd}_h \in X_{h,k}^0(\Lambda)
\\
&   \langle \Pi_1 u_h - \Pi_2 {\ud}_h , M_h \rangle_\Gamma = 0
\quad \forall M_h \in Q_h(\Gamma)\,,
\end{align}
\end{subequations}

Let us denote with $W_{h,k}^0(\Gamma) \subset H^{\frac 12}_{00}(\Gamma)$ the trace space of functions running in $X_{h,k}^0(\Omega)$, namely the space of continuous piecewise polynomials of degree $k$ defined on $\Gamma$ which satisfy homogeneous Dirichlet conditions on $\partial \Omega$. We choose $Q_h(\Gamma)=W_{h,k}^0(\Gamma)$, therefore we impose homogeneous Dirichlet boundary condition on $\partial \Omega$ also for the Lagrange multiplier. For this choice of $Q_h(\Gamma)$ we can prove the well-posedness of the discrete problem, as shown in the following. 

\begin{lemma}
Let $P_h: H^{\frac 12}_{00}(\Gamma) \longrightarrow W_{h,k}^0(\Gamma)$ be the orthogonal projection operator defined  for any $v \in H^{\frac 12}_{00}(\Gamma)$ by
\begin{equation*}
(P_h v , \psi)_\Gamma= (v, \psi)_\Gamma \qquad \forall \psi \in W_{h,k}^0(\Gamma).  
\end{equation*} 
Then, $P_h$ is continuous on $H^{\frac 12}_{00}(\Gamma)$, namely
\begin{equation}\label{continuity_projoper}
\|P_h v\|_{H^{\frac 12}_{00}(\Gamma)} \leq C \|v\|_{H^{\frac 12}_{00}(\Gamma)},
\end{equation}
where $C$ is a positive constant independent of $h$.
\end{lemma}
\begin{proof}
We prove that $P_h$ is continuous on $L^2(\Gamma)$ and on $H^1_0(\Gamma)$ following \cite[Section 1.6.3]{MR2050138}.  Then, the inequality \eqref{continuity_projoper} can be dirived by Hilbertian interpolation. For the $L^2$-continuity, we exploit the fact that, from the definition of $P_h$,
\begin{equation*}
(v-P_h v,P_h v)_{\Gamma}=0.
\end{equation*} 
Therefore, by Pythagoras identity,
\begin{equation*}
\|v\|^2_{L^2(\Gamma)} = \|v-P_h v\|_{L^2(\Gamma)}^2 + \|P_h v\|_{L^2(\Gamma)}^2 \geq \|P_h v\| _{L^2(\Gamma)}.
\end{equation*}
Let us now consider $v\in H^1_0(\Gamma)$.The Scott-Zhang interpolation operator $SZ_h$ from $H^1_0(\Gamma)$ to $W_{h,k}^0(\Gamma)$ satisfies the following inequalities,
\begin{equation}\label{SZ_stability}
\|SZ_h v\|_{H^1(\Gamma)} \leq C_1 \|v\|_{H^1(\Gamma)}
\end{equation} 
\begin{equation}\label{SZ_approx}
\|v -SZ_h v \|_{L^2(\Gamma)}\leq C_2 h \|v\|_{H^1(\Gamma)}.
\end{equation}
Therefore,
\begin{equation*}
\begin{split}
\|\nabla P_h v\|_{L^2(\Gamma)} 
&\leq \|\nabla (P_h v - SZ_h v)\|_{L^2(\Gamma)} + \|\nabla SZ_h v\|_{L^2(\Gamma)}\\
&\leq \text{ (using \eqref{SZ_stability}) } \|\nabla (P_h v - SZ_h v)\|_{L^2(\Gamma)} + C_1\|v\|_{H^1(\Gamma)}
\end{split}
\end{equation*}
and by using the inverse inequality we obtain
\begin{equation*}
\begin{split}
\|\nabla (P_h v - SZ_h v)\|_{L^2(\Gamma)} + C_1\|v\|_{H^1(\Gamma)}
&\leq \frac{C_3}{h} \|P_h v - SZ_h v\|_{L^2(\Gamma)} + C_1\|v\|_{H^1(\Gamma)}\\
&= \frac{C_3}{h} \|P_h (v - SZ_h v)\|_{L^2(\Gamma)} + C_1\|v\|_{H^1(\Gamma)}\\
&\leq  \text{ (Stability of $P_h$ in $L^2$) } \frac{C_3}{h} \|v - SZ_h v\|_{L^2(\Gamma)} + C_1\|v\|_{H^1(\Gamma)}\\
&\leq \text{ (using \eqref{SZ_approx}) } \frac{C_3}{h} C_2 h  \| v\|_{H^1(\Gamma)} + C_1\|v\|_{H^1(\Gamma)}\\
&\leq (C_2 C_3 +C_1) \|v\|_{H^1(\Gamma)},
\end{split}
\end{equation*}
from which we obtain the continuity in $H^1_0(\Gamma)$.
\end{proof}

\begin{lemma}\label{lemma:trspace_infsup} 
There exist a constant $\gamma >0$ such that
\begin{equation*}
\sup_{\substack{q_h \in W_{h,k}^0(\Gamma)}} \frac{\langle q_h , M_h \rangle}{ \|q_h\|_{H^{\frac 12}(\Gamma)}} \geq \gamma \|M_h\|_{H^{-\frac 12}(\Gamma)}.
\end{equation*} 
\end{lemma}
\begin{proof}
Let $M_h$ be in $W_{h,k}^0(\Gamma)$. From the continuous case, in particular from \eqref{infsup_traceop}, we have
\begin{equation*}
\|E\|^{-1} \|M_h\|_{H^{-\frac 12}(\Gamma)} \leq \sup_{\substack{v \in H^1_0(\Omega)}} \frac{\langle \Pi_1 v , M_h \rangle}{\|v\|_{H^1(\Omega)}} =\sup_{\substack{v \in H^1_0(\Omega)}} \frac{\langle T_{\Gamma} v , M_h \rangle}{\|v\|_{H^1(\Omega)}}
\end{equation*}
and by the trace inequality $\|T_{\Gamma}v\|_{H^\frac 12 (\Gamma)} \leq K_1 \|v\|_{H^1(\Omega)}$ (see \cite[7.56]{adams1975pure}), we obtain 
\begin{equation*}
\sup_{\substack{v \in H^1_0(\Omega)}} \frac{\langle T_{\Gamma} v , M_h \rangle}{\|v\|_{H^1(\Omega)}}
\leq K_1 \sup_{\substack{v \in H^1_0(\Omega)}} \frac{\langle T_{\Gamma} v , M_h \rangle}{ \|T_{\Gamma}v\|_{H^{\frac 12}(\Gamma)}}.
\end{equation*}
By the definition of $P_h$ and \eqref{continuity_projoper} 
\begin{equation*}
\begin{split}
K_1 \sup_{\substack{v \in H^1_0(\Omega)}} \frac{\langle T_{\Gamma} v , M_h \rangle}{ \|T_{\Gamma}v\|_{H^{\frac 12}(\Gamma)}}&= K_1 \sup_{\substack{v \in H^1_0(\Omega)}} \frac{\langle P_h(T_{\Gamma} v) , M_h \rangle}{ \|T_{\Gamma}v\|_{H^{\frac 12}(\Gamma)}}\\
&\leq  K_1 C \sup_{\substack{v \in H^1_0(\Omega)}} \frac{\langle P_h(T_{\Gamma}) v , M_h \rangle}{ \|P_h(T_{\Gamma}v)\|_{H^{\frac 12}(\Gamma)}}\\
&= K_1 C \sup_{\substack{q_h \in W_{h,k}^0(\Gamma)}} \frac{\langle q_h , M_h \rangle}{  \|q_h\|_{H^{\frac 12}(\Gamma)}}.
\end{split}
\end{equation*}
\end{proof}

\begin{theorem}[Discrete inf-sup] 
$\exists \gamma _1 >0$ s.t.
\begin{equation}\label{inf_sup_discrete_prob1}
\inf_{M_h \in W^0_{h,k}(\Gamma)} 
\sup_{\substack{v_h \in X^0_{h,k}(\Omega),\\ {\vd}_h \in X^0_{h,k}(\Lambda)}} \frac{ \langle \Pi_1 v_h - \Pi_2 {\vd}_h, M_h \rangle _{\Gamma}} {\vertiii{[v_h, {\vd} _h]} \|M_h\|_{H^{-\frac 12 }(\Gamma)}} 
\geq \gamma _1. 
\end{equation}
\end{theorem}

\begin{proof}
Let $M_h \in W^0_{h,k}$. As in the continuos case, we choose ${\vd}_h =0$ and we have
\begin{equation*}
\sup_{\substack{v_h \in X^0_{h,k}(\Omega),\\ {\vd}_h \in X^0_{h,k}(\Lambda)}} \frac{ \langle \Pi_1 v_h - \Pi_2 {\vd}_h, M_h \rangle _{\Gamma}} {\vertiii{[v_h,  {\vd}_h]}}
\geq \sup_{v_h \in X^0_{h,k}(\Omega)} \frac{ \langle \Pi_1 v_h, M_h \rangle _{\Gamma} } {\|v_h\|_{H^1(\Omega)}}.
\end{equation*}

Therefore, we want to prove that there exists $\gamma _1$ such that
\begin{equation*}
\sup_{v_h \in X^0_{h,k}(\Omega)} \frac{ \langle \Pi_1 v_h, M_h \rangle _{\Gamma} } {\|v_h\|_{H^1(\Omega)}}=\sup_{\substack{v_h \in X^0_{h,k}(\Omega)}} \frac{\langle T_{\Gamma} v_h , M_h \rangle}{\|v_h\|_{H^1(\Omega)}} \geq \gamma_1 \|M_h\|_{H^{-\frac 12}(\Gamma)} \qquad \forall M_h \in W_{h,k}^0.
\end{equation*}

Using Lemma \ref{lemma:trspace_infsup} and the boundedness of the armonic extension operator $E$ from $H^{\frac 12}(\Gamma)$ to $H^1_0(\Omega)$ introduced in the previous section, we have
\begin{equation*}
\gamma \|M_h\|_{H^{-\frac 12}(\Gamma)} \leq  \sup_{q_h \in W_{h,k}^0(\Gamma)} \frac{ \langle q_h, M_h \rangle _{\Gamma} } {\|q_h\|_{H^{\frac 12}(\Gamma)}} 
\leq 
\|E\| \sup_{q_h \in W_{h,k}^0(\Gamma)} \frac{ \langle q_h, M_h \rangle _{\Gamma} } {\|Eq_h\|_{H^1(\Omega)}} .
\end{equation*}
Let $R_h: H^1_0(\Omega) \rightarrow X_{h,k}^0(\Omega)$ be a quasi interpolation operator satisfying 
\begin{equation*}
\|R_h v\|_{H^1(\Omega)} \leq C_R \|v\|_{H^1(\Omega)} \qquad \forall v \in H^1_0(\Omega).
\end{equation*}
Therefore, we obtain 
\begin{equation*}
\|E\| \sup_{q_h \in W_{h,k}^0(\Gamma)} \frac{ \langle q_h, M_h \rangle _{\Gamma} } {\|Eq_h\|_{H^1(\Omega)}} 
\leq
\|E\| C_R \sup_{q_h \in W_{h,k}^0(\Gamma)} \frac{ \langle q_h, M_h \rangle _{\Gamma} } {\|R_h E q_h\|_{H^1(\Omega)}}
\end{equation*}
and we have
\begin{multline}
\gamma \|M_h\|_{H^{-\frac 12}(\Gamma)} 
\leq 
\sup_{q_h \in W_{h,k}^0(\Gamma)} \frac{ \langle q_h, M_h \rangle_{\Gamma} } {\|q_h\|_{H^{\frac 12}(\Gamma)}} 
\leq
\|E\| C_R \sup_{q_h \in W_{h,k}^0(\Gamma)} \frac{ \langle q_h, M_h \rangle_{\Gamma} } {\|R_h E q_h\|_{H^1(\Gamma)}}
\\
=
\|E\| C_R \sup_{q_h \in W_{h,k}^0(\Gamma)} \frac{ \langle \Pi_1 \ R_h E q_h, M_h \rangle_{\Gamma} } {\|R_h E q_h\|_{H^1(\Omega)}} 
\leq \|E\| C_R \sup_{v_h \in X_{h,k}(\Omega)} \frac{ \langle \Pi_1 v_h, M_h \rangle_{\Gamma} } {\|v_h\|_{H^1(\Omega)}}. 
\end{multline}
Therefore the inf-sup condition $\eqref{inf_sup_discrete_prob1}$ holds with $\gamma_1 = \gamma\|E\|^{-1} C_R^{-1} $.
\end{proof}

\begin{remark} We notice that to prove the result in Lemma \ref{lemma:trspace_infsup} (and then the discrete inf-sup condition)  basically we need a projection operator $P_h: H^{\frac 12}_{00} \longrightarrow W_{h,k}^0(\Gamma)$ orthogonal in the multiplier space $Q_h(\Gamma)$, namely such that $\langle P_h v, M_h \rangle = \langle v, M_h \rangle, \, \forall M_h \in Q_h(\Gamma)$, and continuous in $H^{\frac 12}(\Gamma)$. Therefore, in principle different choices than $Q_h(\Lambda)=W_{h,k}^0(\Gamma)$ could be considered if we can build an operator $P_h$ satisfying these properties. In \cite{belgacem1999mortar} such operator $P_h$  is built for a particular choice of $Q_h(\Gamma)$ but it is not clear how they prove the $H^1$-stability inequality (and consequently the $H^{\frac 12 }$-stability) with a constant independent of the mesh size $h$...
\end{remark}  
\subsection{Problem 2}
This problem requires to find  $u_h \in X^0_{h,k}(\Omega) ,\ {\ud}_h \in X^0_{h,k}(\Lambda), \ L_h \in Q_h(\Lambda) \subset H^{-\frac12}(\Lambda)$, such that
\begin{subequations}
\begin{align*}
&(u_h,v_h)_{H^1(\Omega)} + |{\cal D}|({\ud}_h,{\vd}_h)_{H^1(\Lambda)} 
+ |\partial {\cal D}| \langle  \Pi_1 {\vd}_h - \Pi_2 v_h, L_h \rangle_\Lambda 
\\
\nonumber
&\qquad\qquad= (f,v_h)_{L^2(\Omega)} + |{\cal D}| (\avrd{g},{\vd}_h)_{L^2(\Lambda)}
\quad \forall v_h \in X_h(\Omega), \ {\vd}_h \in X_h(\Lambda)
\\
&  |\partial {\cal D}| \langle \Pi_1 {\ud}_h - \Pi_2 u_h, M_h \rangle_\Lambda = 0
\quad \forall M_h \in Q_h(\Lambda)\,.
\end{align*}
\end{subequations}

We introduce the space
$W_{h,k}^0(\Lambda) \subset H^{\frac 12} _{00} (\Lambda)$, which is the averaged trace space of functions running in $H^1_0(\Omega)$. It coincides with the space of continuous piecewise polynomials of degree $k$ defined on $\Lambda$ and satisfying homogeneous Dirichlet boundary condition. {\color{red} (Add assumptions..)}
We choose $Q_h(\Lambda)=W_{h,k}^0(\Lambda)$, therefore we impose homogeneous Dirichlet boundary condition on $\Lambda \cap \partial \Omega$ also for the Lagrange multiplier. With this choice for $Q_h(\Lambda)$, we can prove the well-posedness of the discrete problem. In particular, following the same steps as for Problem 1, we can prove the following results.
\begin{lemma}
Let $P_h: H^{\frac 12}_{00}(\Lambda) \longrightarrow W_{h,k}^0(\Lambda)$ be the orthogonal projection operator defined  for any $v \in H^{\frac 12}_{00}(\Lambda)$ by
\begin{equation*}
(P_h v , \psi)_\Lambda= (v, \psi)_\Lambda \qquad \forall \psi \in W_{h,k}^0(\Lambda).  
\end{equation*} 
Then, $P_h$ is continuous on $H^{\frac 12}_{00}(\Lambda)$, namely
\begin{equation*}
\|P_h v\|_{H^{\frac 12}_{00}(\Lambda)} \leq C \|v\|_{H^{\frac 12}_{00}(\Lambda)},
\end{equation*}
where $C$ is a positive constant independent of $h$.
\end{lemma}

\begin{lemma}\label{infsup_avr_trspace}
There exist a constant $\gamma >0$ such that
\begin{equation*}
\sup_{\substack{q_h \in W_{h,k}^0(\Lambda)}} \frac{\langle q_h , M_h \rangle}{ \|q_h\|_{H^{\frac 12}(\Lambda)}} \geq \gamma \|M_h\|_{H^{-\frac 12}(\Lambda)} \qquad \forall M_h \in W_{h,k}^0(\Lambda).
\end{equation*} 
\end{lemma}

\begin{theorem}[Discrete inf-sup]
$\exists \gamma _2 >0$ s.t.
\begin{equation}
\inf_{M_h \in Q_h(\Lambda)} 
\sup_{\substack{vh \in X^0_{h,k}(\Omega),\\ {\vd}_h \in X^0_{h,k}(\Lambda)} }\frac{\langle \Pi_1 v_h - \Pi_2 {\vd}_h, M_h \rangle _{\Lambda} } {\vertiii{[v_h, {\vd}_h]} \|M_h\|_{H^{-\frac 12 }(\Lambda)} } 
\geq \gamma _2. 
\end{equation}
\end{theorem}
\begin{proof}
Let $M_h$ be arbitrarly chosen in $W_{h,k}^0(\Lambda)$. Again, we choose ${\vd}_h =0$, so that the proof reduces to show that there exists $\gamma_2$ s.t.
\begin{equation*}
\gamma _2 \|M_h\|_{H^{-\frac 12}(\Lambda)}
\leq 
\sup_{v_h \in X_{h,k}^0(\Omega)} \frac{ \langle \Pi_2 v_h , M_h \rangle _{\Lambda} } {\|v_h\|_{H^1(\Omega)} } \qquad \forall M_h \in W_{h,k}^0(\Lambda).
\end{equation*}
Let us denote with $\mathcal{U}_E$ the uniform extension operator from $\Lambda$ to $\Gamma$. Using Lemma \ref{lemma:H12norm_avrc}, we easily have for any $w \in H^{\frac 12}(\Lambda)$,
\begin{equation*}
\|\mathcal{U}_E w\|_{H^{\frac 12}(\Gamma)}=|\DD| \|w\|_{H^{\frac 12}(\Lambda)}.
\end{equation*}
Consequently, from Lemma \ref{infsup_avr_trspace}, using again the extension operator $E$ from $H^{\frac 12}(\Gamma)$ to $H^1_0(\Omega)$ and the quasi interpolation operator $R_h$ from $H^1_0(\Omega)$ to $X^0_{h,k}(\Omega)$, we obtain
\begin{multline}
\gamma \|M_h\|_{H^{-\frac 12}(\Lambda)} \leq 
\sup_{q_h \in W_{h,k}^0(\Lambda)} \frac{ \langle q_h, M_h \rangle_{\Lambda} } {\|q_h\|_{H^{\frac 12}(\Lambda)}} 
\\
= |\DD| \sup_{q_h \in W_{h,k}^0(\Lambda)} \frac{ \langle q_h, M_h \rangle _{\Lambda}} {\|\mathcal{U}_E q_h\|_{H^{\frac 12}(\Gamma)}} 
\leq |\DD|\|E\| \sup_{q_h \in W_{h,k}^0(\Lambda)} \frac{ \langle q_h, M_h \rangle _{\Lambda} } {\|E \mathcal{U}_E q_h\|_{H^1(\Omega)}} 
\\
\leq |\DD|\|E\| C_R \sup_{q_h \in W_{h,k}^0(\Lambda)} \frac{ \langle q_h, M_h \rangle _{\Lambda} } {\|R_h E \mathcal{U}_E q_h\|_{H^1(\Omega)}}
\\ 
=  |\DD|\|E\| C_R \sup_{q_h \in W_{h,k}^0(\Lambda)} \frac{ \langle \Pi _1  R_h E \mathcal{U}_E q_h, M_h \rangle _{\Lambda}} {\|R_h E \mathcal{U}_E w_h\|_{H^1(\Omega)}}
\\
\leq |\DD|\|E\| C_R \sup_{v_h \in X_h(\Omega)} \frac{ \langle \Pi _2  v_h, M_h \rangle _{\Lambda}} {\|v_h\|_{H^1(\Omega)}}. 
\end{multline}
\end{proof}