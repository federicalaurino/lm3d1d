% SIAM Article Template
\documentclass[r]{siamart171218}
% Packages and macros go here
\usepackage[english]{babel}
\usepackage{amsmath}
\usepackage{amssymb}
\usepackage{amsfonts}
\usepackage{array}
\usepackage{graphicx}
\usepackage{epsfig}
\usepackage{float}
\usepackage{fullpage}
\usepackage{color}
\usepackage{enumitem}  
\usepackage{epstopdf}


% Title. If the supplement option is on, then "Supplementary Material"
% is automatically inserted before the title.
%\title{On the weak coupling of 3D and 1D second order elliptic problems}
\title{Coupling PDEs on XD-YD domains with Lagrange multipliers}

% Authors: full names plus addresses.
\author{Yes}

\begin{document}

\maketitle

% REQUIRED
\begin{abstract}
  This note summarizes numerical experiments investigating stabilization
  techniques for coupled multiscale problems using Lagrange multipliers. 
\end{abstract}

% REQUIRED
\begin{keywords}
No
\end{keywords}

% REQUIRED
\begin{AMS}
n.a.
\end{AMS}

% >>>>>>>>>>>>>>>>>>>>>>>>>>>>>>>>>>>>>>>>>>>>>>>>>>>>>>>>>>>>>>>>>>>>>>>>>>>>>>>>>>>>>>>>>>>>>>>>>>>>
 
\newcommand{\semi}[1]{\lvert{#1}\rvert}
\newcommand{\norm}[1]{\lVert{#1}\rVert}

\newtheorem{thm}{Theorem}[section]
\newtheorem{prop}{Property}[section]
\theoremstyle{remark}
\newtheorem{remark}{Remark}[section]
 

\section{Babu{\v s}ka problem}\label{sec:babuska}
We consider the Poisson problem with boundary conditions enforced by
Lagrange multiplier. Given bounded $\Omega\subset\mathbb{R}^2$, let $\Gamma_D$,
$\Gamma_L\subset \partial \Omega$ be such that $\semi{\Gamma_i}\neq 0$,
$i\in\left\{D, L\right\}$, $\bigcup_{i}\Gamma_i=\partial\Omega$.
We then consider the Poisson problem
\[
\begin{aligned}
-\Delta u &= f &\mbox{ in }\Omega,\\
u &= g &\mbox{ on }\Gamma_D,\\
u &= g &\mbox{ on }\Gamma_L,\\
\end{aligned}
\]
which upon introducing the Lagrange multiplier $p\in Q=(H^{1/2}_{00}(\Gamma_L))^{\prime}$
leads to a variational problem: Find $u\in V=H^1_{0, \Gamma_D}(\Omega)$,
$p\in Q$ such that
%
\begin{equation}\label{eq:bab}
  \begin{aligned}
    &\int_{\Omega} \nabla u\cdot \nabla v &+ \int_{\Gamma_L}p v &= \int_{\Omega} f v - \int_{\Gamma_N}h v &\forall v\in V,\\
    &\int_{\Gamma_L}q u  &\phantom{+\int_{\Omega} \nabla u\cdot \nabla v} &= \int_{\Gamma} g q &\forall q\in Q.
  \end{aligned}
\end{equation}
%

In the following experiments $\Omega=(0, 1)^2$, $\Gamma_L=\left\{(x, y)\in\partial\Omega; x=0\right\}$
while $\Gamma_D=\partial\Omega\setminus \Gamma_L$.

\subsection{Discretization by P1-P1 elements}\label{sec:p1_p1}

\begin{table}
  \begin{center}
    \footnotesize{
  \begin{tabular}{c|cc|c|c||cc|c|c}
    \hline
    $h$ & $\norm{u-u_h}_1$ & $\norm{p-p_h}_0$ & \#{iters} & $\kappa$
        & $\norm{u-u_h}_1$ & $\norm{p-p_h}_0$ & \#{iters} & $\kappa$ \\
    \hline
8.84E-02 & 6.91E+00(--)   & 6.37E+00(--)   & 29 & 5.131 & 6.90E+00(--)  & 5.94E+00(--)    & 31 & 4.836\\
4.42E-02 & 3.54E+00(0.97) & 1.61E+00(1.98) & 26 & 5.176 & 3.54E+00(0.96) & 1.76E+00(1.75) & 29 & 4.846\\
2.21E-02 & 1.78E+00(0.99) & 5.45E-01(1.56) & 26 & 5.191 & 1.78E+00(0.99) & 5.95E-01(1.57) & 28 & 4.851\\
1.10E-02 & 8.93E-01(1.00) & 3.11E-01(0.81) & 25 & 5.195 & 8.93E-01(1.00) & 3.24E-01(0.88) & 26 & 4.853\\
5.52E-03 & 4.46E-01(1.00) & 2.12E-01(0.55) & 24 & 5.196 & 4.46E-01(1.00) & 2.16E-01(0.59) & 26 & 4.853\\
2.76E-03 & 2.23E-01(1.00) & 1.49E-01(0.51) & 24 & 5.196 & 2.23E-01(1.00) & 1.50E-01(0.52) & 24 & 4.854\\
1.38E-03 & 1.12E-01(1.00) & 1.06E-01(0.50) & 22 & --    & 1.12E-01(1.00) & 1.06E-01(0.51) & 22 & --   \\
    \hline
  \end{tabular}
  }
  \caption{}
  \label{tab:p1_p1}
  \end{center}
\end{table}

\subsection{Discretization by P1-P0 elements}\label{sec:p1_p0}

\begin{table}
  \begin{center}
    \footnotesize{
  \begin{tabular}{c|cc|c|c||cc|c|c}
    \hline
    $h$ & $\norm{u-u_h}_1$ & $\norm{p-p_h}_0$ & \#{iters} & $\kappa$
        & $\norm{u-u_h}_1$ & $\norm{p-p_h}_0$ & \#{iters} & $\kappa$ \\
    \hline
8.84E-02 & 6.88E+00(--)   & 2.83E+00(--)  & 26 & 4.749  & 6.91E+00(--)   & 1.17E+01(--)  & 35 & 6.940 \\
4.42E-02 & 3.54E+00(0.96) & 2.12E+00(0.42)& 28 & 4.817  & 3.54E+00(0.96) & 2.33E+00(2.33)& 35 & 6.973 \\
2.21E-02 & 1.78E+00(0.99) & 7.42E-01(1.51)& 28 & 4.840  & 1.78E+00(0.99) & 4.91E-01(2.24)& 30 & 6.987 \\
1.10E-02 & 8.93E-01(1.00) & 2.17E-01(1.78)& 25 & 4.850  & 8.93E-01(1.00) & 1.46E-01(1.75)& 28 & 6.994 \\
5.52E-03 & 4.46E-01(1.00) & 8.90E-02(1.28)& 23 & 4.853  & 4.46E-01(1.00) & 5.90E-02(1.31)& 27 & 6.996 \\
2.76E-03 & 2.23E-01(1.00) & 4.28E-02(1.06)& 23 & 4.853  & 2.23E-01(1.00) & 2.77E-02(1.09)& 26 & 6.995 \\
1.38E-03 & 1.12E-01(1.00) & 2.12E-02(1.01)& 23 & --     & 1.12E-01(1.00) & 1.36E-02(1.02)& 25 & --    \\  
    \hline
  \end{tabular}
  }
  \caption{}
  \label{tab:p1_p1}
  \end{center}
\end{table}


%\section{Coupled multiscale problem}\label{sec:coupled}
%bar

%\bibliographystyle{siamplain}
%\bibliography{3d1d_coupled}
\end{document}
