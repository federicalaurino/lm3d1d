% SIAM Article Template
\documentclass[r]{siamart171218}
% Packages and macros go here
\usepackage[english]{babel}
\usepackage{amsmath}
\usepackage{amssymb}
\usepackage{amsfonts}
\usepackage{array}
\usepackage{graphicx}
\usepackage{epsfig}
\usepackage{float}
\usepackage{fullpage}
\usepackage{color}
\usepackage{enumitem}  
\usepackage{epstopdf}
\usepackage{stmaryrd}

% Title. If the supplement option is on, then "Supplementary Material"
% is automatically inserted before the title.
%\title{On the weak coupling of 3D and 1D second order elliptic problems}
\title{Coupling PDEs on 3D-1D domains with Lagrange multipliers}

% Authors: full names plus addresses.
\author{Miroslav Kuchta, Federica Laurino, Kent-Andre Mardal, Paolo Zunino,\thanks{Authors are listed in alphabetical order}}

\begin{document}

\maketitle

% REQUIRED
\begin{abstract}
These are personal notes written to keep track of the developments on this topic, to be kept confidential.
\end{abstract}

% REQUIRED
\begin{keywords}
elliptic problems, high dimensionality gap, essential coupling conditions, Lagrange multipliers
\end{keywords}

% REQUIRED
\begin{AMS}
n.a.
\end{AMS}

% >>>>>>>>>>>>>>>>>>>>>>>>>>>>>>>>>>>>>>>>>>>>>>>>>>>>>>>>>>>>>>>>>>>>>>>>>>>>>>>>>>>>>>>>>>>>>>>>>>>>
 
% definitions here

\def\ud{u_{\odot}}
\def\vd{v_{\odot}}
\def\uf{u_{\ominus}}
\def\up{u_{\oplus}}
\def\eps{\epsilon}
\def\nn{\boldsymbol n}
\def\rr{\boldsymbol r}
\def\RR{\boldsymbol R}
\def\kk{\boldsymbol k}
\def\ss{\boldsymbol s}
\def\uu{\boldsymbol u}
\def\vv{\boldsymbol v}
\def\xx{\boldsymbol x}
\def\bu{\overline{u}}
\def\bv{\overline{v}}
\def\tu{\widetilde{u}}
\def\tv{\widetilde{v}}
\def\TT{\boldsymbol T}
\def\NN{\boldsymbol N}
\def\BB{\boldsymbol B}
\def\ttu{\widetilde{\widetilde{u}}}
\def\ttv{\widetilde{\widetilde{v}}}
\def\cv{\check{v}}
\def\mesh{{\cal T}^h}
\def\ball{{\cal B}}
\def\R{\mathbb{R}}
\def\D{\mathcal{D}}
\def\DD{\partial\mathcal{D}}
\def\trace{\overline{\mathcal{R}}}
\def\ext{\mathcal{E}}
\def\ide{\mathcal{I}}
\def\ii{\hat{\imath}}
\newcommand{\avrd}[1]{\overline{\overline{#1}}}
\newcommand{\avrc}[1]{\overline{#1}}
\newcommand{\refe}[1]{{#1}_{\mathrm{ref}}}

\newcommand{\vertiii}[1]{{\left\vert\kern-0.25ex\left\vert\kern-0.25ex\left\vert #1 
    \right\vert\kern-0.25ex\right\vert\kern-0.25ex\right\vert}}

\newtheorem{thm}{Theorem}[section]
\newtheorem{prop}{Property}[section]
\theoremstyle{remark}
\newtheorem{remark}{Remark}[section]
 
% >>>>>>>>>>>>>>>>>>>>>>>>>>>>>>>>>>>>>>>>>>>>>>>>>>>>>>>>>>>>>>>>>>>>>>>>>>>>>>>>>>>>>>>>>>>>>>>>>>>>
\section{A benchmark problem with analytical solution}

Let $\Omega=[0,1]\times [0,1]\times [0,1]$, \\$\Lambda=\{x=\tfrac{1}{2}\}\times \{y=\tfrac{1}{2}\} \times [0,1] $
and $\Sigma=[\tfrac{1}{4}, \tfrac{3}{4}]\times [\tfrac{1}{4}, \tfrac{3}{4}]\times [0, 1]$.
Finally we let $\DD$ be the cross section of the virtual interface $\Gamma=\partial \Sigma$.
As a benchmark problem for the two formulations we consider the following coupled problems
%
\begin{subequations}\label{benchmark}
\begin{align}
\label{benchm_3d}
-\Delta u=f \quad &\text{in $\Omega$}\\
\label{benchm_1d}
-d_{zz}^2 \ud =g \quad &\text{on $\Lambda$}\\
u=h \quad &\text{on $\partial \Omega$},
\end{align}
\end{subequations}
where for formulation \eqref{eq:problem1} the mix-dimensional coupling constraint reads
\begin{equation}
  \label{eq:couple_1}
\mathcal{T}_{\Gamma}u - \mathcal{E}_{\Gamma}\ud = q_1\quad\text{ on }\Gamma,
\end{equation}
while for \eqref{eq:problem2} we set
\begin{equation}
    \label{eq:couple_2}
\avrc{u} - \ud = q_2\quad\text{ on }\Lambda.
\end{equation}
%
In \eqref{benchmark} the right-hand side data shall be defined as 
\begin{eqnarray*}
f=8\pi ^2 \sin (2\pi x) \sin (2\pi y)\\
g={\pi ^2}\sin \left({\pi z}\right)\\
h=\sin (2\pi x) \sin (2\pi y)\\
q_1=\sin (2\pi x) \sin (2\pi y) - \sin \left({\pi z}\right)\\.
q_2=\sin \left({\pi z}\right)\\.
\end{eqnarray*}

The exact solution of \eqref{benchmark}, regardless of the coupling constraing,
is given by
%
\begin{eqnarray}
\label{benchm_sol3d}
u=\sin (2\pi x) \sin (2\pi y)\\
\label{benchm_sol1d}
%\ud=1+\exp(-z).
\ud=\sin \left(\frac{\pi z}{H}\right) 
\end{eqnarray}
%
Let us notice that $\ud$ satisfies homogeneous Dirichlet conditions at the boundary of $\Lambda$.
Moreover, the solution \eqref{benchm_sol3d}-\eqref{benchm_sol1d} satisfies on $\Gamma$ the relation
\begin{equation}\label{benchm_flux}
\lambda=\nabla u \cdot \textbf{n}_{\oplus}=d_z \ud n_{\oplus,z}=0,
\end{equation}
being $n_{\oplus,z}$ the $z-$component of the normal unit vector to $\Gamma$.

We prove that \eqref{benchmark} with the couplin \eqref{eq:couple2 } is solution
of \eqref{eq:problem2} in the simplified case in which the starting 3D-3D problem is
\begin{subequations}\label{eq:dirneu_simple}
\begin{align}
- \Delta \up  &= f  && \text{ in } \Omega_{\oplus},\\
- \Delta \uf &= g  && \text{ in } \Sigma,\\
-\nabla \uf \cdot \nn_{\ominus} &= -\nabla \up \cdot \nn_{\ominus}  && \text{ on } \Gamma,\\
\uf - \up &= q  && \text{ on }  \Gamma,\\
\up &= h && \text{ on } \partial \Omega.
\end{align}
\end{subequations}
instead of \eqref{eq:dirneu}. Therefore the reduced problem in \eqref{eq:problem2} becomes
 
\begin{subequations}\label{eq:problem2_simple}
\begin{align}
&(\nabla u,\nabla v)_{L^2(\Omega)} + |{\cal D}|(d_s \ud,d_s \vd)_{L^2(\Lambda)} 
+ |\partial {\cal D}| \langle \Pi_1 v - \Pi_2 \vd, L \rangle_{H^{-\frac12}(\Lambda)} 
\\
\nonumber
&\qquad\qquad= (f,v)_{L^2(\Omega)} + |{\cal D}| (\avrd{g},V)_{L^2(\Lambda)}
\quad \forall v \in H^1_0(\Omega), \ \vd \in H^1_0(\Lambda)
\\
&  |\partial {\cal D}| \langle \Pi_1 u -  \Pi_2 \ud, M \rangle_{H^{-\frac12}(\Lambda)} =|\partial {\cal D}| \langle \avrc{q}, M \rangle_{H^{-\frac12}(\Lambda)}
\quad \forall M \in H^{-\frac12}(\Lambda)\,.
\end{align}
\end{subequations}

%% Let us prove that \eqref{benchm_sol3d}-\eqref{benchm_sol1d} is solution of
%% \eqref{eq:problem2_simple}. Using the integration by part formula and homogeneous
%% boundary conditions on $\Omega$ and $\Lambda$, from \eqref{eq:problem1_simple_eq1} we have
%% \begin{align*}
%% &-(\Delta u, v)_{L^2(\Omega)} - |{\cal D}|(d^2_{ss} \ud, \vd)_{L^2(\Lambda)} 
%% + \langle \Pi_1 v  - \Pi_2 \vd, L \rangle_\Gamma
%% \\
%% \nonumber
%% &\qquad\qquad= (f,v)_{L^2(\Omega)} + |{\cal D}| (\avrd{g},\vd)_{L^2(\Lambda)}
%% \quad \forall v \in H^1_0(\Omega), \ \vd \in H^1(\Lambda).
%% \\
%% \end{align*}
%% Clearly, since \eqref{benchm_sol3d} satisfies \eqref{benchm_3d} and \eqref{benchm_sol1d} satisfies \eqref{benchm_1d}, we have that \begin{align*}
%% -(\Delta u, v)_{L^2(\Omega)} =  (f,v)_{L^2(\Omega)} \\
%% -|{\cal D}|(d^2_{ss} \ud, \vd)_{L^2(\Lambda)}  = |{\cal D}| (\avrd{g},\vd)_{L^2(\Lambda)}
%% \end{align*}
%% and being $L=\lambda=0$, we can conclude that \eqref{benchm_sol3d}-\eqref{benchm_sol1d} satisfy \eqref{eq:problem1_simple_eq1}. The fact that the solution satisfy \eqref{eq:problem1_simple_eq2} follows from \eqref{benchm_coupl}. We can prove in the same way that \eqref{benchm_sol3d}-\eqref{benchm_sol1d} is solution of \eqref{eq:problem2_simple}, exploiting the fact that in this case $L=\avrc{\lambda}=0$. 

%% \begin{remark}
%% Let us notice that the 3D solution \eqref{benchm_sol3d} is such that $\avrc{u}=0$. Therefore in \eqref{benchmark} it is like we are solving two separated problems, one in $\Omega$ and the other on $\Lambda$.
%% \end{remark}
%% \begin{remark}
%% It would be interesting to make a comparison between the solution of the fully coupled 3D-3D problem \eqref{eq:dirneu} (also in the simplified case of type \eqref{eq:dirneu_simple}) and the solution of the reduced problems \eqref{eq:problem1} and \eqref{eq:problem2}. 
%% Therefore, we could set the values of the data of the problem such that the reduced formulation becomes non-trivial and fully coupled.
%% Then, we will solve both the original and reduced problem to observe the differences in the solutions and the values of the Lagrange multiplier.
%% \end{remark}

%-----------------
\bibliographystyle{siamplain}
\bibliography{3d1d_coupled}
\end{document}
