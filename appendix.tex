% >>>>>>>>>>>>>>>>>>>>>>>>>>>>>>>>>>>>>>>>>>>>>>>>>>>>>>>>>>>

\section{Derivation of the model}
This section provides a rigorous derivation of 3D-1D-2D problem \eqref{eq:problem2} and 3D-1D-1D problem \eqref{eq:problem1}. 
The steps are similar to the derivation presented in \cite{laurino_m2an}, 
however, here the coupling conditions are different,
giving rise to coupled problems featuring Lagrange multipliers. 
Precisely, the starting point is the problem arising from \emph{Dirichlet-Neumann} conditions. 
Find $\up,\uf$ s.t.:
\begin{subequations}\label{eq:dirneu}
\begin{align}
\label{eq:dirneu1}
- \Delta \up  + \up &= f  && \text{ in } \Omega_{\oplus},\\
\label{eq:dirneu2}
- \Delta \uf  + \uf &= g  && \text{ in } \Omega_\ominus,\\
\label{eq:dirneu3}
-\nabla \uf \cdot \nn_{\ominus} &= -\nabla \up \cdot \nn_{\ominus}  && \text{ on } \Gamma,\\
\label{eq:dirneu4}
\uf &= \up && \text{ on }  \Gamma,\\
\label{eq:dirneu5}
\up &= 0 && \text{ on } \partial \Omega\,.
\end{align}
\end{subequations}
The coupling constraints defined on $\Gamma$ involve essential or strong conditions.
Such conditions will be enforced weakly by using the method of Lagrange multipliers~\cite{MR359352}. Then, the variational formulation of problem \eqref{eq:dirneu}
is to find $\up \in H^1_{\partial\Omega}(\Omega_\oplus), \ \uf \in H^1_{\partial\Omega_\ominus\setminus\Gamma}(\Omega_\ominus), \ \lambda \in H^{-\frac 1 2}(\Gamma)$ s.t.
\begin{subequations}\label{eq:weak_dirneu}
\begin{align}
&(u_\oplus,v_\oplus)_{H^1(\Omega_\oplus)} + (u_\ominus,v_\ominus)_{H^1(\Omega_\ominus)} 
+ \langle  v_\oplus - v_\ominus, \lambda \rangle_{\Gamma} 
\\
\nonumber
&\quad = (f,v_\oplus)_{L^2(\Omega_\oplus)} + (g,v_\ominus)_{L^2(\Omega_\ominus)}
\quad \forall v_\oplus \in H^1_{\partial\Omega}(\Omega_\oplus), \ v_\ominus \in H^1_{\partial\Omega_\ominus\setminus\Gamma}(\Omega_\ominus),
\\
& \langle u_\oplus - u_\ominus, \mu \rangle_{\Gamma} = 0
\quad \forall  \mu \in H^{-\frac 1 2}(\Gamma)\,.
\end{align}
\end{subequations}
%where $\langle v, \mu \rangle_{H^{-\frac 1 2}(\Gamma)}$ denotes the duality pairing between 
%$ \mu \in H^{-\frac 1 2}(\Gamma)$ and $v \in H^{\frac 1 2}(\Gamma)$.
where $\lambda$ is the Lagrange multiplier and it is equivalent to $\nabla \uf \cdot \nn_\ominus$.

\subsubsection*{Model reduction of the problem on $\Omega_{\ominus}$}

We apply the averaging technique to equation \eqref{eq:dirneu2}. In particular, we consider an arbitrary portion $\mathcal{P}$ of the cylinder $\Omega_\ominus$, with lateral surface $\Gamma _{\mathcal{P}}$ and bounded by two perpendicular sections to $\Lambda$, namely $\mathcal{D}(s_1), \ \mathcal{D}(s_2)$ with $s_1<s_2$. We have,
\begin{multline*}
\int_{\mathcal{P}} -\Delta \uf + \uf d\omega =
-\int_{\partial \mathcal{P}} \nabla \uf \cdot \nn_{\ominus} \, d\sigma  + \int_{\mathcal{P}}\uf d\omega=
\\
 \int_{\mathcal{D}(s_1)} \partial_s \uf d\sigma -  \int_{\mathcal{D}(s_2)} \partial_s \uf d\sigma -  \int_{\Gamma_{\mathcal{P}}} \nabla \uf \cdot \nn_{\ominus} d\sigma +\int_{\mathcal{P}}\uf d\omega
\end{multline*}
By the fundamental theorem of integral calculus 
\begin{equation*}
\int_{\mathcal{D}(s_1)} \partial_s \uf d\sigma -  \int_{\mathcal{D}(s_2)} \partial_s \uf d\sigma 
= -\int_{s_1}^{s_2} d_s \int_{\mathcal{D}(s)}  \partial_s \uf d\sigma ds
= -\int_{s_1}^{s_2} d_s \left( |\mathcal{D}(s)| \avrd{\partial_s \uf} \right)
\end{equation*}
Moreover, we have
\begin{equation*}
\int_{\Gamma_{\mathcal{P}}} \nabla \uf \cdot \nn_{\ominus} d\sigma =  \int_{\Gamma_{\mathcal{P}}} \lambda \, d\sigma
=  \int_{s_1}^{s_2} \int_{\partial\mathcal{D}(s)} \lambda d\gamma \, ds 
= \int_{s_1}^{s_2} |\DD| \avrc{\lambda} \,ds\,.
\end{equation*}
From the combination of all the above terms with the right hand side, we obtain that the solution $\uf$ of \eqref{eq:dirneu2} satisfies,
\begin{equation*}
\int_{s_1}^{s_2} \left[ 
-d_s (|\D(s)|\avrd{\partial_s \uf}) + |\D(s)|\avrd{u}_{\ominus} 
- |\partial\mathcal{D}(s)| \avrc{\lambda} - |\mathcal{D}(s)| \avrd{g}
\right]\, ds = 0.
\end{equation*}
Since the choice of the points $s_1,s_2$ is arbitrary, we conclude that the following equation holds true,
\begin{equation}\label{eq:averaged_f}
-d_s (|\D(s)|\avrd{\partial_s \uf}) + |\D(s)|\avrd{u}_{\ominus} 
- |\partial\mathcal{D}(s)| \avrc{\lambda}
= |\mathcal{D}(s)| \avrd{g} \quad \text{on} \ \Lambda\,,
\end{equation}
which is complemented by the following conditions at the boundary of $\Lambda$,
\begin{equation}\label{eq:averaged_f_boundary}
|\D(s)| \avrd{\partial_s \uf}= 0, \quad \text{on} \quad s=0,S.
\end{equation}
Then, we consider variational formulation of the averaged equation \eqref{eq:averaged_f}.
After multiplication by a test function $\vd \in H^1(\Lambda)$, integration on $\Lambda$ and suitable application of integration by parts, we obtain,
\begin{multline*}
\int_\Lambda |\D(s)| \avrd{\partial_s \uf} d_s \vd \, ds - (|\D(s)| \avrd{\partial_s \uf}) \vd |_{s=0}^{s=S}- \int_\Lambda |\DD(s)| \avrc{\lambda} \vd \, ds
+\int_\Lambda |\D(s)| \avrd{u}_\ominus \vd 
\\
= \int_\Lambda |\D(s)| \avrd{g} V\, ds\,.
\end{multline*}
Using boundary conditions, we obtain,
\begin{equation}\label{eq:averaged_f_weak}
  (\avrd{\partial_s \uf}, d_s \vd )_{\Lambda,|\D|} 
+ (\avrd{u}_\ominus, \vd)_{\Lambda, |\D|}
- (\avrc{\lambda},\vd)_{\Lambda,|\DD|}
= (\avrd{g},V)_{\Lambda,|\D|}\,.
\end{equation}
Let us now formulate the modelling assumption that allows us to reduce equation \eqref{eq:averaged_f_weak} to a solvable one-dimensional (1D) model.
More precisely, we assume that the function $\uf$ has a \emph{uniform profile} on each cross section $\mathcal{D}(s)$, namely $\uf(r,s,t) = \ud(s)$.
Therefore, observing that $\ud=\avrc{u}_{\ominus}=\avrd{u}_{\ominus}$, 
and that $\avrd{\partial_s \uf} = \avrd{\partial_s \ud} = d_s \ud$,
problem \eqref{eq:averaged_f_weak} turns out to find $\ud \in H^1(\Lambda)$ such that
\begin{equation}\label{eq:1Ddirneu_weak}
 (d_s \ud, d_s \vd)_{\Lambda,|\mathcal{D}|} 
+(\ud, \vd)_{\Lambda, |\D|}
-(\avrc{\lambda},\vd)_{\Lambda, |\DD|}
=  (\avrd{g},\vd)_{\Lambda, |\mathcal{D}|}
\quad \forall \vd \in H^1(\Lambda)\,.
\end{equation}

% >>>>>>>>>>>>>>>>>>>>>>>>>>>>>>>>>>>>>>>>>>>>>>>>>>>>>>>>>>>
\subsubsection*{Topological model reduction of the problem on $\Omega_{\oplus}$}
We focus here on the subproblem of \eqref{eq:dirneu1} related to $\Omega_{\oplus}$.
We multiply both sides of \eqref{eq:dirneu1} by a test function $v\in H^1_0(\Omega)$ and integrate on $\Omega_\oplus$. Integrating by parts and using boundary and interface conditions, we obtain
\begin{equation*}
\begin{split}
\int_{\Omega _{\oplus}}fv\,d\omega&=
\int _{\Omega _{\oplus}}\nabla \up \cdot\nabla v\,d\omega -\int _{\partial \Omega _{\oplus}}\nabla \up \cdot \nn_{\oplus} v\,d\sigma + \int_{\Omega_{\oplus}} \up v
\\
&=\int _{\Omega _{\oplus}}\nabla \up \cdot\nabla v\,d\Omega -\int_{\Gamma} \nabla \up \cdot \nn_{\oplus} v + \int_{\Omega_{\oplus}} \up v
\\
&=\int _{\Omega _{\oplus}}\nabla \up \cdot\nabla v\,d\Omega +\int_{\Gamma} \lambda v + \int_{\Omega_{\oplus}} \up v.
\end{split}
\end{equation*} 
Then, we make the following modelling assumption:
we identify the domain $\Omega_{\oplus}$ with the entire $\Omega$, 
and we correspondingly omit the subscript $\oplus$ to the functions defined on $\Omega_{\oplus}$,
namely
\begin{equation*}
\int_{\Omega_{\oplus}} u_\oplus\, d\omega \simeq \int_{\Omega} u\, d\omega\,.
\end{equation*}
Therefore, we obtain 
\begin{equation*}
(\nabla u ,\nabla v)_{\Omega} +(u,v)_{\Omega}+(\lambda, v)_{\Gamma}  =(f,v)_{\Omega}
\end{equation*}
and combining with \eqref{eq:1Ddirneu_weak} we obtain the first formulation of the reduced problem.

Hence, we have obtained the Problem 3D-1D-2D, equation 
\eqref{eq:problem1}:
Find $u \in H^1_0(\Omega), \ \lambda \in H^{-\frac 1 2}(\Gamma ),\ \ud \in H^1_0(\Lambda)$, such that
\begin{subequations}
\begin{align*}
&(u,v)_{H^1(\Omega)} + (\ud,\vd)_{H^1(\Lambda),|\D|} 
+ \langle \trace v  - \ext \vd, \lambda \rangle_\Gamma &&
\\
\nonumber
&\qquad= (f,v)_{L^2(\Omega)} +  (\avrd{g},\vd)_{L^2(\Lambda),|\D|}\,,
\quad &&\forall v \in H^1_0(\Omega), \ \vd \in H^1(\Lambda)\,,
\\
& \langle \trace u - \ext \ud , \mu \rangle_\Gamma = 0\,,
\quad && \forall \mu \in H^{-\frac 1 2}(\Gamma)\,.
\end{align*}
\end{subequations}
This coupled problem is classified as 3D-1D-2D because the unknowns $u$, $\ud$, $\lambda$ belong to 
$\Omega \subset \R^3$, $\Lambda \subset \R$ and $\Gamma \subset \R^2$ respectively.
Then, we apply a topological model reduction of the interface conditions, namely we go from a 3D-1D-2D formulation
involving sub-problems on $\Omega$ and $\Lambda$ and coupling operators defined on $\Gamma$
to a 3D-1D-1D formulation where the coupling terms are set on $\Lambda$. 
To this purpose, let us write the Lagrange multiplier and the test functions on every cross section $\partial\mathcal{D}(s)$ as their average plus some fluctuation,
\begin{equation*}
\lambda=\avrc{\lambda}+\tilde{\lambda}, \qquad v=\avrc{v}+\tilde{v},
\quad \text{on} \ \partial\mathcal{D}(s)\,,
\end{equation*}
where $\avrc{\tilde{\lambda}}=\avrc{\tilde{v}}=0$. 
Therefore, the coupling term on $\Gamma$ can be decomposed as,
\begin{equation*}
\int_{\Gamma}\lambda v\, d\sigma
=\int _{\Lambda}  \int_{\partial\mathcal{D}(s)} (\avrc{\lambda}+\tilde{\lambda})(\avrc{v}+\tilde{v})d\gamma ds
= \int_{\Lambda}|\partial\mathcal{D}(s)| \avrc{\lambda}\avrc{v}\,ds+\int_{\Lambda}  \int_{\partial\mathcal{D}(s)} \tilde{\lambda}\tilde{v}d\gamma ds\,.
\end{equation*}
Thanks to the additional assumption that the product of fluctuations is small,
\begin{equation*}
\int_{\partial\mathcal{D}(s)} \tilde{\lambda}\tilde{v} d\gamma \simeq 0\,
\end{equation*}
the term $\left(\trace v, \lambda \right)_{\Gamma}$ becomes $\left( \mtrace v, \avrc{\lambda} \right)_{\Lambda, |\DD|}$, where $\mtrace$ denotes the composition of operators $\avrc{(\cdot)}\circ \trace$.  Combined with \eqref{eq:1Ddirneu_weak}, this leads to the 3D-1D-1D formulation of the reduced problem, namely equation \eqref{eq:problem2}:
find $u \in H^1_0(\Omega)$, $\ud \in H^1_0(\Lambda)$, $\ld \in H^{-\frac 1 2}(\Lambda)$, such that
\begin{subequations}
\begin{align*}
&(u,v)_{H^1(\Omega)} + (\ud,\vd)_{H^1(\Lambda),|\D|} 
+  \langle \mtrace v - \vd, \ld \rangle_{\Lambda, |\DD|} &&
\\
\nonumber
&\qquad\qquad= (f,v)_{L^2(\Omega)} +  (\avrd{g},V)_{L^2(\Lambda),|\D|}\,,
\quad && \forall v \in H^1_0(\Omega), \ \vd \in H^1_0(\Lambda)\,,
\\
& \langle \mtrace u -   \ud, \md \rangle_{\Lambda,|\DD|} = 0\,,
\quad &&\forall \md \in H^{-\frac 1 2}(\Lambda)\,.
\end{align*}
\end{subequations}
% We notice that all the integrals of the reduced problem are well defined because 
% $\mtrace: H^1_0(\Omega) \rightarrow H^\frac 1 2_{00}(\Lambda)$ as shown in Corollary \ref{corollary:H12norm}.

% The well-posedness of $\eqref{eq:problem1}$ and $\eqref{eq:problem2}$ 
% can be studied in the framework of the classical theory of saddle point problems as shown in the following.

%>>>>>>>>>>>>>>>>>>>>>>>>>>>>>>>>>>>>>>>>>>>>>>>>>>>>>>>>>>>>>
\section{Proof of Lemma \ref{lemma:H12norm}}

\begin{proof}
Let us consider the eigenvalue problems for the Laplace operator on $\Gamma$ with homogeneous Dirichlet conditions at $x=0,\, X$ and periodic boundary conditions at $y=0,\, Y$. Let us also consider the Laplace eigenproblem on $(0,X)$ with homogeneous Dirichlet conditions. Let us denote as $\phi _{ij}(x,y)$ and $\rho _{ij}$, for $i=1,2,\dots$, $j=0,1,\dots$, the eigenfunctions and the eigenvalues of the Laplacian on $\Gamma$, and with $\phi _i(x)$ and $\rho _i$ the eigenfunctions and the eigenvalues of the Laplacian on $(0,X)$. In particular,
\begin{align*}
&\phi _{ij}(x,y)=\sin \left(\frac{i\pi x}{X}\right)\left( \cos\left(\frac{j2\pi y}{Y}\right)+ \sin\left(\frac{j2\pi y}{Y}\right) \right),\quad
&&\rho_{ij}=\left(\frac{i\pi}{X}\right) ^2+\left(\frac{j2\pi}{Y}\right)^2,
\\
&\phi _{i}(x)=\sin \left(\frac{i\pi x}{X}\right),\quad
&&\rho _i = \left(\frac{i\pi}{X}\right) ^2.
\end{align*}

Further, we have
the following concrete representation of the fractional norms, 
\begin{equation}\label{eq:fracnorm_lambda}
\begin{aligned}
&\|u\|_{H^{\frac 12}_{00}(\Lambda)}=\left(\sum_{i=1}^{\infty} \left( 1+ \rho_{i}\right)^{\frac 12}|a_i|^2\right)^{\frac 12},\\
&\|u\|_{H^{\frac 12}_{00}(\Gamma)}=
\sum_{i=1}^{\infty} \sum_{j=1}^{\infty}  \left( 1+ \left(\frac{i\pi}{X}\right)^2 
+ \left(\frac{j2\pi}{Y}\right)^2\right)^{\frac 12} |a_{i,j}|^2
\end{aligned}
\end{equation} 
with $a_i=\left(u, \phi _i \right)_{\Lambda}$ and $a_{ij}=\left( u, \phi_{ij}\right)_{\Gamma}$.
It is easy to verify that 
\begin{equation}\label{int_eigenf}
\int_0^{Y} \phi _{ij}(x,y)=0 \quad \forall j>0, \forall i\,,
\quad
\int_0^{Y} \phi _{ij}(x,y)= Y \, \sin\left(\frac{i\pi x}{X}\right) \quad \mbox{if } j=0, \forall i  .
\end{equation}
Moreover we recall that $\phi_{i,j}(x,y)$ and $\phi _i(x)$ form an orthogonal basis of $L^2(\Gamma)$ and $L^2(0,X)$ respectively. Therefore,
\begin{equation*}
\avrc{u}(x)=\frac{1}{Y}\int_0^{Y} u(x,y)\, dy
%= \frac{1}{Y}\int_0^{Y} \sum_{i,j} a_{i,j} \phi_{i,j}(x,y) \, dy
= \frac{1}{Y}\sum_{i,j} a_{i,j}\int_0^{Y}  \phi_{i,j}(x,y) \, dy
=  \sum_{i} a_{i,0} \phi_{i}(x).
\end{equation*}

Let $C=C(X)=\sum_{i=1}^{\infty}\left( 1+ \left(\frac{i\pi}{X}\right)^2\right)^{\frac 12}$. Then,  
from \eqref{eq:fracnorm_lambda} we have $\|\avrc{u}\|^2_{H^{\frac 12}_{00}(0,X)}
=\sum_{i=1}^{\infty} \left( 1+ \rho_{i}\right)^{\frac 12}a_i^2=$
\begin{equation*}
\begin{split}
=&C\left( \int_0^X \avrc{u}(x) \sin\left(\frac{i\pi x}{X}\right)\, dx \right)^2
= C\left( \sum_{j=1}^\infty a_{j,0}\int_0^X \sin\left(\frac{j\pi x}{X}\right) \sin\left(\frac{i\pi x}{X}\right) \, dx  \right)\\
=& \sum_{i=1}^{\infty} \frac{X^2}{4} \left( 1+ \left(\frac{i\pi}{X}\right)^2\right)^{\frac 12}a_{i,0}^2
\leq \frac{X^2}{4}\sum_{i=1}^{\infty} \sum_{j=1}^{\infty}  \left( 1+ \left(\frac{i\pi}{X}\right)^2 
+ \left(\frac{j2\pi}{Y}\right)^2\right)^{\frac 12} |a_{i,j}|^2\\ =&\frac{X^2}{4}\|u\|^2_{H^{\frac 12}_{00}(\Gamma)},
\end{split}
\end{equation*}
where we have used the orthogonality property
\begin{equation*}
\int_0^X \sin\left(\frac{i\pi x}{X}\right) \sin\left(\frac{j\pi x}{X}\right)\, dx=\begin{cases} 0 & i\neq j\\
\frac X 2 & i = j\\
\end{cases}
\end{equation*}
and we have applied \eqref{eq:fracnorm_lambda} in the last equality.
Moreover, in the case in which $u$ is constant with respect to $y$, we have
\begin{equation*}
    \begin{aligned}
&\|u\|^2_{H^{\frac 12}_{00}(\Gamma)}=\sum_{i=1}^{\infty}\sum_{j=0}^{\infty} \left( 1+ \rho_{ij}\right)^{\frac 12}|a_{ij}|^2\\
&=\sum_{i=1}^{\infty}\sum_{j=0}^{\infty} \left(  1+ \left(\frac{i\pi}{X}\right)^2 + \left(\frac{j2\pi}{Y}\right)^2\right)^{\frac 12}\left( \int _0^X\int _0^Y u(x,y )\phi_{ij}(x,y) \right)^2
\\
&=\sum_{i=1}^{\infty}\sum_{j=0}^{\infty} \left(  1+ \left(\frac{i\pi}{X}\right)^2 + \left(\frac{j2\pi}{Y}\right)^2\right)^{\frac 12}\left( \int _0^X u(x) \int _0^Y \phi_{ij}(x,y) \right)^2,
\end{aligned}
\end{equation*}
and using \eqref{int_eigenf} we obtain
\begin{equation*}
\begin{aligned}
\|u\|^2_{H^{\frac 12}_{00}(\Gamma)}
&=\sum_{i=1}^{\infty}\left( 1+ \left(\frac{i\pi}{X}\right)^2\right)^{\frac 12}\left(\int _0^X Yu(x)\, \sin\left(\frac{i\pi x}{X}\right)\right)^2\\
&=Y^2 \sum_{i=1}^{\infty}\left( 1+ \rho _i\right)^{\frac 12}|a_i|^2 = Y^2  \|u\|^2_{H^{\frac 12}_{00}(0,X)}.
\end{aligned}
\end{equation*}
\end{proof}

%>>>>>>>>>>>>>>>>>>>>>>>>>>>>>>>>>>>>>>>>>>>>>>>>>>>>>>>>>>>>>
\section{System sizes in benchmark formulations} \label{sec:appendix}
Below we list dimensions of the finite element spaces used to discretize
formualations \eqref{eq:problem1}, \eqref{eq:problem2} and stabilized
\eqref{eq:problem2} on different levels of refinement. The number of 
degrees of freedom in subspace $W_{i, h}$ is denote as $\lvert W_{i, h}\rvert$. 
We recall that the discrete spaces in the respected formulations 
are $X^1_{h, 0}(\Omega)\times X^1_{h, 0}(\Lambda)\times Q_h(\Gamma)$,
      $X^1_{h, 0}(\Omega)\times X^1_{h, 0}(\Lambda)\times Q_h(\Lambda)$,
      and 
      $X^1_{h, 0}(\Omega)\times X^1_{\mathfrak{h}, 0}(\Lambda)\times Q_h(\mathcal{G}_h)$.

\begin{center}
  \scriptsize{
    \begin{tabular}{l|lll|lll|lll}
      %\toprule
      \hline
      \multirow{2}{*}{$l$} & \multicolumn{3}{c|}{\eqref{eq:problem1}} & \multicolumn{3}{c|}{\eqref{eq:problem2}} & \multicolumn{3}{c}{ Stabilized \eqref{eq:problem2}}\\
      \cline{2-10}
       & $\lvert W_{1, h}\rvert$  & $\lvert W_{2, h}\rvert$ & $\lvert W_{3, h}\rvert$ 
       & $\lvert W_{1, h}\rvert$  & $\lvert W_{2, h}\rvert$ & $\lvert W_{3, h}\rvert$
       & $\lvert W_{1, h}\rvert$  & $\lvert W_{2, \mathfrak{h}}\rvert$ & $\lvert W_{3, h}\rvert$\\
      %& $\lvert X^1_{h, 0}(\Omega) \rvert$ & $\lvert X^1_{h, 0}(\Lambda) \rvert$ & $\lvert Q_h(\Gamma) \rvert$ &
      %$\lvert X^1_{h, 0}(\Omega) \rvert$ & $\lvert X^1_{h, 0}(\Lambda) \rvert$ & $\lvert Q_h(\Lambda) \rvert$ &
      %$\lvert X^1_{h, 0}(\Omega) \rvert$ & $\lvert X^1_{h^{\prime}, 0}(\Lambda) \rvert$ & $\lvert Q_h(\mathcal{G}_h) \rvert$\\
      \hline
     1& 125    & 5  & 40   & 125     & 5   & 5   & 180     & 13  & 24  \\
     2& 729    & 9  & 144  & 729     & 9   & 9   & 900     & 25  & 48  \\
     3& 4913   & 17 & 544  & 4913    & 17  & 17  & 5508    & 49  & 96  \\
     4& 35937  & 33 & 2112 & 35937   & 33  & 33  & 38148   & 97  & 192 \\
     5& 275K & 65 & 8320 & 275K  & 65  & 65  & 283K  & 193 & 384 \\
     6& --     & -- & -    & 2.15M & 129 & 129 & 2.18M & 385 & 768 \\
     %\bottomrule
     \hline
    \end{tabular}
    }
\end{center}
