\section{Saddle-point problem analysis}
Let $a: X \times X \rightarrow \mathbb{R}$ and $b: X\times Q \rightarrow \mathbb{R}$ be bounded bilinear forms. Let us consider the general saddle point problem of the form: find $u\in X$, $\lambda\in Q$ s.t.
\begin{eqnarray}\label{eq:saddle-point}
\begin{cases}
a(u,v)+b(v,\lambda)=c(v)\quad &\forall v\in X\\
b(u,\mu)=d(\mu) \quad &\forall \mu\in Q.
\end{cases}
\end{eqnarray}
We denote with $A$ and $B$ the operators associated to the bilinear forms $a$ and $b$, namely $A: X \longrightarrow X'$ with $\langle Au,v\rangle _{X',X} = a(u,v)$ and $B: X \longrightarrow Q'$ with $\langle Bv,\mu\rangle_{Q',Q} = b(v,\mu)$. Problem \eqref{eq:saddle-point} embraces problems 1 and 2 described before. 
For the analysis of such problems we apply the following general abstract theorem, see for example \cite[Theorem 2.34]{MR2050138} and \cite{MR365287} for a seminal work on this topic.
\begin{theorem}{}\label{th:bnb}
Problem \eqref{eq:saddle-point} is well posed iff 
\begin{eqnarray}\label{BNB1}
\begin{cases}
\exists \alpha >0 :\, \inf_{u\in ker(B)}\sup_{v\in ker(B)} \frac{a(u,v)}{\|u\|_{X}\|v\|_{X}}\geq \alpha\\
\forall v \in ker(B), \, \left( \forall u \in ker(B),\, a(u,v)=0 \right)\implies v=0.
\end{cases}
\end{eqnarray}
\begin{equation}\label{eq:infsup}
\exists \beta >0:\,\inf_{\mu\in Q}\sup_{v\in X} \frac{b(v,\mu)}{\|v\|_{X}\|\mu\|_{Q}}\geq \beta .
\end{equation}
\end{theorem} 
Notice that if $a$ is coercive on $ker(B)$, \eqref{BNB1} is clearly fulfilled. \\

% >>>>>>>>>>>>>>>>>>>>>>>>>>>>>>>>>>>>>>>>>>>>>>>>>>>>>>>>>>>>>>>>>>>>>>>>>>>>>>>>>>>>>>>>>>>>>>>>>>>>
\subsection{Problem 1}
We aim to find $u \in H^1_0(\Omega),\ \ud \in H_0^1(\Lambda), \ \lambda \in H^{-\frac12}(\Gamma )$,
solutions of \eqref{eq:saddle-point}, where
\begin{align*}
a([u, \ud], [v, \vd])&= (u,v)_{H^1(\Omega)} + (\ud,\vd)_{H^1(\Lambda),|\D|}
\\
b([v, \vd], \mu)&= \langle \trace v - \ext \vd , \mu \rangle_\Gamma
\\
c([v,\vd])&= (f,v)_{L^2(\Omega)} + (\avrd{g},\vd)_{L^2(\Lambda),|\D|}
\\
d(\mu)&=0
\end{align*}
We prove that the hypothesis of \ref{th:bnb} are fullfilled choosing 
$X=H^1_0(\Omega) \times H^1_0(\Lambda)$, $Q=H^{-\frac 12}(\Gamma)$, where $X$  is equipped with the norm $\vertiii{[u,\ud ]}^2=\|u\|^2_{H^1(\Omega)} + \|\ud\|^2_{H^1(\Lambda),|\D|}$.
\begin{lemma}\label{lemma:prob1_boundedness} 
The bilinear forms $a(\cdot \ , \ \cdot)$ and $b(\cdot \ , \ \cdot)$ are bounded.
\end{lemma}
\begin{proof}
The bilinear form $a(\cdot \ , \ \cdot)$ is clearly bounded since
\begin{equation*}
a([u, \ud], [v, \vd])\leq \|u\|_{H^1(\Omega)}\|v\|_{H^1(\Omega)} + \|\ud\|_{H^1(\Lambda), |\D|}\|\vd\|_{H^1(\Lambda), |\D|} \leq 2 \vertiii{[u, \ud]}\vertiii{[v, \vd]}.
\end{equation*}
Concerning the bilinear form $b(\cdot \ , \ \cdot)$ we have
\begin{multline*}
b([v, \vd], \mu)= \langle \trace v - \ext \vd , \mu \rangle_\Gamma 
\leq \|\trace v - \ext \vd\|_{H^{\frac 12}_{00}(\Gamma)}\|\mu\|_{H^{-\frac 12}(\Gamma)}\\
\leq \left(\|\trace v\|_{H^{\frac 12}_{00}(\Gamma)} + \|\ext \vd\|_{H^{\frac 12}_{00}(\Gamma)}\right)\|\mu\|_{H^{-\frac 12}(\Gamma)}
\leq \left(C_T \|v\|_{H^1(\Omega)} + \|\ext \vd\|_{H^1(\Gamma)}\right)\|\mu\|_{H^{-\frac 12}(\Gamma)}\\
\leq \left(C_T \|v\|_{H^1(\Omega)} + \left(\frac{\max |\DD|}{\min |\D|}\right)^{\frac 12} \|\vd\|_{H^1(\Lambda),|\D|}\right)\|\mu\|_{H^{-\frac 12}(\Gamma)}\\
\leq \left( C_T + \left(\frac{\max |\DD|}{\min |\D|}\right)^{\frac 12}\right) \vertiii{[v,\vd]}\|\mu\|_{H^{-\frac 12}(\Gamma)}
\end{multline*}
\end{proof}

\begin{lemma}\label{lemma:prob1_coercivity}
The bilinear form $a(\cdot \ , \ \cdot)$ is coercive .
\end{lemma}
\begin{proof}
We have
$a([u,\ud], [u,\ud])= (u,u)_{H^1(\Omega)} + |{\cal D}|(\ud,\ud)_{H^1(\Lambda)} = \vertiii{[u,\ud]}^2\,.$
\end{proof}

We denote with $\mathcal{H}_\Omega$ the harmonic extension operator defined as $\mathcal{H}_\Omega \xi = v$ where $v$ is the weak solution of
\begin{equation}\label{eq:harmonic}
-\Delta v=0 \ \text{in} \ \Omega,
\quad
v=0 \ \text{on} \ \partial \Omega,
\quad
v=\xi \ \text{on} \ \Gamma. 
\end{equation}
The boundedness/stability of this operator ensures that there exists $\| \mathcal{H}_\Omega \| \in \mathbb{R}$ such that
$\|v \|_{H^1(\Omega)}\leq \|\mathcal{H}_\Omega\| \|\xi \|_{H^{\frac 12}_{00}(\Gamma)}$. 

\begin{lemma}
The inf-sup inequality \eqref{eq:infsup} is fulfilled, namely $\exists \beta_1 >0$ such that $\forall \mu \in H^{-\frac 12}(\Gamma)$:
\begin{equation*}
\sup _{\substack{v\in H^1_0(\Omega),\\ \vd \in H^1_0(\Lambda)}} \frac{ \langle \trace v  - \ext \vd, \mu \rangle_\Gamma}{\vertiii{[v, \vd]}}
\geq \beta_1 \sup_{q\in H^{\frac 12}_{00}(\Gamma)}\frac{\langle q, \mu\rangle_{\Gamma}}{\|q\|_{H^{\frac 12}_{00}(\Gamma)}}.
\end{equation*}
\end{lemma} 
\begin{proof}
We choose $\vd \in H^1_0(\Lambda)$ such that $\ext\vd =0$. Therefore,
\begin{equation*}
\sup _{\substack{v\in H^1_0(\Omega),\\ \vd \in H^1_0(\Lambda)}} \frac{ \langle \trace v  - \ext \vd, \mu\rangle_\Gamma}{\vertiii{[v, \vd]}} 
\geq \sup _{v\in H^1_0(\Omega)} \frac{ \langle \trace v, \mu \rangle_\Gamma}{\|v\|_{H^1(\Omega)}}.
\end{equation*}
We notice that the trace operator is surjective from $H^1_0(\Omega)$ to $H^{\frac12}_{00}(\Gamma)$. Indeed, $\forall \xi \in H^{\frac 12}_{00}(\Gamma)$, 
we  can find $v=\mathcal{H}_\Omega \xi$ solution of \eqref{eq:harmonic}. Using the stability of the harmonic extension we obtain
\begin{equation}\label{infsup_traceop}
\sup _{v\in H^1_0(\Omega)} \frac{ \langle \trace v, \mu \rangle_\Gamma}{\|v\|_{H^1(\Omega)}}
\geq  \sup _{\xi \in H^{\frac 12}_{00}(\Gamma )} \frac{ \langle \xi , \mu \rangle_\Gamma}{\|\mathcal{H}_\Omega\| \|\xi\|_{H^{\frac 12}_{00}(\Gamma)}}
= \|\mathcal{H}_\Omega\|^{-1} \|\mu\|_{H^{-\frac 12}(\Gamma)},
\end{equation}
where in the last inequality we exploited the fact that $H^{-\frac 12}(\Gamma)=(H^{\frac 12 }_{00}(\Gamma))^*$. 
\end{proof}


% >>>>>>>>>>>>>>>>>>>>>>>>>>>>>>>>>>>>>>>>>>>>>>>>>>>>>>>>>>>>>>>>>>>>>>>>>>>>>>>>>>>>>>>>>>>>>>>>>>>>
\subsection{Problem 2}
We aim to find $u \in H^1_0(\Omega),\ \ud \in H^1_0(\Lambda), \ \ld \in H^{-\frac12}(\Lambda)$,
solution of \eqref{eq:saddle-point} with
\begin{align*}
a([u, \ud], [v, \vd])&= (u,v)_{H^1(\Omega)} + (\ud,\vd)_{H^1(\Lambda),|\D|}
\\
b([v, \vd], \md)&=  \langle  \mtrace v - \vd, \md \rangle_{\Lambda, |\DD|} 
\\
c([v,\vd])&= (f,v)_{L^2(\Omega)} + (\avrd{g},\vd)_{L^2(\Lambda),|\D|}
\\
d(\md)&=0
\end{align*}

We prove that the hypotesis of  Theorem \ref{th:bnb} are fulfilled with the following spaces $X=H^1_0(\Omega) \times H^1_0(\Lambda)$, $Q=H^{-\frac 12}(\Lambda)$.
Let us consider $X$ equipped again with the norm $\vertiii{[\cdot,\cdot ]}$ and  
$Q$ equipped with the norm $\|\cdot \|_{H^{-\frac 12}(\Lambda),|\DD|}$.
Then, we have the following lemmas.
\begin{lemma}
The bilinear forms $a(\cdot \ , \ \cdot)$ and $b(\cdot \ , \ \cdot)$ are bounded.
\end{lemma}
\begin{proof}
The boundedness of $a(\cdot \ , \ \cdot)$ can be proved as in Lemma \ref{lemma:prob1_boundedness}. Concerning $b(\cdot \ , \ \cdot)$, we have
\begin{multline*}
b([v, \vd], \md)=  \langle  \mtrace v - \vd, \md \rangle_{\Lambda, |\DD|} 
\leq \|\mtrace v - \vd\|_{H^{\frac 12 }_{00}(\Lambda), |\DD|}\|\md\|_{H^{-\frac 12}(\Lambda),|\DD|}\\
\leq \left(\|\mtrace v\|_{H^{\frac 12 }_{00}(\Lambda), |\DD|}+\|\vd\|_{H^{\frac 12 }_{00}(\Lambda), |\DD|}\right)\|\md\|_{H^{-\frac 12}(\Lambda),|\DD|}\\
\leq \left(C_\Gamma \|\trace v\|_{H^{\frac 12 }_{00}(\Gamma)}+ \|\vd\|_{H^1(\Lambda), |\DD|}\right)\|\md\|_{H^{-\frac 12}(\Lambda),|\DD|} \\
\leq \left(C_\Gamma C_T\|v\|_{H^1(\Omega)}+ \left(\frac{\max |\DD|}{\min |\D|}\right)^{\frac 12} \|\vd\|_{H^1(\Lambda), |\D|}\right) \|\md\|_{H^{-\frac 12}(\Lambda),|\DD|} \\
\leq  \left( C_\Gamma C_T + \left(\frac{\max |\DD|}{\min |\D|}\right)^{\frac 12} \right) \vertiii{[v,\vd]}\|\md\|_{H^{-\frac 12}(\Lambda),|\DD|}.
\end{multline*} 
\end{proof}

\begin{lemma}\label{lemma:prob2_coercivity}
The bilinear form $a(\cdot \ , \ \cdot)$ is coercive.
\end{lemma}

\begin{lemma}
The inf-sup inequality \eqref{eq:infsup} holds, namely $ \exists \beta_2 >0$ such that $\forall \md \in H^{-\frac 12}(\Lambda),\,$:
\begin{equation*}
\sup _{\substack{v\in H^1_0(\Omega),\\ \vd \in H^1_0(\Lambda)}} \frac{ \langle \mtrace v - \vd, \md \rangle_{\Lambda,|\DD|}}{\vertiii{[v,\vd]}}
\geq \beta_2 \|\md\|_{H^{-\frac 12}(\Lambda)}.
\end{equation*}
\end{lemma} 
We choose $\vd=0$ and we obtain
\begin{equation*}
\sup _{\substack{v\in H^1_0(\Omega),\\ \vd \in H^1_0(\Lambda)}} \frac{ \langle \mtrace v - \vd, \md \rangle_{\Lambda,|\DD|}}{\vertiii{[v,\vd]}}
\geq \sup _{v\in H^1_0(\Omega)} \frac{ \langle \mtrace v, \md \rangle_{\Lambda,|\DD|}}{\|v\|_{H^1(\Omega)}}. 
\end{equation*}

For any $q \in H^{\frac 12}_{00}(\Lambda)$, we consider its uniform extension to $\Gamma$ named as $\ext q$
and then we consider the harmonic extension $v=\mathcal{H}_\Omega \ext q\in H^1_0(\Omega)$. It follows that $\mtrace v=q$. Therefore, 
\begin{equation*}
\sup _{v\in H^1_0(\Omega)}  \langle \mtrace v, \md \rangle_{\Lambda,|\DD|} \geq\sup_{q \in H^{\frac 12}_{00}(\Lambda)} \langle q, \md  \rangle_{\Lambda,|\DD|}\,.
\end{equation*}
Moreover, using Lemma \ref{lemma:H12norm} we obtain
\begin{equation*}
\|v\|_{H^1_0(\Omega)}\leq \|\mathcal{H}_\Omega\| \|\ext q\|_{H^{\frac 12}_{00}(\Gamma)}  = \|\mathcal{H}_\Omega\| \|q\|_{H^{\frac 12}_{00}(\Lambda),|\DD|}.
\end{equation*}
 Therefore,
\begin{multline*}
\sup _{v\in H^1_0(\Omega)} \frac{ \langle \mtrace v, \md \rangle_{\Lambda,|\DD|}}{\|v\|_{H^1(\Omega)}}
\geq \sup _{q\in H^{\frac 12}_{00}(\Lambda)} \frac{ \langle q, \md \rangle_{\Lambda,|\DD|}}{\|v\|_{H^1(\Omega)}}
\geq \|\mathcal{H}_\Omega\|^{-1} \sup _{q\in H^{\frac 12}_{00}(\Lambda)} \frac{ \langle q, \md \rangle_{\Lambda,|\DD|}}{\|q\|_{H^{\frac 12}_{00}(\Lambda),|\DD|}} 
\\= \|\mathcal{H}_\Omega\|^{-1} \|\md\|_{H^{-\frac 12}(\Lambda),|\DD|}.
\end{multline*}



