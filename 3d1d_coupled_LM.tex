% SIAM Article Template
\documentclass[r]{siamart171218}
% Packages and macros go here
\usepackage[english]{babel}
\usepackage{amsmath}
\usepackage{amssymb}
\usepackage{amsfonts}
\usepackage{array}
\usepackage{graphicx}
\usepackage{epsfig}
\usepackage{float}
\usepackage{fullpage}
\usepackage{color}
\usepackage{enumitem}  
\usepackage{epstopdf}
\usepackage{stmaryrd}

% Title. If the supplement option is on, then "Supplementary Material"
% is automatically inserted before the title.
%\title{On the weak coupling of 3D and 1D second order elliptic problems}
\title{Coupling PDEs on 3D-1D domains with Lagrange multipliers}

% Authors: full names plus addresses.
\author{Miroslav Kuchta, Federica Laurino, Kent-Andre Mardal, Paolo Zunino,\thanks{Authors are listed in alphabetical order}}

\begin{document}

\maketitle

% REQUIRED
\begin{abstract}
These are personal notes written to keep track of the developments on this topic, to be kept confidential.
\end{abstract}

% REQUIRED
\begin{keywords}
elliptic problems, high dimensionality gap, essential coupling conditions, Lagrange multipliers
\end{keywords}

% REQUIRED
\begin{AMS}
n.a.
\end{AMS}

% >>>>>>>>>>>>>>>>>>>>>>>>>>>>>>>>>>>>>>>>>>>>>>>>>>>>>>>>>>>>>>>>>>>>>>>>>>>>>>>>>>>>>>>>>>>>>>>>>>>>
 
% definitions here

\def\ud{u_{\odot}}
\def\vd{v_{\odot}}
\def\ld{\lambda_{\odot}}
\def\md{\mu_{\odot}}
\def\lld{l_{\odot}}
\def\uf{u_{\ominus}}
\def\up{u_{\oplus}}
\def\eps{\epsilon}
\def\nn{\boldsymbol n}
\def\rr{\boldsymbol r}
\def\RR{\boldsymbol R}
\def\kk{\boldsymbol k}
\def\ss{\boldsymbol s}
\def\uu{\boldsymbol u}
\def\vv{\boldsymbol v}
\def\xx{\boldsymbol x}
\def\bu{\overline{u}}
\def\bv{\overline{v}}
\def\tu{\widetilde{u}}
\def\tv{\widetilde{v}}
\def\TT{\boldsymbol T}
\def\NN{\boldsymbol N}
\def\BB{\boldsymbol B}
\def\ttu{\widetilde{\widetilde{u}}}
\def\ttv{\widetilde{\widetilde{v}}}
\def\cv{\check{v}}
\def\mesh{{\cal T}^h}
\def\ball{{\cal B}}
\def\R{\mathbb{R}}
\def\D{\mathcal{D}}
\def\DD{\partial\mathcal{D}}
\def\trace{\overline{\mathcal{R}}}
\def\ext{\mathcal{E}}
\def\ide{\mathcal{I}}
\def\ii{\hat{\imath}}
\newcommand{\avrd}[1]{\overline{\overline{#1}}}
\newcommand{\avrc}[1]{\overline{#1}}
\newcommand{\refe}[1]{{#1}_{\mathrm{ref}}}

\newcommand{\vertiii}[1]{{\left\vert\kern-0.25ex\left\vert\kern-0.25ex\left\vert #1 
    \right\vert\kern-0.25ex\right\vert\kern-0.25ex\right\vert}}

\newtheorem{thm}{Theorem}[section]
\newtheorem{prop}{Property}[section]
\theoremstyle{remark}
\newtheorem{remark}{Remark}[section]
 
% >>>>>>>>>>>>>>>>>>>>>>>>>>>>>>>>>>>>>>>>>>>>>>>>>>>>>>>>>>>>>>>>>>>>>>>>>>>>>>>>>>>>>>>>>>>>>>>>>>>>
\section{Introduction}\label{sec:intro}

We address the geometrical configuration of the problem for a 3D coupled problem formulation based on from Dirichlet-Neumann interface conditions. Then, we apply a model reduction technique that transforms the problem into 3D-1D coupled PDEs. We develop and analyze a robust definition of the coupling operators form a 3D domain, $\Omega$, to 1D manifold, $\Lambda$, and vice versa. This is a non trivial objective because the standard trace operator form a domain $\Omega$ to a subset $\Lambda$ is not well posed if $\Lambda$ is a manifold of co-dimension two of $\Omega$.

% >>>>>>>>>>>>>>>>>>>>>>>>>>>>>>>>>>>>>>>>>>>>>>>>>>>>>>>>>>>>>>>>>>>>>>>>>>>>>>>>>>>>>>>>>>>>>>>>>>>>
\section{Problem setting}\label{sec:setting}

Let $\Omega \subset \mathbb{R}^3$ be a bounded, convex open set. Let $\Omega_\ominus$ be a generalized cylinder embedded into $\Omega$ and be $\Omega_\oplus = \Omega \setminus \overline{\Omega_\ominus}$ be the complementary set of the cylinder. We also introduce the set $\Lambda$, a 1D manifold that is the centerline of $\Omega_\ominus$. We define the arc-length coordinate along $\Lambda$, denoted by $s \in (0,S)$. We denote with $\mathcal{D}(s)$ and $\partial\mathcal{D}(s)$ a cross section of $\Omega_\ominus$ and its boundary, respectively. 
%In what follows, we assume for simplicity of notation that $\Sigma$ has a constant cross section, but this is not a restriction of the approach. 
We also assume that $\Omega_\ominus$ crosses $\Omega$ from side to side and we call $\Gamma$ the lateral (cylindrical) surface of $\Omega_\ominus$, while the upper and lower side faces of $\Omega_\ominus$ belong to $\partial\Omega$. We refer to Figure \ref{fig1} for an illustration of the notation. 

\begin{figure}
\begin{center}
\includegraphics[width=0.5\textwidth]{3D-1D-simple.png}
\end{center}
\caption{Geometrical setting of the problem}
\label{fig1}
\end{figure}

We consider the problem arising from \emph{Dirichlet-Neumann} conditions. 
It consists to find $\up,\uf$ s.t.:
\begin{subequations}\label{eq:dirneu}
\begin{align}
\label{eq:dirneu1}
- \Delta \up  + \up &= f  && \text{ in } \Omega_{\oplus},\\
\label{eq:dirneu2}
- \Delta \uf  + \uf &= g  && \text{ in } \Omega_\ominus,\\
\label{eq:dirneu3}
-\nabla \uf \cdot \nn_{\ominus} &= -\nabla \up \cdot \nn_{\ominus}  && \text{ on } \Gamma,\\
\label{eq:dirneu4}
\uf &= \up && \text{ on }  \Gamma,\\
\label{eq:dirneu5}
\up &= 0 && \text{ on } \partial \Omega\,.
\end{align}
\end{subequations}

The objective of this work is to derive and alalyze a simplified version of problem \eqref{eq:dirneu}, where the domain $\Omega_\ominus$ shrinks to its centerline $\Lambda$ and the corresponding partial differential equation is averaged on the cylinder cross section, namely $\mathcal{D}$. This new problem setting will be called the \emph{reduced} problem. Form the mathematical standpoint it is more challenging than \eqref{eq:dirneu}, because it involves the coupling of 3D/1D elliptic problems.
For the model reduction process, we decompose integrals as follows, for any sufficiently regular function $w$,
\begin{equation*}
\int_{\Omega_\ominus} w d\omega 
= \int_\Lambda \int_{\mathcal{D}} w d\sigma ds
= \int_\Lambda |\mathcal{D}|\avrd{w} ds\,,
\quad
\int_{\Gamma} w d\sigma 
= \int_\Lambda \int_{\partial\mathcal{D}} w d\gamma ds
= \int_\Lambda  |\partial \mathcal{D}| \avrc{w} ds\,,
\end{equation*}
where $\avrd{w}, \ \avrc{w}$ denote the following mean values respectively,
\begin{equation*}
\avrd{w} = |\mathcal{D}|^{-1} \int_{\mathcal{D}} w d\sigma\,,
\quad
\avrc{w} = |\partial\mathcal{D}|^{-1} \int_{\partial\mathcal{D}} w d\gamma\,.
\end{equation*}

We apply the model reduction approach at the level of the variational formulation.
We start from the variational formulation of problem \eqref{eq:dirneu}, 
that is to find $\up \in H^1_{\partial\Omega}(\Omega_\oplus), \ \uf \in H^1_{\partial\Omega_\ominus\setminus\Gamma}(\Omega_\ominus), \ \lambda \in H^{-\frac12}(\partial\Omega_\ominus)$ s.t.
\begin{subequations}\label{eq:weak_dirneu}
\begin{align}
&(u_\oplus,v_\oplus)_{H^1(\Omega_\oplus)} + (u_\ominus,v_\ominus)_{H^1(\Omega_\ominus)} 
+ \langle  v_\oplus - v_\ominus, \lambda \rangle_{H^{-\frac12}(\Gamma)} 
\\
\nonumber
&\qquad\qquad = (f,v_\oplus)_{L^2(\Omega_\oplus)} + (g,v_\ominus)_{L^2(\Omega_\ominus)}
\quad \forall v_\oplus \in H^1_{\partial\Omega}(\Omega_\oplus), \ v_\ominus \in H^1_{\partial\Omega_\ominus\setminus\Gamma}(\Omega_\ominus)
\\
& \langle u_\oplus - u_\ominus, \mu \rangle_{H^{-\frac12}(\Gamma)} = 0
\quad \forall  \mu \in H^{-\frac12}(\Gamma)\,,
\end{align}
\end{subequations}
where $\langle v, \mu \rangle_{H^{-\frac12}(\Gamma)}$ denotes the duality pairing between 
$ \mu \in H^{-\frac12}(\Gamma)$ and $v \in H^{\frac12}(\Gamma)$.
In this case, the additional variable $\lambda$ is equivalent to $\lambda  =  \nabla \uf \cdot \nn_\ominus$.

Using the averaging tools for the model reduction approach, we end up with two different formulations of a reduced problem
for the unknown $u$ defined on the entire 3D domain $\Omega$, coupled with the unknown $\ud$, defined on the 1D manifold $\Lambda$
and a Lagrange multiplier defined either on $\Gamma$ (problem 1) or on $\Lambda$ (problem 2). 
%In the reduced problem the multiplier assumes the following interpretation,
%\begin{equation*}
%L = - \frac{1}{|\partial{\cal D}|} \int_{\partial{\cal D}} \nabla \uf \cdot \nn_\ominus
%   = - \frac{1}{|\partial{\cal D}|} \int_{\partial{\cal D}} \lambda \,.
%\end{equation*}

%We consider two alternative formulations.
The scope of this work is to compare them, 
with the aim to determine which is the most suitable as a computational model based on 3D-1D coupled PDEs.

% >>>>>>>>>>>>>>>>>>>>>>>>>>>>>>>>>>>>>>>>>>>>>>>>>>>>>>>>>>>>>>>>>>>>>>>>>>>>>>>>>>>>>>>>>>>>>>>>>>>>
\subsection{Topological model reduction}
\subsubsection*{Model reduction of the problem on $\Omega_{\ominus}$}

We apply the averaging technique to equation \eqref{eq:dirneu2}. In particular, we consider an arbitrary portion $\mathcal{P}$ of the cylinder $\Omega_\ominus$, with lateral surface $\Gamma _{\mathcal{P}}$ and bounded by two perpendicular sections to $\Lambda$, namely $\mathcal{D}(s_1), \ \mathcal{D}(s_2)$ with $s_1<s_2$. We have,
\begin{multline*}
\int_{\mathcal{P}} -\Delta \uf + \uf d\omega =
-\int_{\partial \mathcal{P}} \nabla \uf \cdot \nn_{\ominus} \, d\sigma  + \int_{\mathcal{P}}\uf d\omega=
\\
 \int_{\mathcal{D}(s_1)} \partial_s \uf d\sigma -  \int_{\mathcal{D}(s_2)} \partial_s \uf d\sigma -  \int_{\Gamma_{\mathcal{P}}} \nabla \uf \cdot \nn_{\ominus} d\sigma +\int_{\mathcal{P}}\uf d\omega
\end{multline*}
By the fundamental theorem of integral calculus combined with the Reynolds transport Theorem, 
%where $d_s |\partial\mathcal{D}(s)|$ is the rate at which the cross section changes size,
being $\nu$ the normal component of the velocity of the boundary, 
we have,
%\begin{multline*}
%-\int_{\mathcal{D}(s_1)} \partial_s \uf d\sigma +  \int_{\mathcal{D}(s_2)} \partial_s \uf d\sigma 
%= \int_{s_1}^{s_2} d_s \int_{\mathcal{D}(s)}  \partial_s \uf d\sigma ds
%\\
%= \int_{s_1}^{s_2} \int_{\mathcal{D}(s)} \partial^2_{ss} \uf d\sigma ds + \int_{s_1}^{s_2} \int_{\partial\mathcal{D}(s)} \nu \partial_s \uf d\gamma ds
%\\
%= \int_{s_1}^{s_2} \left[ |\mathcal{D}(s)| d^2_{ss} \avrd{u}_{\ominus}  + |\partial\mathcal{D}(s)| \nu  d_s \avrc{u}_{\ominus} \right] ds.
%\end{multline*}

\begin{equation*}
\begin{split}
\int_{\mathcal{D}(s_1)} \partial_s \uf d\sigma -  \int_{\mathcal{D}(s_2)} \partial_s \uf d\sigma 
&= -\int_{s_1}^{s_2} d_s \int_{\mathcal{D}(s)}  \partial_s \uf d\sigma ds
\\
&= -\int_{s_1}^{s_2} d_{ss}^2 \int_{\D(s)} u_{\ominus} \, d\sigma \, ds +  \int_{s_1}^{s_2} d_s \left( \int_{\DD(s)} \nu u_{\ominus} \, d\gamma \right) ds\,,
\end{split}
\end{equation*}
and assuming that $\D(s)$  can not change shape, we have
\begin{equation*}
\begin{split}
-\int_{s_1}^{s_2} d_{ss}^2 \int_{\D(s)} u_{\ominus} \, d\sigma \, ds + \int_{s_1}^{s_2} d_s \left( \int_{\DD(s)} \nu u_{\ominus} \, d\gamma \right)\, ds
&=-\int_{s_1}^{s_2} \left[d_{ss}^2 (|\D(s)|\avrd{u}_{\ominus}) -  d_s(  \nu |\DD(s)| \avrc{u}_{\ominus} ) \right] \, ds
\\
&=-\int_{s_1}^{s_2} \left[d_{ss}^2 (|\D(s)|\avrd{u}_{\ominus}) -  d_s\left(d_s(|\D(s)|) \avrc{u}_{\ominus} \right) \right] \, ds.
\end{split}
\end{equation*}
Moreover, we have
\begin{equation*}
\int_{\Gamma_{\mathcal{P}}} \nabla \uf \cdot \nn_{\ominus} d\sigma =  \int_{\Gamma_{\mathcal{P}}} \lambda \, d\sigma
=  \int_{s_1}^{s_2} \int_{\partial\mathcal{D}(s)} \lambda d\gamma \, ds 
= \int_{s_1}^{s_2} |\DD| \avrc{\lambda} \,ds\,.
\end{equation*}
From the combination of all the above terms with the right hand side, we obtain that the solution $\uf$ of \eqref{eq:dirneu2} satisfies,
\begin{equation*}
\int_{s_1}^{s_2} \left[ 
-d_{ss}^2 (|\D(s)|\avrd{u}_{\ominus}) +  d_s\left(d_s(|\D(s)|) \avrc{u}_{\ominus}\right) + |\D(s)|\avrd{u}_{\ominus} 
- |\partial\mathcal{D}(s)| \avrc{\lambda} 
\right] \, ds = \int_{s_1}^{s_2} |\mathcal{D}(s)| \avrd{g}\, ds \,.
\end{equation*}
Since the choice of the points $s_1,s_2$ is arbitrary, we conclude that the following equation holds true,
\begin{equation}\label{eq:averaged_f}
-d_{ss}^2 (|\D(s)|\avrd{u}_{\ominus}) +  d_s\left(d_s(|\D(s)|) \avrc{u}_{\ominus}\right) + |\D(s)|\avrd{u}_{\ominus} 
- |\partial\mathcal{D}(s)| \avrc{\lambda}
= |\mathcal{D}(s)| \avrd{g} \quad \text{on} \ \Lambda\,,
\end{equation}
which is complemented by the following conditions at the boundary of $\Lambda$,
\begin{equation}\label{eq:averaged_f_boundary}
|\D(s)| d_s \avrd{u}_{\ominus} = 0, \quad d_s |\D(s)| = 0, \quad \text{on} \quad s=0,S.
\end{equation}
Then, we consider variational formulation of the averaged equation \eqref{eq:averaged_f}.
After multiplication by a test function $\vd \in H^1(\Lambda)$, integration on $\Lambda$ and suitable application of integration by parts, we obtain,
\begin{multline*}
\int_\Lambda d_s (|\D(s)| \avrd{u}_{\ominus} ) d_s \vd \, ds - d_s (|\D(s)| \avrd{u}_{\ominus} ) \vd |_{s=0}^{s=S}
- \int_\Lambda (d_s |\D(s)|)\avrc{u}_{\ominus} d_s \vd \, ds + (d_s |\D(s)|) \avrc{u}_{\ominus} \vd |_{s=0}^{s=S}
\\
+\int_\Lambda |\D(s)| \avrd{u_\ominus} \vd - \int_\Lambda |\DD(s)| \avrc{\lambda} \vd \, ds
= \int_\Lambda |\D(s)| \avrd{g} V\, ds\,.
\end{multline*}
Using boundary conditions, 
the identity $d_s (|\D(s)| \avrd{u}_{\ominus} ) = |\D(s)| d_s \avrd{u}_{\ominus} + d_s (|\D(s)|) \avrd{u}_{\ominus}$
and reminding that $d_s |\D(s)|)/|\DD(s)| = \nu$,
we obtain,
\begin{equation}\label{eq:averaged_f_weak}
(d_s \avrd{u}_{\ominus}, d_s \vd )_{\Lambda,|\D|} 
+ ( \nu (\avrd{u}_{\ominus} - \avrc{u}_{\ominus}), d_s \vd)_{\Lambda,|\DD|} +(\avrd{u_\ominus}, \vd)_{\Lambda, |\D|}
- (\avrc{\lambda},\vd)_{\Lambda,|\DD|}
= (\avrd{g},V)_{\Lambda,|\D|}\,.
\end{equation}
where we have introduced the following weighted inner product notation,
\begin{equation*}
(\ud,\vd)_{\Lambda,w} = \int_0^S w(s) \ud(s) \vd(s) ds\,.
\end{equation*}
Let us now formulate the modelling assumption that allows us to reduce equation \eqref{eq:averaged_f_weak} to a solvable one-dimensional (1D) model.
More precisely, we assume that:
\begin{description}
\item[A1] the function $\uf$ has a \emph{uniform profile} on each cross section $\mathcal{D}(s)$, namely $\uf(r,s,t) = \ud(s)$.
\end{description}
Therefore, observing that $\ud=\avrc{u}_{\ominus}=\avrd{u}_{\ominus}$, problem \eqref{eq:averaged_f_weak} consists to find $\ud \in H^1(\Lambda)$ such that
\begin{equation}\label{eq:1Ddirneu_weak}
 (d_s \ud, d_s \ud)_{\Lambda,|\mathcal{D}|} 
+(\ud, \vd)_{\Lambda, |\D|}
-(\avrc{\lambda},\vd)_{\Lambda, |\DD|}
=  (\avrd{g},\vd)_{\Lambda, |\mathcal{D}|}
\quad \forall \vd \in H^1(\Lambda)\,.
\end{equation}

\subsubsection*{Topological model reduction of the problem on $\Omega_{\oplus}$}
We focus here on the subproblem of \eqref{eq:dirneu1} related to $\Omega_{\oplus}$.
We multiply both sides of \eqref{eq:dirneu1} by a test function $v\in H^1_0(\Omega)$ and integrate on $\Omega_\oplus$. Integrating by parts and using boundary and interface conditions, we obtain
\begin{equation*}
\begin{split}
\int_{\Omega _{\oplus}}fv\,d\omega&=
\int _{\Omega _{\oplus}}\nabla \up \cdot\nabla v\,d\omega -\int _{\partial \Omega _{\oplus}}\nabla \up \cdot \nn_{\oplus} v\,d\sigma + \int_{\Omega_{\oplus}} \up v
\\
&=\int _{\Omega _{\oplus}}\nabla \up \cdot\nabla v\,d\Omega -\int_{\Gamma} \nabla \up \cdot \nn_{\oplus} v + \int_{\Omega_{\oplus}} \up v
\\
&=\int _{\Omega _{\oplus}}\nabla \up \cdot\nabla v\,d\Omega +\int_{\Gamma} \lambda v + \int_{\Omega_{\oplus}} \up v.
\end{split}
\end{equation*} 
Then, we make the following modelling assumptions:
\begin{description}
\item[A2] we identify the domain $\Omega_{\oplus}$ with the entire $\Omega$, 
and we correspondingly omit the subscript $\oplus$ to the functions defined on $\Omega_{\oplus}$,
namely
\begin{equation*}
\int_{\Omega_{\oplus}} u_\oplus\, d\omega \simeq \int_{\Omega} u\, d\omega\,.
\end{equation*}
\end{description}
Therefore, we obtain 
\begin{equation*}
(\nabla u ,\nabla v)_{\Omega} +(u,v)_{\Omega}+(\lambda, v)_{\Gamma}  =(f,v)_{\Omega}
\end{equation*}
and combining with \eqref{eq:1Ddirneu_weak} we obtain the first formulation of the reduced problem.

\subsubsection*{Problem 1 (3D-1D-3D)}
Let $\langle \cdot , \cdot \rangle_\Gamma$ denote the duality pairing between 
$H^\frac12_{00}(\Gamma)$ and $H^{-\frac12}(\Gamma)$. The problem consists to find $u \in H^1_0(\Omega),\ \ud \in H^1_0(\Lambda), \ \lambda \in H^{-\frac12}(\Gamma )$, such that
\begin{subequations}\label{eq:problem1}
\begin{align}
&(u,v)_{H^1(\Omega)} + (\ud,\vd)_{H^1(\Lambda),|\D|} 
+ \langle T v  - \mathcal{U}_E \vd, \lambda \rangle_\Gamma
\\
\nonumber
&\qquad\qquad= (f,v)_{L^2(\Omega)} +  (\avrd{g},\vd)_{L^2(\Lambda),|\D|}
\quad \forall v \in H^1_0(\Omega), \ \vd \in H^1(\Lambda)
\\
&   \langle T u - \mathcal{U}_E \ud , \mu \rangle_\Gamma = 0
\quad \forall \mu \in H^{-\frac12}(\Gamma)\,.
\end{align}
\end{subequations}
Here, $T:H^1_0(\Omega) \rightarrow H^{\frac 12}_{00}(\Gamma)$ denotes the trace operator on $\Gamma$ and $\mathcal{U}_E: H^1_0(\Lambda) \rightarrow H^1_{0}(\Gamma)$ denotes the uniform extension from $\Lambda$ to $\Gamma$. The idea is then to couple a 3D PDE with a 1D one, using a Lagrange multiplier space defined on a 2D surface that surrounds the 1D manifold.\\

%Here, $\Pi_1: H^1_0(\Omega) \rightarrow H^{\frac12}_{00}(\Gamma)$
%and $\Pi_2: H^1_0(\Lambda) \rightarrow H^{\frac12}_{00}(\Gamma)$
%and we remark that $\Sigma$ may be considered as a virtual surface
%not necessarily of the same size as the underlying physical structure that is modeled.  
%The $\Pi_1$ and $\Pi_2$ operators may be defined in terms of the averaging operators above, but may also be realized in terms of e.g. Green functions (I don't know if this is a good idea). Furthermore,  $\Gamma$ may be discretized in terms of facets neighboring $\Lambda$ and may as such not be represented as a separate structure in the implementation. 

% >>>>>>>>>>>>>>>>>>>>>>>>>>>>>>>>>>>>>>>>>>>>>>>>>>>>>>>>>>>>>>>>>>>>>>>>>>>>>>>>>>>>>>>>>>>>>>>>>>>>
Now, we apply a topological model reduction of the interface conditions, namely we go from a 3D-1D-3D to a 3D-1D-1D formulation. 
To this purpose, let us write the lagrange multiplier and the test functions on every cross section $\partial\mathcal{D}(s)$ as their average plus some fluctuation,
\begin{equation*}
\lambda=\avrc{\lambda}+\tilde{\lambda}, \qquad v=\avrc{v}+\tilde{v},
\quad \text{on} \ \partial\mathcal{D}(s)\,,
\end{equation*}
where $\avrc{\tilde{\lambda}}=\avrc{\tilde{v}}=0$. 
Therefore, using the coordinates system $(r,s,t)$ on $\Gamma$, we have
\begin{equation*}
\int_{\Gamma}\lambda v\, d\sigma
=\int _{\Lambda}  \int_{\partial\mathcal{D}(s)} (\avrc{\lambda}+\tilde{\lambda})(\avrc{v}+\tilde{v})d\gamma ds
= \int_{\Lambda}|\partial\mathcal{D}(s)| \avrc{\lambda}\avrc{v}\,ds+\int_{\Lambda}  \int_{\partial\mathcal{D}(s)} \tilde{\lambda}\tilde{v}d\gamma ds\,.
\end{equation*}

Then, we make the following modelling assumptions:
\begin{description}
\item[A3] we assume that the product of fluctuations is small, namely
\begin{equation*}
\int_{\partial\mathcal{D}(s)} \tilde{\lambda}\tilde{v} d\gamma \simeq 0\,
\end{equation*}
\end{description}
and we obtain 
\begin{equation*}
(\nabla u ,\nabla v)_{\Omega} +(u,v)_{\Omega}+(\avrc \lambda, \avrc v)_{\Lambda , |\DD|}  =(f,v)_{\Omega},
\end{equation*}
which, combined with \eqref{eq:1Ddirneu_weak} leads to the second formulation of the reduced problem.

\subsection{Problem 2 (3D-1D-1D)}
Let $\langle \cdot , \cdot \rangle_\Lambda$ denote the duality pairing between 
$H^\frac12_{00}(\Lambda)$ and $H^{-\frac12}(\Lambda)$.
The problem requires to find $u \in H^1_0(\Omega),\ \ud \in H^1_0(\Lambda), \ \ld \in H^{-\frac12}(\Lambda)$, such that
\begin{subequations}\label{eq:problem2}
\begin{align}
&(u,v)_{H^1(\Omega)} + (\ud,\vd)_{H^1(\Lambda),|\D|} 
+  \langle \avrc{T v} - \vd, \ld \rangle_{H^{-\frac12}(\Lambda), |\DD|} 
\\
\nonumber
&\qquad\qquad= (f,v)_{L^2(\Omega)} +  (\avrd{g},V)_{L^2(\Lambda),|\D|}
\quad \forall v \in H^1_0(\Omega), \ \vd \in H^1_0(\Lambda)
\\
&   \langle \avrc{T u} -   \ud, \md \rangle_{H^{-\frac12}(\Lambda),|\DD|} = 0
\quad \forall \md \in H^{-\frac12}(\Lambda)\,.
\end{align}
\end{subequations}
We notice that all the integrals of the reduced problem are well defined because $u,v \in H^1_0(\Omega)$, 
$Tu, \ Tv \in H^\frac12_{00}(\Gamma)$ and thus $\avrc{Tu},\avrc{Tv} \in H^\frac12_{00}(\Lambda)$, as shown in the following lemma.
%More precisely, $\Pi_1 : H^1_0(\Omega) \rightarrow H^\frac12_{00}(\Lambda$ such that
%$\Pi_1 u = \avrc{(u|_\Gamma)}$ is the combination between the trace on $\Gamma$ and the average on $\partial{\cal D}$.
%The operator $\Pi_2: H^1_0(\Lambda) \rightarrow H^\frac12_{00}(\Lambda)$ is the injection form $H^1_0(\Lambda)$ and $H^\frac12_{00}(\Lambda)$.

\begin{lemma}\label{lemma:H12norm}
{\color{red}TO DO: generalize to not uniform $\DD$} 
When $\Gamma$ is a cylinder, if $u\in H_{00}^{\frac 12}(\Gamma)$, then $\avrc{u}\in H_{00}^{\frac 12}(\Lambda)$. Moreover, if $u\in H^{\frac 12}_{00}(\Gamma)$ is constant on each cross section, namely $u(s,\theta)=u(s)$, then 
\begin{equation*}
\|u\|_{H^{\frac 12}_{00}(\Gamma)}=2\pi R \|u\|_{H^{\frac 12}_{00}(\Lambda)}.
\end{equation*}
\end{lemma}
\begin{proof}
Let us denote as $\phi _{ij}$ and $\rho _{ij}$, for $i=1,2,\dots$, $j=0,1,\dots$, the eigenfunctions and the eigenvalues of the laplacian on $\Gamma$, and with $\phi _i$ and $\rho _i$ the eigenfunctions and the eigenvalues of the laplacian on $\Lambda$. In particular,
\begin{align*}
\phi _{ij}(s,\theta)=sin (i\pi s)\left( cos(j\theta)+ sin(j\theta) \right),\\
\rho_{ij}=i\pi ^2+\frac{j^2}{R^2},\\
\phi _{i}(s)=sin (i\pi s),\\
\rho _i = i\pi ^2.
\end{align*}
It is easy to verify that 
\begin{eqnarray}
\label{null_int_eigenf}
\int_0^{2\pi} \phi _{ij}(s,\theta)=0 \quad \forall j>0, \forall i \\
\label{nonull_int_eigenf}
\int_0^{2\pi} \phi _{ij}(s,\theta)= 2\pi R \, \sin(i \pi s) \quad \mbox{if } j=0, \forall i  .\\
\end{eqnarray}
Moreover we recall that $\phi_{i,j}(s,\theta)$ and $\phi _i(s)$ are orthogonal basis of $L^2(\Gamma)$ and $L^2(\Lambda)$ respectively. Therefore,
\begin{multline*}
\avrc{u}(s)=\frac{1}{2\pi R}\int_0^{2\pi} u(s,\theta)R\, d\theta
= \frac{1}{2\pi R}\int_0^{2\pi} \sum_{i,j} a_{i,j} \phi_{i,j}(s,\theta) R\, d\theta
\\= \frac{1}{2\pi R}\sum_{i,j} a_{i,j}\int_0^{2\pi}  \phi_{i,j}(s,\theta) R\, d\theta
=  \sum_{i} a_{i,0} \phi_{i}(s).
\end{multline*}
From \cite[Lemma 4.11]{chandler2015interpolation} we have
\begin{equation}\label{H12norm_Gamma}
\|u\|^2_{H^{\frac 12}(\Gamma)}=\sum_{i=1}^{\infty}\sum_{j=0}^{\infty} \left( 1+ \rho_{ij}\right)^{\frac 12}|a_{ij}|^2,
\text{ with }
a_{ij}=\int _0^1\int _0^{2\pi} u(s,\theta )\phi_{ij}\, R d\theta ds.
\end{equation}
and 
\begin{equation*}
\|\avrc{u}\|^2_{H^{\frac 12}(\Lambda)}=\sum_{i=1}^{\infty} \left( 1+ \rho_{i}\right)^{\frac 12}|\avrc{a}_i|^2,
\text{ with }
\avrc{a}_i=\int _0^1 \avrc{u}(s )\phi_{i}(s) ds.
\end{equation*}

Therefore, we have
\begin{multline*}
\|\avrc{u}\|^2_{H^{\frac 12}(\Lambda)}=
\sum_{i=1}^{\infty}\left( 1+ i^2\pi^2\right)^{\frac 12}\left( \int_0^1 \avrc{u}(s) sin(i\pi s)\, ds \right)^2\\
= \sum_{i=1}^{\infty} \left( 1+ i^2\pi^2\right)^{\frac 12}\left( \sum_{j=1}^\infty a_{j,0}\int_0^1 \sin(j\pi s) \sin(i\pi s) \, ds  \right)^2\\
= \sum_{i=1}^{\infty} \frac 14 \left( 1+ i^2\pi^2\right)^{\frac 12}a_{i,0}^2\\
\leq \sum_{i=1}^{\infty} \sum_{j=1}^{\infty}  \left( 1+ i^2\pi^2 + \frac{j^2}{R^2}\right)^{\frac 12} |a_{i,j}|^2 =\|u\|^2_{H^{\frac 12}(\Gamma)},
\end{multline*}

where we have used the fact that
\begin{eqnarray*}
&\int_0^1 \sin(i\pi s) \sin(j\pi s)\, ds=0 \quad \text{if $i\neq j$}\\
&\int_0^1 \sin(i\pi s) \sin(j\pi s)\, ds=\frac 12 \quad \text{if $i =j$}.
\end{eqnarray*}
Moreover, in the case in which $u$ is constant on each cross section, from \eqref{H12norm_Gamma} we have
\begin{multline*}
\|u\|^2_{H^{\frac 12}(\Gamma)}=\sum_{i=1}^{\infty}\sum_{j=0}^{\infty} \left( 1+ \rho_{ij}\right)^{\frac 12}|a_{ij}|^2
=\sum_{i=1}^{\infty}\sum_{j=0}^{\infty} \left(  1+ i\pi ^2+\frac{j^2}{R^2}\right)^{\frac 12}\left( \int _0^1\int _0^{2\pi} u(s,\theta )\phi_{ij}\, R d\theta ds \right)^2\\
=\sum_{i=1}^{\infty}\sum_{j=0}^{\infty} \left(  1+ i\pi ^2+\frac{j^2}{R^2}\right)^{\frac 12}\left( \int _0^1 u(s) \int _0^{2\pi} \phi_{ij}\, R d\theta ds \right)^2,
\end{multline*}
and using \eqref{null_int_eigenf} and \eqref{nonull_int_eigenf}, we obtain
\begin{multline*}
\|u\|^2_{H^{\frac 12}(\Gamma)}=
\sum_{i=1}^{\infty}\left( 1+ i\pi ^2\right)^{\frac 12}\left(\int _0^1 u(s )sin (i\pi s) 2\pi R ds\right)^2\\
=4\pi ^2 R^2 \sum_{i=1}^{\infty}\left( 1+ \rho _i\right)^{\frac 12}|a_i|^2 = 4\pi ^2 R^2  \|u\|^2_{H^{\frac 12}(\Lambda)}.
\end{multline*}
\hspace*{0.9\textwidth} c.v.d.
\end{proof}


\begin{remark}
The results of \eqref{lemma:H12norm}  can be generalized to the case of a  different geometry of $\Gamma$, for example a parallelepiped.  
\end{remark}

It is apparent that problems \eqref{eq:weak_dirneu} and \eqref{eq:red_dirneu} share the same mathematical structure.
For this reason, the well-posedness of \eqref{eq:red_dirneu} can be studied in the framework of the classical theory of saddle point problems.

%Assuming now that 
%$\Pi_1: H^1(\Omega) \rightarrow H^{\frac12}(\Gamma)$ or $\Pi_2: H^1(\Gamma) \rightarrow H^{\frac12}(\Gamma)$ are isomorphisms in the sense that 
%\[
%\| \Pi_1 \|_{L(H^1(\Omega), H^{\frac12}(\Gamma))} \le C \mbox{, } 
%\| \Pi_1^{-1} \|_{L(H^{\frac12}(\Gamma), H^1(\Omega))} \le 1/c,  
%\]
%and 
%\[
%\| \Pi_2 \|_{L(H^1(\Omega), H^{\frac12}(\Gamma))} \le C \mbox{, } 
%\| \Pi_2^{-1} \|_{L(H^{\frac12}(\Gamma), H^1(\Omega))} \le 1/c .   
%\]
%We obtain the inf-sup condition as follows. Assume that $\Pi_1$ is an isomorphism then  
%we may show the inf-sup conditioning by letting $\Pi_1 \hat u = M $ and $ \hat \ud = 0$ (i.e., we need the lower bound only for one
%of the operators).  
%\begin{eqnarray}
%\sup_{u,\ud}\frac{
%\langle \Pi_1 u - \Pi_2 \ud , M \rangle_{(H^{\frac12}(\Gamma), H^{-\frac12}(\Gamma))} 
%}{(|u|_1^2 + |\ud|_1^2)^{1/2}} 
%\ge \\ 
%\frac{ \langle \Pi_1 \hat u - \Pi_2 \hat \ud , M \rangle_{(H^{\frac12}(\Gamma), H^{-\frac12}(\Gamma))} 
%}{(|\hat u|_1^2 + |\hat \ud|_1^2)^{1/2}}  \ge \\
%\frac{ \langle M , M \rangle_{(H^{\frac12}(\Gamma), H^{-\frac12}(\Gamma))} 
%}{|\Pi_1^{-1} M |_{H^1(\Omega})} \ge \\ 
%\frac{ \langle M , M \rangle_{(H^{\frac12}(\Gamma), H^{-\frac12}(\Gamma))} 
%}{| M |_{H^\frac12(\Gamma)}} \ge \\ 
%c | M |_{H^{-\frac12}(\Gamma)}  
%\end{eqnarray}
%I think the other Brezzi conditions also follows, but for the boundedness we
%would need that both $\Pi_1$ and $\Pi_2$ are bounded. Not sure I am 
%happy with the $\langle \cdot, \cdot \rangle$ notation and in particular not when we spesify the
%spaces. I would rather just say that $(\cdot, \cdot)$ denotes both the $L_2$ inner product and the
%duality pairing.  
%If we find it useful we may also  
%consider $\Pi_2: H^1(\Gamma) \rightarrow X(\Gamma)$ where $X$ may e.g. be $H^1$.  
%If so, $L, M$ in $ H^{-\frac12}(\Gamma) \cap X(\Gamma)$  

%>>>>>>>>>>>>>>>>>>>>>>>>>>>>>>>>>>>>>>>>>>>>>>>>>
%Analysis of the continuous problem 
\section{Saddle-point problem analysis}
Let $a: X \times X \rightarrow \mathbb{R}$ and $b: X\times Q \rightarrow \mathbb{R}$ be bounded bilinear forms. Let us consider the general saddle point problem of the form: find $u\in X$, $\lambda\in Q$ s.t.
\begin{eqnarray}\label{eq:saddle-point}
\begin{cases}
a(u,v)+b(v,\lambda)=c(v)\quad &\forall v\in X\\
b(u,\mu)=d(\mu) \quad &\forall \mu\in Q.
\end{cases}
\end{eqnarray}
We denote with $A$ and $B$ the operators associated to the bilinear forms $a$ and $b$, 
namely $A: X \longrightarrow X'$ with $\langle Au,v\rangle _{X',X} = a(u,v)$ and $B: X \longrightarrow Q'$ with $\langle Bv,\mu\rangle_{Q',Q} = b(v,\mu)$. Problem \eqref{eq:saddle-point} embraces problems 1 and 2 described before. For the analysis of such problems we apply the following general abstract theorem.
\begin{theorem}{\cite[Theorem 2.34]{MR2050138}}\label{th:bnb}
Problem \eqref{eq:saddle-point} is well posed iff 
\begin{eqnarray}\label{BNB1}
\begin{cases}
\exists \alpha >0 :\, \inf_{u\in ker(B)}\sup_{v\in ker(B)} \frac{a(u,v)}{\|u\|_{X}\|v\|_{X}}\geq \alpha\\
\forall v \in ker(B), \, \left( \forall u \in ker(B),\, a(u,v)=0 \right)\implies v=0.
\end{cases}
\end{eqnarray}
and 
\begin{equation}\label{eq:infsup}
\exists \beta >0:\,\inf_{\mu\in Q}\sup_{v\in X} \frac{b(v,\mu)}{\|v\|_{X}\|\mu\|_{Q}}\geq \beta .
\end{equation}
\end{theorem} 
Notice that if $a$ is coercive on $ker(B)$, \eqref{BNB1} is clearly fulfilled. \\

% >>>>>>>>>>>>>>>>>>>>>>>>>>>>>>>>>>>>>>>>>>>>>>>>>>>>>>>>>>>>>>>>>>>>>>>>>>>>>>>>>>>>>>>>>>>>>>>>>>>>
\subsection{Problem 1}
It consists to find $u \in H^1_0(\Omega),\ \ud \in H_0^1(\Lambda), \ \lambda \in H^{-\frac12}(\Gamma )$, %such that
%\begin{subequations}\label{eq:red_dirneu}
%\begin{align}
%&(u,v)_{H^1(\Omega)} + |{\cal D}| (\ud,\vd)_{H^1(\Lambda)} 
%+ \langle \Pi_1 v  - \Pi_2 \vd, L \rangle_\Gamma 
%\\
%\nonumber
%&\qquad\qquad= (f,v)_{L^2(\Omega)} + |{\cal D}|(\avrd{g},\vd)_{L^2(\Lambda)}
%\quad \forall v \in H^1_0(\Omega), \ \vd \in H^1_0(\Lambda)
%\\
%&   \langle \Pi_1 u - \Pi_2 \ud , M \rangle_\Gamma = 0
%\quad \forall M \in H^{-\frac12}(\Gamma)\,,
%\end{align}
%\end{subequations}

solutions of \eqref{eq:saddle-point}, where
\begin{equation*}
a([u, \ud], [v, \vd])= (u,v)_{H^1(\Omega)} + (\ud,\vd)_{H^1(\Lambda),|\D|}
\end{equation*} 
\begin{equation*}
b([v, \vd], \mu)= \langle \trace v - \ext \vd , \mu \rangle_\Gamma
\end{equation*} 
\begin{equation*}
c([v,\vd])= (f,v)_{L^2(\Omega)} + (\avrd{g},\vd)_{L^2(\Lambda),|\D|}
\end{equation*}
\begin{equation*}
d(\mu)=0
\end{equation*}

%Here, $\Pi_1: H^1_0(\Omega) \rightarrow H^{\frac12}_{00}(\Gamma)$ is the trace operator 
%while $\Pi_2$ is the uniform extension from $H^1_0(\Lambda)$ to  $H^{\frac12}_{00}(\Gamma)$. 
We prove that the hypothesis of \ref{th:bnb} are fullfilled choosing 
$X=H^1_0(\Omega) \times H^1_0(\Lambda)$, $Q=H^{-\frac 12}(\Gamma)$, where $X$  is equipped with the norm $\vertiii{[u,\ud ]}^2=\|u\|^2_{H^1(\Omega)} + \|\ud\|^2_{H^1(\Lambda),|\D|}$ and $Q$ equipped with the norm
\begin{equation*}
\|\mu \|_{H^{-\frac 12}(\Gamma)} := \sup_{q\in H^{\frac 12}_{00}(\Gamma)}\frac{\langle q, \mu\rangle_\Gamma}{\|q\|_{H^{\frac 12}(\Gamma)}}
\end{equation*}. 
\begin{lemma}\label{lemma:prob1_boundedness} 
The bilinear forms $a(\cdot \ , \ \cdot)$ and $b(\cdot \ , \ \cdot)$ are bounded.
\end{lemma}
\begin{proof}
The bilinear form $a(\cdot \ , \ \cdot)$ is clearly bounded since
\begin{equation*}
a([u, \ud], [v, \vd])\leq \|u\|_{H^1(\Omega)}\|v\|_{H^1(\Omega)} + \|\ud\|_{H^1(\Lambda), |\D|}\|\vd\|_{H^1(\Lambda), |\D|} \leq 2 \vertiii{[u, \ud]}\vertiii{[v, \vd]}.
\end{equation*}
Concerning the bilinear form $b(\cdot \ , \ \cdot)$ we have
\begin{multline*}
b([v, \vd], \mu)= \langle \trace v - \ext \vd , \mu \rangle_\Gamma 
\leq \|\trace v - \ext \vd\|_{H^{\frac 12}(\Gamma)}\|\mu\|_{H^{-\frac 12}(\Gamma)}\\
\leq \left(\|\trace v\|_{H^{\frac 12}(\Gamma)} + \|\ext \vd\|_{H^{\frac 12}(\Gamma)}\right)\|\mu\|_{H^{-\frac 12}(\Gamma)}
\leq \left(C_T \|v\|_{H^1(\Omega)} + \|\ext \vd\|_{H^1(\Gamma)}\right)\|\mu\|_{H^{-\frac 12}(\Gamma)}\\
\leq \left(C_T \|v\|_{H^1(\Omega)} + \left(\frac{\max |\DD|}{\min |\D|}\right)^{\frac 12} \|\vd\|_{H^1(\Lambda),|\D|}\right)\|\mu\|_{H^{-\frac 12}(\Gamma)}\\
\leq \left( C_T + \left(\frac{\max |\DD|}{\min |\D|}\right)^{\frac 12}\right) \vertiii{[v,\vd]}\|\mu\|_{H^{-\frac 12}(\Gamma)}
\end{multline*}
\end{proof}

\begin{lemma}\label{lemma:prob1_coercivity}
The bilinear form $a(\cdot \ , \ \cdot)$ is coercive .
\end{lemma}
\begin{proof}
 Indeed, we have,
\begin{equation*}
a([u,\ud], [u,\ud])= (u,u)_{H^1(\Omega)} + |{\cal D}|(\ud,\ud)_{H^1(\Lambda)} = \vertiii{[u,\ud]}^2\,.
\end{equation*}
\end{proof}
\begin{lemma}
The inf-sup inequality \eqref{eq:infsup} is fulfilled, namely $\exists \beta_1 >0$ such that $\forall \mu \in H^{-\frac 12}(\Gamma)$:
\begin{equation*}
\sup _{\substack{v\in H^1_0(\Omega),\\ \vd \in H^1_0(\Lambda)}} \frac{ \langle \trace v  - \ext \vd, \mu \rangle_\Gamma}{\vertiii{[v, \vd]}}
\geq \beta_1 \sup_{q\in H^{\frac 12}_{00}(\Gamma)}\frac{\langle q, \mu\rangle_{\Gamma}}{\|q\|_{H^{\frac 12}(\Gamma)}}.
\end{equation*}
\end{lemma} 
\begin{proof}
We choose $\vd \in H^1_0(\Lambda)$ such that $\ext\vd =0$. Therefore,
\begin{equation*}
\sup _{\substack{v\in H^1_0(\Omega),\\ \vd \in H^1_0(\Lambda)}} \frac{ \langle \trace v  - \ext \vd, \mu\rangle_\Gamma}{\vertiii{[v, \vd]}} 
\geq \sup _{v\in H^1_0(\Omega)} \frac{ \langle \trace v, \mu \rangle_\Gamma}{\|v\|_{H^1(\Omega)}}.
\end{equation*}

We notice that the trace operator is surjective from $H^1_0(\Omega)$ to $H^{\frac12}_{00}(\Gamma)$. Indeed, $\forall \xi \in H^{\frac 12}_{00}(\Gamma)$, we  can find $v$ solution of
\begin{eqnarray*}
-\Delta v&=0 \quad &\text{in }\Omega\\
v&=0 &\text{on }\partial \Omega\\
v&=\xi &\text{on } \Gamma. 
\end{eqnarray*}
We denote with $\mathcal{E}_\Omega$ the harmonic extension operator defined above.
The boundedness/stability of this operator ensures that there exists $\| \mathcal{E}_\Omega \| \in \mathbb{R}$ such that
$v=\mathcal{E}_\Omega(\xi) $ and $\|v \|_{H^1(\Omega)}\leq \|\mathcal{E}_\Omega\| \|\xi \|_{H^{\frac 12}(\Gamma)}$. 
Substituting in the previous inequalities we obtain
\begin{equation}\label{infsup_traceop}
\sup _{v\in H^1_0(\Omega)} \frac{ \langle \trace v, \mu \rangle_\Gamma}{\|v\|_{H^1(\Omega)}}
\geq  \sup _{\xi \in H^{\frac 12}_{00}(\Gamma )} \frac{ \langle \xi , \mu \rangle_\Gamma}{\|\mathcal{E}_\Omega\| \|\xi\|_{H^{\frac 12}(\Gamma)}}
= \|\mathcal{E}_\Omega\|^{-1} \|\mu\|_{H^{-\frac 12}(\Gamma)},
\end{equation}
where in the last inequality we exploited the fact that $H^{-\frac 12}(\Gamma)=(H^{\frac 12 }_{00}(\Gamma))^*$. 
\end{proof}


% >>>>>>>>>>>>>>>>>>>>>>>>>>>>>>>>>>>>>>>>>>>>>>>>>>>>>>>>>>>>>>>>>>>>>>>>>>>>>>>>>>>>>>>>>>>>>>>>>>>>
\subsection{Problem 2}
This problem requires to find $u \in H^1_0(\Omega),\ \ud \in H^1_0(\Lambda), \ \ld \in H^{-\frac12}(\Lambda)$, %such that
%\begin{subequations}\label{eq:red_dirneu}
%\begin{align}
%&(u,v)_{H^1(\Omega)} + |{\cal D}|(\ud,\vd)_{H^1(\Lambda)} 
%+ |\partial {\cal D}| \langle  \Pi_1 \vd - \Pi_2 v, L \rangle_\Lambda 
%\\
%\nonumber
%&\qquad\qquad= (f,v)_{L^2(\Omega)} + |{\cal D}| (\avrd{g},V)_{L^2(\Lambda)}
%\quad \forall v \in H^1_0(\Omega), \ \vd \in H^1(\Lambda)
%\\
%&  |\partial {\cal D}| \langle \Pi_1 \ud - \Pi_2 u, M \rangle_\Lambda = 0
%\quad \forall M \in H^{-\frac12}(\Lambda)\,.
%\end{align}
%\end{subequations}

solution of \eqref{eq:saddle-point} with
\begin{equation*}
a([u, \ud], [v, \vd])= (u,v)_{H^1(\Omega)} + (\ud,\vd)_{H^1(\Lambda),|\D|}
\end{equation*} 
\begin{equation*}
b([v, \vd], \md)=  \langle  \mtrace v - \vd, \md \rangle_{\Lambda, |\DD|} 
\end{equation*} 
\begin{equation*}
c([v,\vd])= (f,v)_{L^2(\Omega)} + (\avrd{g},\vd)_{L^2(\Lambda),|\D|}
\end{equation*}
\begin{equation*}
d(\md)=0
\end{equation*}

%Here, $\Pi_1: H^1_0(\Lambda)\rightarrow H^{\frac 12}_{00}(\Lambda)$ is the immersion operator and $\Pi_2: H^1_0(\Omega)\rightarrow H^{\frac 12}_{00}(\Lambda)$ is defined as the composition of the trace operator $T_{\Gamma}: H^1_0(\Omega) \rightarrow H^{\frac 12}_{00}(\Gamma)$ and the average operator $\bar{(\,)}:H^{\frac 12}_{00}(\Gamma) \rightarrow H^{\frac 12}_{00}(\Lambda)$, namely $\Pi_2= \bar{(\,)}\circ T_{\Gamma}$. 
%First of all we prove that if $u\in H^1_0(\Omega)$, than $\avrc{T u} \in H^{\frac 12}_{00}(\Lambda)$. In particular, from standard trace theory, we have that $T u\in H^{\frac 12}_{00}(\Gamma)$, therefore we have to prove that if $u \in H^{\frac 12 }_{00}(\Gamma)$ then $\avrc{u}\in H^{\frac 12}_{00}(\Lambda)$. 


%%%%%Analysis with a non-weighted norm for the multiplier: the constants depend on the min and max \DD
%We prove that the hypotesis of  Theorem \ref{th:bnb} are fulfilled with the following spaces $X=H^1_0(\Omega) \times H^1_0(\Lambda)$, $Q=H^{-\frac 12}(\Lambda)$.
%Let us consider $X$ equipped again with the norm $\vertiii{[\cdot,\cdot ]}$ and  
%$Q$ equipped with the norm $\|\cdot \|_{H^{-\frac 12}}$.
%Then, we have the following lemmas.
%\begin{lemma}
%The bilinear forms $a(\cdot \ , \ \cdot)$ and $b(\cdot \ , \ \cdot)$ are bounded.
%\end{lemma}
%\begin{proof}
%The boundedness of $a(\cdot \ , \ \cdot)$ can be proved as in Lemma \ref{lemma:prob1_boundedness}. Concerning $b(\cdot \ , \ \cdot)$, we have
%\begin{multline*}
%b([v, \vd], \md)=  \langle  \avrc{Tv} - \vd, \md \rangle_{\Lambda, |\DD|} 
%\leq \|\avrc{Tv} - \vd\|_{H^{\frac 12 }(\Lambda), |\DD|}\|\md\|_{H^{-\frac 12}(\Lambda)}\\
%\leq \left(\|\avrc{Tv}\|_{H^{\frac 12 }(\Lambda), |\DD|}+\|\vd\|_{H^{\frac 12 }(\Lambda), |\DD|}\right)\|\md\|_{H^{-\frac 12}(\Lambda)} \\
%\leq \left(\|Tv\|_{H^{\frac 12 }(\Gamma)}+ \|\vd\|_{H^1(\Lambda), |\DD|}\right)\|\md\|_{H^{-\frac 12}(\Lambda)} \\
%\leq \left(C_T\|v\|_{H^1(\Omega)}+ \left(\frac{\max |\D|}{\min |\D|}\right)^{\frac 12} \|\vd\|_{H^1(\Lambda), |\D|}\right) \|\md\|_{H^{-\frac 12}(\Lambda)} \\
%\lesssim \vertiii{[v,\vd]}\|\md\|_{H^{-\frac 12}(\Lambda)} 
%\end{multline*}
%{\color{red} check $\||\DD| \avrc{Tv}\|_{H^{\frac 12}(\Lambda)}\leq \|Tv\|_{H^{\frac 12}(\Gamma)}$} 
%\end{proof}
%
%\begin{lemma}\label{lemma:prob2_coercivity}
%The bilinear form $a(\cdot \ , \ \cdot)$ is coercive.
%\end{lemma}
%
%\begin{lemma}
%The inf-sup inequality \eqref{eq:infsup} holds, namely $ \exists \beta_2 >0$ such that $\forall \md \in H^{-\frac 12}(\Lambda),\,$:
%\begin{equation*}
%\sup _{\substack{v\in H^1_0(\Omega),\\ \vd \in H^1_0(\Lambda)}} \frac{ \langle \avrc{T v} - \vd, \md \rangle_{\Lambda,|\DD|}}{\vertiii{[v,\vd]}}
%\geq \beta_2 \|\md\|_{H^{\frac 12}(\Lambda)}.
%\end{equation*}
%\end{lemma} 
%We choose $\vd=0$ and we obtain
%\begin{equation*}
%\sup _{\substack{v\in H^1_0(\Omega),\\ \vd \in H^1_0(\Lambda)}} \frac{ \langle \avrc{T v} - \vd, \md \rangle_{\Lambda,|\DD|}}{\vertiii{[v,\vd]}}
%\geq \sup _{v\in H^1_0(\Omega)} \frac{ \langle \avrc{T v}, \md \rangle_{\Lambda,|\DD|}}{\|v\|_{H^1(\Omega)}}. 
%\end{equation*}
%
%For any $q \in H^{\frac 12}_{00}(\Lambda)$, we consider its uniform extension to $\Gamma$ $\ext q$
%and then we consider the harmonic extension $v=\mathcal{E}(\ext q)\in H^1_0(\Omega)$. It follows that $\avrc{T v}=q$. Therefore, 
%\begin{equation*}
%\sup _{v\in H^1_0(\Omega)}  \langle \avrc{T v}, \md \rangle_{\Lambda,|\DD|} \gtrsim \sup_{q \in H^{\frac 12}_{00}(\Lambda)} \langle q, \md  \rangle_\Lambda\,.
%\end{equation*}
%Moreover, using Lemma \ref{lemma:H12norm} we obtain
%\begin{equation*}
%\|v\|_{H^1_0(\Omega)}\leq \|\mathcal{E}\| \|\ext q\|_{H^{\frac 12}(\Gamma)}  \lesssim \|\mathcal{E}\| \|q\|_{H^{\frac 12}(\Lambda)}.
%\end{equation*}
% Therefore,
%\begin{multline*}
%\sup _{v\in H^1_0(\Omega)} \frac{ \langle \avrc{T v}, \md \rangle_{\Lambda,|\DD|}}{\|v\|_{H^1(\Omega)}}
%\gtrsim \sup _{q\in H^{\frac 12}_{00}(\Lambda)} \frac{ \langle q, \md \rangle_\Lambda}{\|v\|_{H^1(\Omega)}}
%\gtrsim \|\mathcal{E}\|^{-1} \sup _{q\in H^{\frac 12}_{00}(\Lambda)} \frac{ \langle q, \md \rangle_\Lambda}{\|q\|_{H^{\frac 12}(\Lambda)}} 
%\\= \|\mathcal{E}\|^{-1} \|\md\|_{H^{-\frac 12}(\Lambda)}
%\end{multline*}
%and the constants in the inequalities depend on $\min _{s \in (0,S)} |\DD(s)|$ and $\max _{s \in (0,S)} |\DD(s)|$ which we suppose to be strictly positive.\\

%%%%Analysis with a weighted norm for the multiplier
We prove that the hypotesis of  Theorem \ref{th:bnb} are fulfilled with the following spaces $X=H^1_0(\Omega) \times H^1_0(\Lambda)$, $Q=H^{-\frac 12}(\Lambda)$.
Let us consider $X$ equipped again with the norm $\vertiii{[\cdot,\cdot ]}$ and  
$Q$ equipped with the norm $\|\cdot \|_{H^{-\frac 12}(\Lambda),|\DD|}$, defined as
\begin{equation*}
\|\md \|_{H^{-\frac 12}(\Lambda),|\DD|}:= \sup_{q\in H^{\frac 12}_{00}(\Lambda)}\frac{\langle q, \md\rangle_{\Lambda, |\DD|}}{\|q\|_{H^{\frac 12}(\Lambda),|\DD|}}
\end{equation*}
Then, we have the following lemmas.
\begin{lemma}
The bilinear forms $a(\cdot \ , \ \cdot)$ and $b(\cdot \ , \ \cdot)$ are bounded.
\end{lemma}
\begin{proof}
The boundedness of $a(\cdot \ , \ \cdot)$ can be proved as in Lemma \ref{lemma:prob1_boundedness}. Concerning $b(\cdot \ , \ \cdot)$, we have
\begin{multline*}
b([v, \vd], \md)=  \langle  \mtrace v - \vd, \md \rangle_{\Lambda, |\DD|} 
\leq \|\mtrace v - \vd\|_{H^{\frac 12 }(\Lambda), |\DD|}\|\md\|_{H^{-\frac 12}(\Lambda),|\DD|}\\
\leq \left(\|\mtrace v\|_{H^{\frac 12 }(\Lambda), |\DD|}+\|\vd\|_{H^{\frac 12 }(\Lambda), |\DD|}\right)\|\md\|_{H^{-\frac 12}(\Lambda),|\DD|} \\
\leq \left(C_\Gamma \|\trace v\|_{H^{\frac 12 }(\Gamma)}+ \|\vd\|_{H^1(\Lambda), |\DD|}\right)\|\md\|_{H^{-\frac 12}(\Lambda),|\DD|} \\
\leq \left(C_\Gamma C_T\|v\|_{H^1(\Omega)}+ \left(\frac{\max |\DD|}{\min |\D|}\right)^{\frac 12} \|\vd\|_{H^1(\Lambda), |\D|}\right) \|\md\|_{H^{-\frac 12}(\Lambda),|\DD|} \\
\leq  \left( C_\Gamma C_T + \left(\frac{\max |\DD|}{\min |\D|}\right)^{\frac 12} \right) \vertiii{[v,\vd]}\|\md\|_{H^{-\frac 12}(\Lambda),|\DD|}.
\end{multline*} 
\end{proof}

\begin{lemma}\label{lemma:prob2_coercivity}
The bilinear form $a(\cdot \ , \ \cdot)$ is coercive.
\end{lemma}

\begin{lemma}
The inf-sup inequality \eqref{eq:infsup} holds, namely $ \exists \beta_2 >0$ such that $\forall \md \in H^{-\frac 12}(\Lambda),\,$:
\begin{equation*}
\sup _{\substack{v\in H^1_0(\Omega),\\ \vd \in H^1_0(\Lambda)}} \frac{ \langle \mtrace v - \vd, \md \rangle_{\Lambda,|\DD|}}{\vertiii{[v,\vd]}}
\geq \beta_2 \|\md\|_{H^{\frac 12}(\Lambda)}.
\end{equation*}
\end{lemma} 
We choose $\vd=0$ and we obtain
\begin{equation*}
\sup _{\substack{v\in H^1_0(\Omega),\\ \vd \in H^1_0(\Lambda)}} \frac{ \langle \mtrace v - \vd, \md \rangle_{\Lambda,|\DD|}}{\vertiii{[v,\vd]}}
\geq \sup _{v\in H^1_0(\Omega)} \frac{ \langle \mtrace v, \md \rangle_{\Lambda,|\DD|}}{\|v\|_{H^1(\Omega)}}. 
\end{equation*}

For any $q \in H^{\frac 12}_{00}(\Lambda)$, we consider its uniform extension to $\Gamma$ named as $\ext q$
and then we consider the harmonic extension $v=\mathcal{E}_\Omega \ext q\in H^1_0(\Omega)$. It follows that $\mtrace v=q$. Therefore, 
\begin{equation*}
\sup _{v\in H^1_0(\Omega)}  \langle \mtrace v, \md \rangle_{\Lambda,|\DD|} \geq\sup_{q \in H^{\frac 12}_{00}(\Lambda)} \langle q, \md  \rangle_{\Lambda,|\DD|}\,.
\end{equation*}
Moreover, using Lemma \ref{lemma:H12norm} we obtain
\begin{equation*}
\|v\|_{H^1_0(\Omega)}\leq \|\mathcal{E}_\Omega\| \|\ext q\|_{H^{\frac 12}(\Gamma)}  = \|\mathcal{E}_\Omega\| \|q\|_{H^{\frac 12}(\Lambda),|\DD|}.
\end{equation*}
 Therefore,
\begin{multline*}
\sup _{v\in H^1_0(\Omega)} \frac{ \langle \mtrace v, \md \rangle_{\Lambda,|\DD|}}{\|v\|_{H^1(\Omega)}}
\geq \sup _{q\in H^{\frac 12}_{00}(\Lambda)} \frac{ \langle q, \md \rangle_{\Lambda,|\DD|}}{\|v\|_{H^1(\Omega)}}
\geq \|\mathcal{E}_\Omega\|^{-1} \sup _{q\in H^{\frac 12}_{00}(\Lambda)} \frac{ \langle q, \md \rangle_{\Lambda,|\DD|}}{\|q\|_{H^{\frac 12}(\Lambda),|\DD|}} 
\\= \|\mathcal{E}_\Omega\|^{-1} \|\md\|_{H^{-\frac 12}(\Lambda),|\DD|}.
\end{multline*}





%>>>>>>>>>>>>>>>>>>>>>>>>>>>>>>>>>>>>>>>>>>>>>>>>>
%Analysis of the discrete problem in the case in which the LM lives on the coarser mesh w.r.t the trace mesh deriving from the 3D mesh

%\input{infsup_discrete_coarserLM}

%>>>>>>>>>>>>>>>>>>>>>>>>>>>>>>>>>>>>>>>>>>>>>>>>>>
%Analysis of the discrete problem in the case in which the LM lives on the the trace mesh deriving from the 3D mesh (conforming to the interface cylinder \Gamma)

\section{Finite element approximation}
In this section we consider the discretization of Problem 1 and 2 by means of the finite element method. We address two main challenges; first we aim to identify a suitable approximation space for the Lagrange multiplier and to analyze the stability of the discrete saddle point problem; second we aim to derive a stable discretization method that uses indepent and conforming computational meshes for $\Omega$, $\Gamma$ and $\Lambda$. Let us introduce a shape-regular triangulation $\mathcal{T}^{\Omega}_h$ of $\Omega$ and an admissible partition $\mathcal{T}^{\Lambda}_{h}$ of $\Lambda$.
We analyze two different cases: the one in which the 3D mesh is conforming to the interface $\Gamma$, namely the set of the intersections of the 3D elements of $\mathcal{T}^{\Omega}_h$ with $\Gamma$ is constituted by facets of such elements, and the non conforming case, namely the interface $\Gamma$ cuts the mesh arbitrarly. The discrete equivalent of \eqref{eq:saddle-point} reads as finding $u_h\in X_h\subset X$, $\lambda_h\in Q_h\subset Q$ s.t.
\begin{eqnarray}\label{eq:saddle-point_discrete}
\begin{cases}
a(u_h,v_h)+b(v_h,\lambda_h)=c(v_h)\quad &\forall v_h\in X_h\\
b(u_h,\mu_h)=d(\mu_h) \quad &\forall \mu_h\in Q_h.
\end{cases}
\end{eqnarray}
Let $B_h: Q'_h \longrightarrow Q_h$ be the operator induced by $b$ such that $\langle B_h v_h,\mu _h\rangle_{Q'_h,Q_h} = b(v_h,\mu _h)$.
The well posedness of such problem is governed by the classical inf-sup theory in Banach spaces. The main result is reported below.

\begin{corollary}{\cite[Theorem 2.42]{MR2050138}}Let $a(\cdot, \cdot)$ and $b(\cdot, \cdot)$ be continuous bilinear forms. 
Problem \eqref{eq:saddle-point_discrete} is well-posed if and only if 
\begin{equation}\label{BNB1_discrete}
\exists \alpha_h >0 :\, \inf_{u_h\in ker(B_h)}\sup_{v_h\in ker(B_h)} \frac{a(u_h,v_h)}{\|u_h\|_{X}\|v_h\|_{X}}\geq \alpha_h
\end{equation}
and 
\begin{equation}\label{eq:infsup_discrete}
\exists \beta_h >0:\,\inf_{\mu_h\in Q_h}\sup_{v_h\in X_h} \frac{b(v_h,\mu_h)}{\|v_h\|_{X}\|\mu_h\|_{Q}}\geq \beta_h .
\end{equation}
\end{corollary}
This corollary is the discrete counterpart of Theorem \ref{th:bnb} where at the discrete level condition \eqref{BNB1_discrete} implies both of \eqref{BNB1}. Conversely, \eqref{eq:infsup_discrete} does not follow from the conformity of the finte element spaces and it must be analysed independetly of \eqref{eq:infsup}.  
Let us notice that for both problem 1 and problem 2 the bilinear form $a(\cdot, \cdot)$ is coercive as stated in Lemmas \eqref{lemma:prob1_coercivity} and \eqref{lemma:prob2_coercivity}. Consequently, \eqref{BNB1_discrete} is automatically satisfied, being $\alpha_h$ the coercivity constant.\\

\subsection{$\mathcal{T}^{\Omega}_h$ conforming to $\Gamma$}
 We first analyze the case in which the 3D mesh is conforming to the interface $\Gamma$. With this aim, we define conformity conditions between $\mathcal{T}^{\Omega}_h$ and $\mathcal{T}^{\Lambda}_h$ with $\Gamma$. More precisely we require that the intersection of $\mathcal{T}^{\Omega}_h$ and $\Gamma$ is made of entire faces of elemets $K \in \mathcal{T}^{\Omega}_h$. Furthermore, we also set a restriction between $\mathcal{T}^{\Omega}_h$ and $\mathcal{T}^{\Lambda}_h$. We assume that $\Lambda$ is a piecewise linear manifold. We want that the intersection of $\Gamma$ with any orthogonal plane to $\Lambda$ that crosses $\Lambda$ at the internal nodes of $\mathcal{T}^{\Lambda}_h$, consists of entire edges of $\mathcal{T}^{\Omega}_h$. Namely the intersection of $\Gamma$ with orthogonal planes to $\Lambda$ is conformal with $\mathcal{T}^{\Lambda}_h$.
\subsubsection{Problem 1}
We denote by $X_{h,0}^k(\Omega)\subset H^1_0(\Omega)$, with $k>0$, the conforming finite element space of continuous piecewise polynomials of degree $k$ defined on $\Omega$ satisfying homogeneous Dirichlet conditions on the boundary and by $X_{h,0}^k(\Lambda)\subset H^1_0(\Lambda)$ the space of continuous piecewise polynomials of degree $k$ defined on $\Lambda$, satisfying homogeneous Dirichlet conditions on $\Lambda \cap \partial \Omega$. 
%Moreover, $Q_{h}$ denotes a suitable trial space for the lagrange multiplier $\lambda_h$. In particular, $Q_h \subset H^{-\frac12}(\Gamma )$ in the case of Problem 1 and $Q_h \subset H^{-\frac12}(\Lambda)$ in the case of Problem 2.
Problem 1 consists to find $u_h \in X_{h,0}^k (\Omega) ,\, {\ud}_h \in X_{h,0}^k(\Lambda) ,\, \lambda_h \in Q_h \subset H^{-\frac12}(\Gamma )$, such that
\begin{subequations}
\begin{align}
&(u_h,v_h)_{H^1(\Omega)} + ({\ud}_h,{\vd}_h)_{H^1(\Lambda),|\D|} 
+ \langle \trace v_h  - \mathcal{E}_{\Lambda} {\vd}_h, \lambda_h \rangle_\Gamma 
\\
\nonumber
&\qquad\qquad= (f,v_h)_{L^2(\Omega)} + (\avrd{g},{\vd}_h)_{L^2(\Lambda),|\D|}
\quad \forall v_h \in X_{h,0}^k(\Omega), \ {\vd}_h \in X_{h,0}^k(\Lambda)
\\
&   \langle \trace u_h - \mathcal{E}_{\Lambda} {\ud}_h , \mu_h \rangle_\Gamma = 0
\quad \forall \mu_h \in Q_h\,,
\end{align}
\end{subequations}
The space $Q_h$ must be suitably chosen such that \eqref{eq:infsup_discrete} holds. Let $Q_h$ be the trace space of functions running in $X_{h,0}^k(\Omega)$, namely the space of continuous piecewise polynomials of degree $k$ defined on $\Gamma$ which satisfy homogeneous Dirichlet conditions on $\partial \Omega$.As a result, $Q_h=X_{h,0}^k(\Gamma)$. Therefore we impose homogeneous Dirichlet boundary condition on $\partial \Omega$ also for the Lagrange multiplier. For this choice of $Q_h$ we can prove the well-posedness of the discrete problem, as shown in the following. 

\begin{lemma}
Let $P_h: H^{\frac 12}_{00}(\Gamma) \longrightarrow Q_h$ be the orthogonal projection operator defined  for any $v \in H^{\frac 12}_{00}(\Gamma)$ by
\begin{equation*}
(P_h v , \psi_h)_\Gamma= (v, \psi_h)_\Gamma \qquad \forall \psi_h \in Q_h.  
\end{equation*} 
Then, $P_h$ is continuous on $H^{\frac 12}_{00}(\Gamma)$, namely
\begin{equation}\label{continuity_projoper}
\|P_h v\|_{H^{\frac 12}(\Gamma)} \leq C \|v\|_{H^{\frac 12}(\Gamma)},
\end{equation}
where $C$ is a positive constant independent of $h$.
\end{lemma}
\begin{proof}
We prove that $P_h$ is continuous on $L^2(\Gamma)$ and on $H^1_0(\Gamma)$ following \cite[Section 1.6.3]{MR2050138}.  Then, the inequality \eqref{continuity_projoper} can be dirived by Hilbertian interpolation. For the $L^2$-continuity, we exploit the fact that, from the definition of $P_h$,
\begin{equation*}
(v-P_h v,P_h v)_{\Gamma}=0.
\end{equation*} 
Therefore, by Pythagoras identity,
\begin{equation*}
\|v\|^2_{L^2(\Gamma)} = \|v-P_h v\|_{L^2(\Gamma)}^2 + \|P_h v\|_{L^2(\Gamma)}^2 \geq \|P_h v\|^2 _{L^2(\Gamma)}.
\end{equation*}
Let us now consider $v\in H^1_0(\Gamma)$.The Scott-Zhang interpolation operator $SZ_h$ from $H^1_0(\Gamma)$ to $Q_h$ satisfies the following inequalities,
\begin{equation}\label{SZ_stability}
\|SZ_h v\|_{H^1(\Gamma)} \leq C_1 \|v\|_{H^1(\Gamma)}
\end{equation} 
\begin{equation}\label{SZ_approx}
\|v -SZ_h v \|_{L^2(\Gamma)}\leq C_2 h \|v\|_{H^1(\Gamma)}.
\end{equation}
Therefore, using \eqref{SZ_stability},
\begin{equation*}
\begin{split}
\|\nabla P_h v\|_{L^2(\Gamma)} 
&\leq \|\nabla (P_h v - SZ_h v)\|_{L^2(\Gamma)} + \|\nabla SZ_h v\|_{L^2(\Gamma)}\\
&\leq  \|\nabla (P_h v - SZ_h v)\|_{L^2(\Gamma)} + C_1\|v\|_{H^1(\Gamma)}
\end{split}
\end{equation*}
and by using the inverse inequality we obtain
\begin{equation*}
\begin{split}
\|\nabla (P_h v - SZ_h v)\|_{L^2(\Gamma)} + C_1\|v\|_{H^1(\Gamma)}
&\leq \frac{C_3}{h} \|P_h v - SZ_h v\|_{L^2(\Gamma)} + C_1\|v\|_{H^1(\Gamma)}\\
&= \frac{C_3}{h} \|P_h (v - SZ_h v)\|_{L^2(\Gamma)} + C_1\|v\|_{H^1(\Gamma)}\\
&\leq  \text{ (Stability of $P_h$ in $L^2$) } \frac{C_3}{h} \|v - SZ_h v\|_{L^2(\Gamma)} + C_1\|v\|_{H^1(\Gamma)}\\
&\leq \text{ (using \eqref{SZ_approx}) } \frac{C_3}{h} C_2 h  \| v\|_{H^1(\Gamma)} + C_1\|v\|_{H^1(\Gamma)}\\
&\leq (C_2 C_3 +C_1) \|v\|_{H^1(\Gamma)},
\end{split}
\end{equation*}
from which we obtain the continuity in $H^1_0(\Gamma)$.\\
\end{proof}
\begin{lemma}\label{lemma:trspace_infsup} 
There exists a constant $\gamma >0$ such that for any $\mu_h\in Q_h$
\begin{equation*}
\sup_{\substack{q_h \in Q_h}} \frac{\langle q_h , \mu_h \rangle}{ \|q_h\|_{H^{\frac 12}(\Gamma)}} \geq \gamma \|\mu_h\|_{H^{-\frac 12}(\Gamma)}.
\end{equation*} 
\end{lemma}
\begin{proof}
Let $\mu_h$ be in $Q_h$. From the continuous case, in particular from \eqref{infsup_traceop}, we have
\begin{equation*}
\|\mathcal{E}_\Omega\|^{-1} \|\mu_h\|_{H^{-\frac 12}(\Gamma)} \leq \sup_{\substack{v \in H^1_0(\Omega)}} \frac{\langle \trace v , \mu_h \rangle}{\|v\|_{H^1(\Omega)}} 
\end{equation*}
and by the trace inequality $\|\trace v\|_{H^\frac 12 (\Gamma)} \leq C_T \|v\|_{H^1(\Omega)}$ (see \cite[7.56]{adams1975pure}), we obtain 
\begin{equation*}
\sup_{\substack{v \in H^1_0(\Omega)}} \frac{\langle \trace v , \mu_h \rangle}{\|v\|_{H^1(\Omega)}}
\leq C_T \sup_{\substack{v \in H^1_0(\Omega)}} \frac{\langle \trace v , \mu_h \rangle}{ \|\trace v\|_{H^{\frac 12}(\Gamma)}}.
\end{equation*}
By the definition of $P_h$ and \eqref{continuity_projoper} 
\begin{equation*}
\begin{split}
C_T \sup_{\substack{v \in H^1_0(\Omega)}} \frac{\langle \trace v , \mu_h \rangle}{ \|\trace v\|_{H^{\frac 12}(\Gamma)}}&= C_T \sup_{\substack{v \in H^1_0(\Omega)}} \frac{\langle P_h(\trace v) , \mu_h \rangle}{ \|\trace v\|_{H^{\frac 12}(\Gamma)}}\\
&\leq  C_T C \sup_{\substack{v \in H^1_0(\Omega)}} \frac{\langle P_h(\trace v) , \mu_h \rangle}{ \|P_h(\trace v)\|_{H^{\frac 12}(\Gamma)}}\\
&= C_T C \sup_{\substack{q_h \in Q_h}} \frac{\langle q_h , \mu_h \rangle}{  \|q_h\|_{H^{\frac 12}(\Gamma)}}.
\end{split}
\end{equation*}
\end{proof}

\begin{theorem}[Discrete inf-sup] The inequality \eqref{eq:infsup_discrete} holds, namely 
$\exists \beta_{h,1} >0$ s.t.
\begin{equation}\label{inf_sup_discrete_prob1}
\inf_{\mu_h \in Q_h} 
\sup_{\substack{v_h \in X_{h,0}^k(\Omega),\\ {\vd}_h \in X_{h,0}^k(\Lambda)}} \frac{ \langle \trace v_h - \ext {\vd}_h, \mu_h \rangle _{\Gamma}} {\vertiii{[v_h, {\vd} _h]} \|\mu_h\|_{H^{-\frac 12 }(\Gamma)}} 
\geq \beta_{h,1}. 
\end{equation}
\end{theorem}

\begin{proof}
Let $\mu_h \in Q_h$. As in the continuos case, we choose ${\vd}_h =0$ and we have
\begin{equation*}
\sup_{\substack{v_h \in X_{h,0}^k(\Omega),\\ {\vd}_h \in X_{h,0}^k(\Lambda)}} \frac{ \langle \trace v_h - \ext {\vd}_h, \mu_h \rangle _{\Gamma}} {\vertiii{[v_h,  {\vd}_h]}}
\geq \sup_{v_h \in X_{h,0}^k(\Omega)} \frac{ \langle \trace v_h, \mu_h \rangle _{\Gamma} } {\|v_h\|_{H^1(\Omega)}}.
\end{equation*}

Therefore, we want to prove that there exists $\beta_{h,1}$ such that
\begin{equation*}
\sup_{v_h \in X_{h,0}^k(\Omega)} \frac{ \langle \trace v_h, \mu_h \rangle _{\Gamma} } {\|v_h\|_{H^1(\Omega)}} \geq \beta_{h,1} \|\mu_h\|_{H^{-\frac 12}(\Gamma)} \qquad \forall \mu_h \in Q_h.
\end{equation*}

Using Lemma \ref{lemma:trspace_infsup} and the boundedness of the armonic extension operator $\mathcal{E}_{\Omega}$ from $H^{\frac 12}_{00}(\Gamma)$ to $H^1_0(\Omega)$ introduced in the previous section, we have
\begin{equation*}
\gamma \|\mu_h\|_{H^{-\frac 12}(\Gamma)} \leq  \sup_{q_h \in Q_h} \frac{ \langle q_h, \mu_h \rangle _{\Gamma} } {\|q_h\|_{H^{\frac 12}(\Gamma)}} 
\leq 
\|\mathcal{E}_{\Omega}\| \sup_{q_h \in Q_h} \frac{ \langle q_h, \mu_h \rangle _{\Gamma} } {\|\mathcal{E}_\Omega q_h\|_{H^1(\Omega)}} .
\end{equation*}
Let $R_h: H^1_0(\Omega) \rightarrow X_{h,0}^k(\Omega)$ be a quasi interpolation operator (such as the Scott-Zhang operator) satisfying 
\begin{equation*}
\|R_h v\|_{H^1(\Omega)} \leq C_R \|v\|_{H^1(\Omega)} \qquad \forall v \in H^1_0(\Omega).
\end{equation*}
Therefore, we obtain 
\begin{equation*}
\|\mathcal{E}_{\Omega}\| \sup_{q_h \in Q_h} \frac{ \langle q_h, \mu_h \rangle _{\Gamma} } {\|\mathcal{E}_{\Omega}q_h\|_{H^1(\Omega)}} 
\leq
\|\mathcal{E}_{\Omega}\| C_R \sup_{q_h \in Q_h} \frac{ \langle q_h, \mu_h \rangle _{\Gamma} } {\|R_h \mathcal{E}_{\Omega} q_h\|_{H^1(\Omega)}}
\end{equation*}
and we have
\begin{multline}\label{eq_conformity}
\gamma \|\mu_h\|_{H^{-\frac 12}(\Gamma)} 
\leq 
\sup_{q_h \in Q_h} \frac{ \langle q_h, \mu_h \rangle_{\Gamma} } {\|q_h\|_{H^{\frac 12}(\Gamma)}} 
\leq
\|\mathcal{E}_{\Omega}\| C_R \sup_{q_h \in Q_h} \frac{ \langle q_h, \mu_h \rangle_{\Gamma} } {\|R_h \mathcal{E}_{\Omega} q_h\|_{H^1(\Gamma)}}
\\
=
\|\mathcal{E}_{\Omega}\| C_R \sup_{q_h \in Q_h} \frac{ \langle \trace R_h  \mathcal{E}_{\Omega}q_h, \mu_h \rangle_{\Gamma} } {\|R_h \mathcal{E}_{\Omega} q_h\|_{H^1(\Omega)}} 
\leq \|\mathcal{E}_{\Omega}\| C_R \sup_{v_h \in X_{h,k}(\Omega)} \frac{ \langle \trace v_h, \mu_h \rangle_{\Gamma} } {\|v_h\|_{H^1(\Omega)}}. 
\end{multline}
Therefore the inf-sup condition $\eqref{inf_sup_discrete_prob1}$ holds with $\beta_{h,1} = \gamma \|\mathcal{E}_{\Omega}\|^{-1} C_R^{-1} $. We notice that in \eqref{eq_conformity} we exploit the fact that the operator $\trace R_h  \mathcal{E}_{\Omega}$ coincides with the identity on the space $Q_h$, thanks to the conformity of $\mathcal{T}^{\Omega}_h$ to the interface $\Gamma$.
\end{proof}

%\begin{remark} We notice that to prove the result in Lemma \ref{lemma:trspace_infsup} (and then the discrete inf-sup condition)  basically we need a projection operator $P_h: H^{\frac 12}_{00} \longrightarrow W_{h,0}^k(\Gamma)$ orthogonal in the multiplier space $Q_h$, namely such that $\langle P_h v, \mu_h \rangle = \langle v, \mu_h \rangle, \, \forall \mu_h \in Q_h$, and continuous in $H^{\frac 12}(\Gamma)$. Therefore, in principle different choices than $Q_h=W_{h,0}^k(\Gamma)$ could be considered if we can build an operator $P_h$ satisfying these properties. In \cite{belgacem1999mortar} such operator $P_h$  is built for a particular choice of $Q_h$ but it is not clear how to prove the $H^1$-stability inequality (and consequently the $H^{\frac 12 }$-stability) with a constant independent of the mesh size $h$.
%\end{remark}  

\subsubsection{Problem 2}
This problem requires to find  $u_h \in X_{h,0}^k(\Omega) ,\ {\ud}_h \in X_{h,0}^k(\Lambda), \ {\ld}_h \in Q_h \subset H^{-\frac12}(\Lambda)$, such that
\begin{subequations}
\begin{align}\label{eq:prob2_discrete}
&(u_h,v_h)_{H^1(\Omega)} + ({\ud}_h,{\vd}_h)_{H^1(\Lambda), |{\cal D}|} 
+  \langle  \mtrace v_h -  {\vd}_h, {\ld}_h \rangle_{\Lambda, |\DD|} 
\\
\nonumber
&\qquad\qquad= (f,v_h)_{L^2(\Omega)} + (\avrd{g},{\vd}_h)_{L^2(\Lambda), |{\cal D}|}
\quad \forall v_h \in X_h(\Omega), \ {\vd}_h \in X_h(\Lambda)
\\
&  \langle \mtrace u_h - {\ud}_h, {\md}_h \rangle_{\Lambda,| \DD| } = 0
\quad \forall {\md}_h \in Q_h\,.
\end{align}
\end{subequations}

%We introduce the space
%$W_{h,0}^k(\Lambda) \subset H^{\frac 12} _{00} (\Lambda)$, which is the averaged trace space of functions running in $X_{h,0}^k(\Omega)$. It coincides with the space of continuous piecewise polynomials of degree $k$ defined on $\Lambda$ and satisfying homogeneous Dirichlet boundary condition, namely $X_{h,0}^k(\Lambda)$. 
We choose $Q_h=X_{h,0}^k(\Lambda)$, therefore we impose homogeneous Dirichlet boundary condition on $\Lambda \cap \partial \Omega$ also for the Lagrange multiplier. With this choice for $Q_h$, we can prove the well-posedness of the discrete problem. In particular, following the same steps as for Problem 1, we can prove the following results.

%**** uniform \DD
%\begin{lemma}
%Let $P_h: H^{\frac 12}_{00}(\Lambda) \longrightarrow Q_h$ be the orthogonal projection operator defined  for any $v \in H^{\frac 12}_{00}(\Lambda)$ by
%\begin{equation*}
%(P_h v , \psi)_\Lambda= (v, \psi)_\Lambda \qquad \forall \psi \in Q_h.  
%\end{equation*} 
%Then, $P_h$ is continuous on $H^{\frac 12}_{00}(\Lambda)$, namely
%\begin{equation*}
%\|P_h v\|_{H^{\frac 12}_{00}(\Lambda)} \leq C \|v\|_{H^{\frac 12}_{00}(\Lambda)},
%\end{equation*}
%where $C$ is a positive constant independent of $h$.
%\end{lemma}
%
%\begin{lemma}\label{infsup_avr_trspace}
%There exist a constant $\gamma >0$ such that
%\begin{equation*}
%\sup_{\substack{q_h \in W_{h,0}^k(\Lambda)}} \frac{\langle q_h , \mu_h \rangle}{ \|q_h\|_{H^{\frac 12}(\Lambda)}} \geq \gamma \|\mu_h\|_{H^{-\frac 12}(\Lambda)} \qquad \forall \mu_h \in Q_h.
%\end{equation*} 
%\end{lemma}
%
%\begin{theorem}[Discrete inf-sup] The inequality \eqref{eq:infsup_discrete} holds, namely 
%$\exists \beta_{h,2} >0$ s.t.
%\begin{equation}
%\inf_{\mu_h \in Q_h} 
%\sup_{\substack{v_h \in X_{h,0}^k(\Omega),\\ {\vd}_h \in X_{h,0}^k(\Lambda)} }\frac{\langle \mtrace v_h -  {\vd}_h, {\md}_h \rangle _{\Lambda,|\DD|} } {\vertiii{[v_h, {\vd}_h]} \|{\md}_h\|_{H^{-\frac 12 }(\Lambda)} } 
%\geq \beta_{h,2}. 
%\end{equation}
%\end{theorem}
%\begin{proof}
%Let ${\md}_h$ be arbitrarly chosen in $Q_h$. Again, we choose ${\vd}_h =0$, so that the proof reduces to show that there exists $\beta_{h,2}$ such that
%\begin{equation*}
%\sup_{v_h \in X_{h,0}^k(\Omega)} \frac{ \langle \mtrace v_h , {\md}_h \rangle _{\Lambda,|\DD|} } {\|v_h\|_{H^1(\Omega)} }\geq \beta_{h,2} \|{\md}_h\|_{H^{-\frac 12}(\Lambda)}\ \qquad \forall {\md}_h \in Q_h.
%\end{equation*}
%Let us denote with $\ext$ the uniform extension operator from $\Lambda$ to $\Gamma$. Using Lemma \ref{H12norm_Gamma}, we easily have for any $w \in H^{\frac 12}(\Lambda)$,\\
%{\color{red} TO DO: generalize to non constant $\DD$}
%\begin{equation*}
%\|\mathcal{U}_E w\|_{H^{\frac 12}(\Gamma)}=|\DD| \|w\|_{H^{\frac 12}(\Lambda)}.
%\end{equation*}
%Consequently, from Lemma \ref{infsup_avr_trspace}, using again the extension operator $E$ from $H^{\frac 12}(\Gamma)$ to $H^1_0(\Omega)$ and the quasi interpolation operator $R_h$ from $H^1_0(\Omega)$ to $X_{h,0}^k(\Omega)$, we obtain
%\begin{multline}
%\gamma \|\mu_h\|_{H^{-\frac 12}(\Lambda)} \leq 
%\sup_{q_h \in W_{h,0}^k(\Lambda)} \frac{ \langle q_h, \mu_h \rangle_{\Lambda} } {\|q_h\|_{H^{\frac 12}(\Lambda)}} 
%\\
%= |\DD| \sup_{q_h \in W_{h,0}^k(\Lambda)} \frac{ \langle q_h, \mu_h \rangle _{\Lambda}} {\|\mathcal{U}_E q_h\|_{H^{\frac 12}(\Gamma)}} 
%\leq |\DD|\|E\| \sup_{q_h \in W_{h,0}^k(\Lambda)} \frac{ \langle q_h, \mu_h \rangle _{\Lambda} } {\|E \mathcal{U}_E q_h\|_{H^1(\Omega)}} 
%\\
%\leq |\DD|\|E\| C_R \sup_{q_h \in W_{h,0}^k(\Lambda)} \frac{ \langle q_h, \mu_h \rangle _{\Lambda} } {\|R_h E \mathcal{U}_E q_h\|_{H^1(\Omega)}}
%\\ 
%=  |\DD|\|E\| C_R \sup_{q_h \in W_{h,0}^k(\Lambda)} \frac{ \langle \Pi _1  R_h E \mathcal{U}_E q_h, \mu_h \rangle _{\Lambda}} {\|R_h E \mathcal{U}_E w_h\|_{H^1(\Omega)}}
%\\
%\leq |\DD|\|E\| C_R \sup_{v_h \in X_h(\Omega)} \frac{ \langle \Pi _2  v_h, \mu_h \rangle _{\Lambda}} {\|v_h\|_{H^1(\Omega)}}. 
%\end{multline}
%
%%\begin{equation*}
%%\|\mathcal{U}_E w\|_{H^{\frac 12}(\Gamma)}\lesssim \|w\|_{H^{\frac 12}(\Lambda)}.
%%\end{equation*}
%%Consequently, from Lemma \ref{infsup_avr_trspace}, using again the extension operator $\mathcal{E}$ from $H^{\frac 12}(\Gamma)$ to $H^1_0(\Omega)$ and the quasi interpolation operator $R_h$ from $H^1_0(\Omega)$ to $X_{h,0}^k(\Omega)$, we obtain
%%\begin{multline}
%%\gamma \|{\md}_h\|_{H^{-\frac 12}(\Lambda)} \leq 
%%\sup_{q_h \in Q_h} \frac{ \langle q_h, {\md}_h \rangle_{\Lambda} } {\|q_h\|_{H^{\frac 12}(\Lambda)}} 
%%\\
%%\lesssim \sup_{q_h \in Q_h} \frac{ \langle q_h, {\md}_h \rangle _{\Lambda}} {\|\mathcal{U}_E q_h\|_{H^{\frac 12}(\Gamma)}} 
%%\lesssim \|\mathcal{E}\| \sup_{q_h \in Q_h} \frac{ \langle q_h, {\md}_h \rangle _{\Lambda} } {\|\mathcal{E} \mathcal{U}_E q_h\|_{H^1(\Omega)}} 
%%\\
%%\lesssim \|\mathcal{E}\| C_R \sup_{q_h \in Q_h} \frac{ \langle q_h, {\md}_h \rangle _{\Lambda} } {\|R_h E \mathcal{U}_E q_h\|_{H^1(\Omega)}}
%%\\ 
%%\lesssim\|\mathcal{E}\| C_R \sup_{q_h \in Q_h} \frac{ \langle \avrc{TR_h E \mathcal{U}_E q_h}, {\md}_h \rangle _{\Lambda}} {\|R_h E \mathcal{U}_E q_h\|_{H^1(\Omega)}}
%%\\
%%\lesssim\|\mathcal{E}\| C_R \sup_{v_h \in X_h(\Omega)} \frac{ \langle \avrc{T v_h}, {\md}_h \rangle _{\Lambda}} {\|v_h\|_{H^1(\Omega)}}. 
%%\end{multline}
%
%\end{proof}
%


%*******************extension to non uniform $|\DD|$

\begin{lemma}\label{lemma:prob1_orthproj}
Let $P_h: H^{\frac 12}_{00}(\Lambda) \longrightarrow Q_h$ be the orthogonal projection operator defined  for any $v \in H^{\frac 12}_{00}(\Lambda)$ by
\begin{equation*}
(P_h v , \psi)_{\Lambda,|\DD|}= (v, \psi)_{\Lambda , |\DD|} \qquad \forall \psi \in Q_h.  
\end{equation*} 
Then, $P_h$ is continuous on $H^{\frac 12}_{00}(\Lambda)$, namely
\begin{equation*}
\|P_h v\|_{H^{\frac 12}(\Lambda),|\DD|} \leq C \|v\|_{H^{\frac 12}(\Lambda),|\DD|},
\end{equation*}
where $C$ is a positive constant independent of $h$.
\end{lemma}

\begin{lemma}\label{infsup_avr_trspace}
There exist a constant $\gamma >0$ such that
\begin{equation*}
\sup_{\substack{q_h \in Q_h}} \frac{\langle q_h , {\md}_h \rangle_{\Lambda, |\DD|}}{ \|q_h\|_{H^{\frac 12}(\Lambda),|\DD|}} \geq \gamma \|{\md}_h\|_{H^{-\frac 12}(\Lambda)} \qquad \forall {\md}_h \in Q_h.
\end{equation*} 
\end{lemma}
The proofs are equivalent ot the ones of Lemmas \ref{lemma:prob1_orthproj} and \ref{lemma:trspace_infsup}.
\begin{theorem}[Discrete inf-sup] The inequality \eqref{eq:infsup_discrete} holds, namely 
$\exists \beta_{h,2} >0$ s.t.
\begin{equation}
\inf_{\mu_h \in Q_h} 
\sup_{\substack{v_h \in X_{h,0}^k(\Omega),\\ {\vd}_h \in X_{h,0}^k(\Lambda)} }\frac{\langle \mtrace v_h -  {\vd}_h, {\md}_h \rangle _{\Lambda,|\DD|} } {\vertiii{[v_h, {\vd}_h]} \|{\md}_h\|_{H^{-\frac 12 }(\Lambda)} } 
\geq \beta_{h,2}. 
\end{equation}
\end{theorem}
\begin{proof}
Let ${\md}_h$ be arbitrarly chosen in $Q_h$. Again, we choose ${\vd}_h =0$, so that the proof reduces to show that there exists $\beta_{h,2}$ such that
\begin{equation*}
\sup_{v_h \in X_{h,0}^k(\Omega)} \frac{ \langle \mtrace v_h , {\md}_h \rangle _{\Lambda,|\DD|} } {\|v_h\|_{H^1(\Omega)} }\geq \beta_{h,2} \|{\md}_h\|_{H^{-\frac 12}(\Lambda)}\ \qquad \forall {\md}_h \in Q_h.
\end{equation*}
From Lemma \ref{lemma:H12norm} and its corollaries, for any $w \in H^{\frac 12}(\Lambda)$,
\begin{equation*}
\|\ext w\|_{H^{\frac 12}(\Gamma)}= \|w\|_{H^{\frac 12}(\Lambda),|\DD|}.
\end{equation*}

Consequently, from Lemma \ref{infsup_avr_trspace}, using again the extension operator $\mathcal{E}_{\Omega}$ from $H^{\frac 12}(\Gamma)$ to $H^1_0(\Omega)$ and the quasi interpolation operator $R_h$ from $H^1_0(\Omega)$ to $X_{h,0}^k(\Omega)$, we obtain
\begin{multline}
\gamma \|{\md}_h\|_{H^{-\frac 12}(\Lambda)} \leq 
\sup_{q_h \in Q_h} \frac{ \langle q_h, {\md}_h \rangle_{\Lambda,|\DD|} } {\|q_h\|_{H^{\frac 12}(\Lambda),|\DD|}} 
\\
=  \sup_{q_h \in Q_h} \frac{ \langle q_h, {\md}_h \rangle _{\Lambda,|\DD|}} {\|\ext q_h\|_{H^{\frac 12}(\Gamma)}} 
\leq \|\mathcal{E}_{\Omega}\| \sup_{q_h \in Q_h} \frac{ \langle q_h, {\md}_h \rangle _{\Lambda,|\DD|} } {\|\mathcal{E}_{\Omega} \ext q_h\|_{H^1(\Omega)}} 
\\
\leq \|\mathcal{E}_{\Omega}\| C_R \sup_{q_h \in Q_h} \frac{ \langle q_h, {\md}_h \rangle _{\Lambda,|\DD|} } {\|R_h \mathcal{E}_{\Omega} \ext q_h\|_{H^1(\Omega)}}
\\ 
=  \|\mathcal{E}_{\Omega}\| C_R \sup_{q_h \in Q_h} \frac{ \langle 	\mtrace R_h \mathcal{E}_{\Omega} \ext q_h, {\md}_h \rangle _{\Lambda,|\DD|}} {\|R_h \mathcal{E}_{\Omega} \ext w_h\|_{H^1(\Omega)}}
\\
\leq \|\mathcal{E}_{\Omega}\| C_R \sup_{v_h \in X_h(\Omega)} \frac{ \langle \mtrace v_h, {\md}_h \rangle _{\Lambda,|\DD|}} {\|v_h\|_{H^1(\Omega)}}. 
\end{multline}
Also in this case to prove the disrete inf-sup condition we exploit the conformity of the meshes on $\Omega$, $\Gamma$ and $\Lambda$ and the fact that the operator $\mtrace R_h \mathcal{E}_{\Omega} \ext$ coincides with the identity if applied to functions in $Q_h$. 
\end{proof}

%>>>>>>>>>>>>>>>>>>>>>>>>>>>>>>>>>>>>>>>>>>>>>>>>>>
%Analysis of the discrete problem in the case in which we have a non conforming 3D mesh (just in the case of formulation 2 3D-1D-1D)

\subsection{$\mathcal{T}^{\Omega}_h$ non conforming to $\Gamma$}
We analyze now the case in which the elements of the 3D mesh $\mathcal{T}^{\Omega}_h$ cut the interface $\Gamma$. It is easy to understand that the formulation of Problem 2 is more suitable. 
{\color{red} Add more details, we can refer also to cutFEM (Burman, Massing etc.) explaining the limitation of that approach (network case for example).}\\
Therefore we focus on the analysis of Problem 2.

\subsubsection{Problem 2} We consider for the solutions $u_h$ and ${\ud}_h$ the spaces $X^0_{h,1}(\Omega)$ and $X^0_{h,1}(\Lambda)$, see the previous subsection for the definition. Concerning the multiplier space, we make the following choice, $Q_h(\Lambda)=\{{\ld}_h : {\ld}_h \in P^0(K) \forall K \in \mathcal{T}^{\Lambda}_{h'}\}$, namely the multiplier lives on the same mesh used for the 1D solution ${\ud}_h$. Notice that in this case we suppose that the mesh sizes of the 3D mesh $	\mathcal{T}^{\Omega}_h$ and the 1D mesh $\mathcal{T}^{\Lambda}_{h'}$ are different, in particular we suppose the 1D mesh is finer. With this choice the problem is not inf-sup stable, therefore the idea is to add a stabilization term $s({\ld}_h, {\md}_h)$ to \eqref{eq:prob2_discrete} following the approach introduce in \cite{burman2014}. In particular, we build a new multiplier space $L_h(\Lambda)$ for which the discrete inf-sup condition is fulfilled and we build a projection operator $\pi_L: Q_h(\Lambda) \rightarrow L_h(\Lambda)$. Based on this projection operator, we build the stabilization term $s({\ld}_h, {\md}_h)$ and prove that $\forall [u_h, {\ud}_h]$, there exists $\xi_h([u_h, {\ud}_h]) \in Q_h(\Lambda)$ s.t.  
\begin{equation}\label{stab_coercivity}
a([u_h, {\ud}_h],[u_h, {\ud}_h] ) + b([u_h, {\ud}_h], \xi_h([u_h, {\ud}_h])) \geq \alpha_\xi \vertiii{[u_h, {\ud}_h]}_{X^0_{h,1}(\Omega)\times X^0_{h,1}(\Lambda) },
\end{equation}
\begin{equation}\label{stab_stability}
(s(\xi_{h}, \xi_{h}))^{\frac 12} \leq c_s \vertiii{[u_h, {\ud}_h]}_{X^0_{h,1}(\Omega)\times X^0_{h,1}(\Lambda) },
\end{equation}
being $\vertiii{[\cdot \, ,\, \cdot] }_{X^0_{h,1}(\Omega)\times X^0_{h,1}(\Lambda) }$ a suitable discrete norm. \\

We recall that in the case of Problem 2, 
\begin{equation*}
b([u_h, {\ud}_h], {\ld}_h) = \left(\avrc{T u_h} - {\ud}_h, {\ld}_h\right)_{\Lambda,|\DD|}.
\end{equation*}
The construction of the inf-sup stable space $L_h(\Lambda)$ is based on assembling the elements of the 3D mesh $\mathcal{T}_h^{\Omega}$ intersecting the 1D manifold $\Lambda$ into macro patches $\left\{ F_j \right\}_j$. These patches are such that and  $H\leq |F_j\cap \Lambda|\leq H+h$, where $H$ is sufficiently larger than $h$. Moreover,  we assume there exist constants $c_h$ and $c_H$ such that $c_h h\leq H \leq c_H^{-1}h$. We define the space $L_h(\Lambda)$ as the space of functions which are $P^0$ on each intersection $F_j\cap \Lambda$. Moreover, we associate to each patch $F_j$ a shape regular macro elements $\omega_j$, which is built adding to $F_j$ a sufficient number of elements of $\mathcal{T}_h^{\Omega}$. We make the following technical assumption: $\Gamma \subset \bigcup _{j} \omega_j$. Thanks to the shape regularity of these macro elements,  we have that the discrete trace and Poincarè inequalites hold. More precisely, for every function $v\in H^1(\omega_j)$,
\begin{equation}\label{discr_trace_ineq}
\|Tv\|_{\Gamma\cap \omega_j} \lesssim H^{-\frac 12} \|v\|_{L^2(\omega_j)}
\end{equation}
\begin{equation}\label{disc_poincare_ineq}
\|v- \pi_Lv\|_{L^2(\omega_j)} \leq c_P H \|\nabla v\|_{L^2(\omega_j)},
\end{equation}
where $\pi_L$ is defined as the projection onto piecewise constant functions on $F_j\cap \Lambda$.
 Moreover $\forall u_h \in X_h^\Omega$ we have the following average inequality 
\begin{equation*}
\sum _j \|\avrc{T u_h}\|^2_{L^2(F_j \cap \Lambda),|\DD| } \leq \sum _j \|T u_h\|^2_{L^2(\omega_j\cap \Gamma)}.
\end{equation*}
{\color{red} I think this inequality is valid but only globally. Indeed locally it is not guaranteed that the portion of $\Gamma$ corresponding to $F_j \cap \Lambda$ is contained in $\omega _j \cap \Gamma$}. \\
These choices lead to the following stabilization 
\begin{equation*}
s({\ld}_h, {\md}_h)= \sum _{K\in \mathcal{T}_{h'}^{\Lambda}} \int_{\partial K} h \llbracket {\ld}_h \rrbracket \llbracket {\md}_h \rrbracket,
\end{equation*}
being $\llbracket {\ld}_h \rrbracket$ the jump of ${\ld}_h$ across the internal faces of $\mathcal{T}_{h'}^{\Lambda}$.
\begin{lemma}
The space $L_h$ inf-sup stable, namely $\forall {\lld}_h \in L_h(\Lambda)$, $\exists \beta >0$ s.t.
\begin{equation*}
\sup_{\substack{v_h \in X_{h,1}^0(\Omega),\\ {\vd}_h \in X_{h',1}^0(\Lambda)}} \frac{\left(\avrc{T v_h} - {\vd}_h, {\lld} _h\right)_{\Lambda, |\DD|}}{\vertiii{[v_h, {\vd}_h]}} \geq \beta \|{\lld}_h\|_{H^{-\frac 12}(\Lambda)}.
\end{equation*} 
and the constant is independent of the cuts. 
\end{lemma}
\begin{proof}
% We have to prove that $\exists \beta >0$ such that $\forall l_h \in L_h$,
%\begin{equation*}
%\sup_{\substack{v_h \in X_{h,1}^0(\Omega),\\ {\vd}_h \in X_{h',1}^0(\Lambda)}} \frac{\left(\avrc{T v_h} - {\vd}_h, l _h\right)_{\Lambda, |\DD|}}{\vertiii{[v_h, {\vd}_h]}} \geq \beta \|l_h\|_{H^{-\frac 12}(\Lambda)}.
%\end{equation*}
As in the continuous case, we can choose ${\vd}_h=0$ and we prove that
\begin{equation*} 
\sup_{v_h \in X_{h,1}^0(\Omega)} \frac{\left(\avrc{T v_h} ,{\lld}_h\right)_{\Lambda, |\DD|}}{\|v_h\|_{H^1(\Omega)}} \geq \beta \|{\lld}_h\|_{H^{-\frac 12}(\Lambda)}.
\end{equation*} 
Proving the last inequality it is equivalent to find the Fortin operator $\pi_F: H^1_0(\Omega) \rightarrow X_{h,1}^0(\Omega)$, such that 
\begin{equation*}
\left(\avrc{T v} - \avrc{T \pi _F v}  , {\lld}_h\right)_{\Lambda, |\DD|}=0, \quad \forall v\in H^1_0(\Omega), \, {\lld}_h \in L_h(\Lambda)
\end{equation*} 
and
\begin{equation*}
\|\pi_F v\|_{H^1(\Omega)}\lesssim \|v\|_{H^1(\Omega)}.
\end{equation*} 

We define
\begin{equation*}
\pi_F v = I_h v + \sum _j \alpha _j \varphi _j \qquad \text{with }\alpha_j =\frac{\int_{F_j \cap \Lambda}|\DD| (\avrc{Tv}-\avrc{TI_hv})}{\int_{F_j \cap \Lambda}|\DD|\avrc{T\varphi _j}}
\end{equation*}
and $\varphi_j \in X_{h,1}^0(\Omega)$ s.t. supp$(\varphi_j)\subset \bar{\omega}_j$, $\varphi_j =0$ on $\partial \omega _j$ and 
\begin{equation*}
 \int_{F_j\cap \Lambda}|\DD|\avrc{T\varphi_j}=O(H) \text{ and } \|\nabla \varphi\|_{L^2(\omega _j)}=O(1). 
\end{equation*}
This construction is always possible provided $H$ is sufficiently larger that $h$.
Then we have
\begin{multline*}
\left(\avrc{T v} - \avrc{T \pi _F v}  , {\lld}_h\right)_{\Lambda, |\DD|} 
= \sum _j \int_{F_j\cap \Lambda} |\DD |\left[ \avrc{Tv}-\avrc{TI_hv}-\sum _i \alpha_i \avrc{T\varphi _i} \right]{\lld}_h \\
=(\text{supp} \varphi \subset \omega_j) \sum _j \int_{F_j\cap \Lambda}|\DD| \left[ \avrc{Tv}-\avrc{TI_hv}-\alpha_j \avrc{T\varphi _j} \right]{\lld}_h\\
=\sum _j \int_{F_j\cap \Lambda} |\DD| (\avrc{Tv}-\avrc{T I_h v}) {\lld}_h - \frac{\int_{F_j\cap \Lambda} |\DD| (\avrc{Tv}-\avrc{TI_hv})}{\int_{F_j\cap \Lambda}|\DD|\avrc{T\varphi_j}} \int_{F_j\cap \Lambda} |\DD|\avrc{T\varphi _j}{\lld}_h\\ 
=(\text{using $l_h$ constant on $F_j\cap \Lambda$})\,0.
\end{multline*}
Concerning the continuity of $\pi_F$, we have
\begin{multline*}
\|\nabla \pi_F v \|_{L^2(\Omega)} \leq \|\nabla I_h v\|_{L^2(\Omega)} + \left(\sum_j|\alpha_j|^2\|\nabla \varphi _j\|^2_{L^2(\bar{\omega}_j)}\right)^{\frac 12}\\
(\text{stability of }I_h)\lesssim   \|\nabla  v\|_{L^2(\Omega)} + \left(\sum_j|\alpha_j|^2\|\nabla \varphi _j\|^2_{L^2(\bar{\omega}_j)}\right)^{\frac 12}
\end{multline*}
and for the second term we have
\begin{multline*}
\sum_j|\alpha_j|^2\|\nabla \varphi _j\|^2_{L^2(\bar{\omega}_j)}\leq
\\
\left(\text{using }\|\nabla \varphi _j\|=O(1)\right) \lesssim  \sum_j \frac{\left(\left|\int_{F_j\cap \Lambda} |\DD| (\avrc{Tv}-\avrc{TI_hv})\right|\right)^2}{\left(\int_{F_j\cap \Lambda}|\DD|\avrc{T\varphi_j}\right)^2}
\\
\left(\text{since }\left|\int_{F_j\cap \Lambda}|\DD|\avrc{T\varphi_j}\right|=O(H)\right) \lesssim \frac {1}{H^2} \sum_j \left(\left|\int_{F_j\cap \Lambda} |\DD| (\avrc{Tv}-\avrc{TI_hv})\right| \right)^2
\\
(\text{Jensen}) \lesssim  \frac {1}{H^2} \sum_j |F_j\cap \Lambda| \int_{F_j\cap \Lambda} |\DD|^2(\avrc{Tv}-\avrc{TI_hv})^2
\\
(\text{being }|F_j\cap \Lambda| \leq H+h)\lesssim  \frac {1}{H} \sum_j \| \avrc{Tv}-\avrc{TI_hv}\|^2_{L^2(F_j\cap \Lambda), |\DD|}
\\
(\text{average inequality}) \lesssim  \frac {1}{H} \sum_j \| T(v-I_hv)\|^2_{L^2(\omega _j\cap \Gamma)}  
\\
\left(\text{trace inequality} \right)\lesssim  \frac {1}{H^2} \sum_j  \| v-I_h v\|^2_{L^2(\omega_j)} \lesssim  \frac {1}{H^2}  \| v-I_h v\|^2_{L^2(\Omega)} 
\\
(\text{approximation properties of }I_h)\lesssim \|\nabla  v\|^2_{L^2(\Omega)}
\end{multline*}
and the continuity of $\pi_F$ follows.
\end{proof}

%\subparagraph{Satisfaction of the assumptions of the 
%abstract analysis}
We choose the following discrete norm
\begin{equation*}
\vertiii{[u_h, {\ud}_h]}^2_{X_h(\Omega)\times X_{h'}(\Lambda) }
= \|u_h\|^2_{H^1(\Omega)}+\|{\ud}_h\|^2_{H^1(\Lambda),|\D|} + \|\avrc{Tu_h} - {\ud}_h\|^2_{-\frac 12, h, \Lambda, |\DD|},
\end{equation*}
where $\|\avrc{Tu_h} - {\ud}_h\|^2_{-\frac 12, h, \Lambda, |\DD|} = \|h^{\frac 12} (\avrc{Tu_h} - {\ud}_h)\|^2_{L^2(\Lambda), |\DD|} $. Then, we have the following lemma. 
\begin{lemma}
The inequalities \eqref{stab_coercivity} and \eqref{stab_stability} hold.
\end{lemma}
\begin{proof} 
Concerning the coercivity property \eqref{stab_coercivity}, we have to show that $\forall [u_h, {\ud}_h]$, there esists $\xi_h \in Q_h(\Lambda)$ s.t.
\begin{multline*}
(u_h,u_h)_{H^1(\Omega)}+ ({\ud}_h, {\ud}_h)_{H^1(\Lambda), |\D|} +   (\avrc{Tu_h} - {\ud}_h, \xi_h)_{\Lambda,|\DD|} \\
\geq \alpha_{\xi}(\|u_h\|^2_{H^1(\Omega)}+\|{\ud}_h\|^2_{H^1(\Lambda),|\D|}+ \|\avrc{Tu_h} - {\ud}_h\|^2_{-\frac 12, h, \Lambda, |\DD|}.
\end{multline*}
We choose 
\begin{equation*}
{\xi_h}_{|F_j\cap \Lambda}=\delta \frac 1H \pi_L(\avrc{Tu_h}-{\ud}_h) \qquad \text{with } \pi_L(\avrc{Tu_h}-{\ud}_h) =\frac{1}{|\Gamma_{F_j\cap \Lambda}|}\int_{F_j\cap \Lambda}|\DD| (\avrc{Tu_h}- {\ud}_h),
\end{equation*}
being $\Gamma_{F_j\cap \Lambda}$ the portion of $\Gamma$ with centerline $F_j\cap \Lambda$. 
Actually, $\xi_h\in L_h(\Lambda) \subset Q_h(\Lambda)$. Then,
\begin{multline*}
\left( \avrc{Tu_h} - {\ud}_h, \xi _h \right)_{\Lambda,|\DD|} 
= \sum_j \int_{F_j\cap \Lambda} |\DD|( \avrc{Tu_h} - {\ud}_h)\xi_h
\\
= \delta \frac{1}{H} \sum_j \int_{F_j\cap \Lambda}|\DD| (\pi_L( \avrc{Tu_h} - {\ud}_h) )^2
\\
(\text{orthogonality of $\pi_L$}) =  \delta \frac{1}{H} \|(\pi_L-\mathcal{I})(\avrc{Tu_h} - {\ud}_h)\|^2_{L^2(F_j\cap \Lambda),|\DD|} + \delta \frac{1}{H} \|\avrc{Tu_h} - {\ud}_h\|^2_{L^2(F_j\cap \Lambda),|\DD|}
\\ 
\geq -\delta \frac 1H \sum_j \|(\pi_L - \mathcal{I})\avrc{Tu_h}\|^2_{L^2(F_j\cap \Lambda), |\DD|}
- \delta \frac 1H \sum_j \|(\pi_L - \mathcal{I}){\ud}_h\|^2_{L^2(F_j\cap \Lambda),|\DD|}
\\ 
+ \delta \frac 1H \sum_j \|\avrc{Tu_h}-{\ud}_h\|^2_{L^2(F_j\cap \Lambda),|\DD|}. 
\end{multline*}
For the first term we have
\begin{multline*}
\sum _j  \|(\pi_L - \mathcal{I})\avrc{Tu_h}\|^2_{L^2(F_j\cap \Lambda),|\DD|} =\sum _j \int _{F_j \cap \Lambda} |\DD| (\pi_L \avrc{Tu_h}- \avrc{Tu_h})^2
\\
(\text{Average inequality)} \leq \sum _j  \int_{\omega _j \cap \Gamma} (\pi_L \avrc{Tu_h} -Tu_h)^2 
\\
(\text{trace inequality}) \leq \sum _j  \frac 1H \int_{\omega_j}(\pi_L \avrc{Tu_h} - u_h)^2 
\\ 
(\text{Poincare, see \cite[Corollary B.65]{MR2050138}})\leq \sum _j  H c_P ^2 \|\nabla u_h\|^2_{L^2(\omega _j)}.
\end{multline*}
For the second term we have
\begin{multline*}
\sum _j \|(\pi_L - \mathcal{I}){\ud}_h\|^2_{L^2(F_j\cap \Lambda),|\DD|} = \sum _j \int_{F_j\cap \Lambda} |\DD| (\pi_L {\ud}_h -{\ud}_h)^2
\\
(\text{Poincare, \cite[Corollary B.65]{MR2050138}})\lesssim \sum _j  H^2 c_P^2 \int_{F_j\cap \Lambda} |\DD|(\nabla {\ud}_h)^2
\\
(\text{since $H$ is fixed, we can find a constant s.t. } H|\DD| \lesssim |\D|) \lesssim \sum _j H c_P^2  \int_{F_j\cap \Lambda} |\D|(\nabla {\ud}_h)^2 
\\
\lesssim \sum _j H c_P^2  \|\nabla {\ud}_h\|^2_{L^2(F_j\cap \Lambda),|\D|}.
\end{multline*}
{\color{red} N.B. we are using a kind of weigthed Poincare inequality, check... I think it should work because I can do something like this
\begin{multline*}
\int_{F_j \cap \Lambda} |\DD| u^2 \leq 
max |\DD| \int_{F_j \cap \Lambda} u^2 \leq
max |\DD| \int_{F_j \cap \Lambda} (\nabla u)^2 =
\frac{max |\DD|}{min|\DD|} min|\DD| \int_{F_j \cap \Lambda} u^2 \leq\\
\frac{max |\DD|}{min|\DD|}  \int_{F_j \cap \Lambda} |\DD| u^2 
\end{multline*}
}\\
Therefore, we obtain
\begin{multline*}
a([u_h, {\ud}_h],[u_h, {\ud}_h] ) + b([u_h, {\ud}_h], \xi_h([u_h, {\ud}_h]))
\geq \\
(1-\delta c_P^2) \|\nabla u_h\|^2_{L^2(\Omega)} + (1- \delta c_P^2) \|\nabla {\ud}_h\|^2_{L^2(\Lambda), |\D|}
+\delta c_H  \|\avrc{Tu_h}-{\ud}_h\|^2_{-\frac 12,h,\Lambda, |\DD|}
\end{multline*}
and choosing $\delta=\frac{1}{2c_P^2}$ we obtain the coercivity inequality.\\
Concerning the stability inequality \eqref{stab_stability}, the proof is analogous to the one in \cite{burman2014}.
\end{proof}


%>>>>>>>>>>>>>>>>>>>>>>>>>>>>>>>>>>>>>>>>>>>>>>>>>>

\section{A benchmark problem with analytical solution}


Let $\Omega=[0,1]^3$, $\Lambda=\{x=\tfrac{1}{2}\}\times \{y=\tfrac{1}{2}\} \times [0,1] $
and $\Sigma=[\tfrac{1}{4}, \tfrac{3}{4}]\times [\tfrac{1}{4}, \tfrac{3}{4}]\times [0, 1]$.
Finally we let $\DD$ be the cross section of the virtual interface $\Gamma=\partial \Sigma$.
As a benchmark for the two formulations we consider the following coupled problems
%
\begin{subequations}\label{benchmark}
\begin{align}
\label{benchm_3d}
-\Delta u=f \quad &\text{in $\Omega$}\\
\label{benchm_1d}
-d_{zz}^2 \ud =g \quad &\text{on $\Lambda$}\\
u=u_b \quad &\text{on $\partial \Omega$},
\end{align}
\end{subequations}
where for formulation \eqref{eq:problem1} the mix-dimensional coupling constraint reads
\begin{equation}
  \label{eq:couple_1}
\trace{u} - \ext{\ud} = q_1\quad\text{ on }\Gamma,
\end{equation}
while for \eqref{eq:problem2} we set
\begin{equation}
    \label{eq:couple_2}
\avrc{u} - \ud = q_2\quad\text{ on }\Lambda.
\end{equation}
%
In \eqref{benchmark}-\eqref{eq:couple_2} the right-hand sides shall be defined as 
\begin{eqnarray*}
  &f=8\pi ^2 \sin (2\pi x) \sin (2\pi y),\quad &g={\pi ^2}\sin \left({\pi z}\right),\quad u_b=\sin (2\pi x) \sin (2\pi y),\\
  &q_1=\sin (2\pi x) \sin (2\pi y) - \sin \left({\pi z}\right),\quad &q_2=-\sin \left({\pi z}\right).
\end{eqnarray*}

The exact solution of \eqref{benchmark}, regardless of the coupling constraint,
is given by
%
\begin{eqnarray}
\label{benchm_sol3d}
u=\sin (2\pi x) \sin (2\pi y)\\
\label{benchm_sol1d}
%\ud=1+\exp(-z).
\ud=\sin \left({\pi z}\right).
\end{eqnarray}
%
Let us notice that $\ud$ satisfies homogeneous Dirichlet conditions at the boundary of $\Lambda$.
Moreover, the solution \eqref{benchm_sol3d}-\eqref{benchm_sol1d} satisfies on $\Gamma$ the relation
\begin{equation}\label{benchm_flux}
L=\nabla u \cdot \textbf{n}_{\oplus}=d_z \ud n_{\oplus,z}=0,
\end{equation}
with $n_{\oplus,z}$ the $z-$component of the normal unit vector to $\Gamma$.

We prove that \eqref{benchmark} is solution of \eqref{eq:problem2} in the
simplified case in which the starting 3D-3D problem is
\begin{subequations}\label{eq:dirneu_simple}
\begin{align}
- \Delta \up  &= f  && \text{ in } \Omega_{\oplus},\\
- \Delta \uf &= g  && \text{ in } \Sigma,\\
-\nabla \uf \cdot \nn_{\ominus} &= -\nabla \up \cdot \nn_{\ominus}  && \text{ on } \Gamma,\\
\uf - \up &= q_i  && \text{ on }  \Gamma,\\
\up &= h && \text{ on } \partial \Omega.
\end{align}
\end{subequations}
instead of \eqref{eq:dirneu}. Therefore the reduced problem in \eqref{eq:problem1} and
\eqref{eq:problem2} become respectively
%
\begin{subequations}\label{eq:problem1_simple}
\begin{align}
\label{eq:problem1_simple_eq1}
&(\nabla u,\nabla v)_{L^2(\Omega)} + |{\cal D}|(d_s \ud,d_s \vd)_{L^2(\Lambda)} 
+ \langle \trace{v}  - \ext{\vd}, \lambda \rangle_\Gamma
\\
\nonumber
&\qquad\qquad= (f,v)_{L^2(\Omega)} + |{\cal D}| (\avrd{g},\vd)_{L^2(\Lambda)}
\quad \forall v \in H^1_0(\Omega), \ \vd \in H^1_0(\Lambda)
\\
\label{eq:problem1_simple_eq2}
&   \langle \trace{u} - \ext{\ud} , \mu \rangle_\Gamma =  \langle q_1 , \mu \rangle_\Gamma
\quad \forall \mu \in H^{-\frac12}(\Gamma)\,.
\end{align}
\end{subequations}
and
\begin{subequations}\label{eq:problem2_simple}
  \begin{align}
    \label{eq:problem2_simple_eq1}
&(\nabla u,\nabla v)_{L^2(\Omega)} + |{\cal D}|(d_s \ud,d_s \vd)_{L^2(\Lambda)} 
+ |{\partial \cal D}| \langle \avrc{v} - \vd, \ld \rangle_{H^{-\frac12}(\Lambda)} 
\\
\nonumber
&\qquad\qquad= (f,v)_{L^2(\Omega)} + |{\cal D}| (\avrd{g}, \vd)_{L^2(\Lambda)}
\quad \forall v \in H^1_0(\Omega), \ \vd \in H^1_0(\Lambda)
\\
\label{eq:problem2_simple_eq2}
&  |\partial {\cal D}| \langle \avrc{u} -  \ud, \md \rangle_{H^{-\frac12}(\Lambda)} =
|\partial {\cal D}| \langle \avrc{q_2}, \md \rangle_{H^{-\frac12}(\Lambda)}
\quad \forall \md \in H^{-\frac12}(\Lambda)\,.
\end{align}
\end{subequations}

Let us prove that \eqref{benchm_sol3d}-\eqref{benchm_sol1d} is solution of
\eqref{eq:problem2_simple}. Using the integration by part formula and homogeneous
boundary conditions on $\Omega$ and $\Lambda$, from \eqref{eq:problem2_simple_eq1} we have
\begin{align*}
&-(\Delta u, v)_{L^2(\Omega)} - |{\cal D}|(d^2_{ss} \ud, \vd)_{L^2(\Lambda)} 
+ |{\cal D}|\langle \avrc{v}  - \vd, \ld \rangle_\Lambda
\\
\nonumber
&\qquad\qquad= (f,v)_{L^2(\Omega)} + |{\cal D}| (\avrd{g},\vd)_{L^2(\Lambda)}
\quad \forall v \in H^1_0(\Omega), \vd \in H^1(\Lambda).
\\
\end{align*}
Since $\ld=\avrc{L}=0$ and \eqref{benchm_sol3d} satisfies \eqref{benchm_3d} and \eqref{benchm_sol1d}
satisfies \eqref{benchm_1d}, we have that
\begin{align*}
-(\Delta u, v)_{L^2(\Omega)} =  (f,v)_{L^2(\Omega)} \\
-|{\partial \cal D}|(d^2_{ss} \ud, \vd)_{L^2(\Lambda)}  = |{\cal D}| (\avrd{g},\vd)_{L^2(\Lambda)},
\end{align*}
Thus \eqref{benchm_sol3d}-\eqref{benchm_sol1d} satisfy \eqref{eq:problem2_simple_eq1}.
The fact that the solution satisfy \eqref{eq:problem2_simple_eq2} follows from \eqref{eq:couple_2}.

We can prove in a similar way that \eqref{benchm_sol3d}-\eqref{benchm_sol1d}, with $\lambda=L=0$
satisfy \eqref{eq:problem1_simple}. Note in particular that $q_1$ is such that
$\trace{u} - \ext{\ud} = q_1$ on $\Gamma$.

\subsection{Numerical experiments. $\mathcal{T}^{\Omega}_h$ conforming to $\Gamma$}\label{sec:experiment_conform}
Using the benchmark problem \eqref{benchmark} we now investigate convergence
properties of the two formulations. To this end we consider a \emph{uniform} tessilation
of $\mathcal{T}^{\Omega}_h$ of $\Omega$ consisting of tetrahedra with diameter $h$.
Further, the discretization shall be geometrically \emph{conforming} to both $\Lambda$
and $\Gamma$ such that the tessilations $\mathcal{T}^{\Gamma}_h$, $\mathcal{T}^{\Lambda}_h$
are made up of facets and edges of $\mathcal{T}^{\Omega}_h$ respectively, cf. Figure \ref{fig:mesh}
for illustration.


\begin{table}
%%   %%%
\begin{minipage}[b]{0.35\linewidth}
 \centering
 \includegraphics[width=\textwidth]{graphics/conform_mesh.pdf}
 \vspace{-20pt}
 \captionof{figure}{
$\Lambda$ and $\Gamma$ conforming discretization of $\Omega$
   used for \eqref{eq:problem1_simple} and \eqref{eq:problem2_simple}.   
 }
\label{fig:mesh}
\end{minipage}
\hspace{2pt}
\begin{minipage}[b]{0.63\textwidth}
  \scriptsize{
  \begin{center}
    \begin{tabular}{l|llll}
      \toprule
    $h^{-1}$ & $\norm{u-u_h}_{1, \Omega}$ & $\norm{\ud-\udh}_{1, \Lambda}$ & $\norm{\lambda-\lambda_h}_{-\tfrac{1}{2},\Gamma}$ & $\norm{\lambda-\lambda_h}_{0,\Gamma}$\\
      \hline
4  & 3.4E0(--)    & 5.3E-1(--)   & 2.9E0(--)    &8.7E0(--)    \\
8  & 1.7E0(0.99)  & 2.6E-1(1.06) & 6.1E-1(2.25) &1.9E0(2.21)  \\
16 & 8.7E-1(0.99) & 1.3E-1(1.02) & 1.4E-1(2.13) &4.7E-1(1.99) \\
32 & 4.4E-1(1.00) & 6.3E-2(1.00) & 3.4E-2(2.03) &1.3E-1(1.80) \\
64 & 2.2E-1(1.00) & 3.1E-2(1.00) & 8.6E-3(2.00) &4.2E-2(1.68) \\
\midrule
$h^{-1}$ & $\norm{u-u_h}_{1, \Omega}$ & $\norm{\ud-u_{\odot, h^{\prime}}}_{1, \Lambda}$ & $\norm{\ld-\ldh}_{-\tfrac{1}{2}, \Lambda}$ & $\norm{\ld-\ldh}_{0, \Lambda}$\\
\hline
4   & 3.1E0(--)    & 5.4E-1(--)   & 4.4E-2(--)   & 7.8E-2(--)  \\
8   & 1.7E0(0.87)  & 2.6E-1(1.06) & 1.1E-2(2.01) & 1.9E-2(2.01)\\
16  & 8.6E-1(0.96) & 1.3E-1(1.02) & 2.7E-3(2.01) & 4.8E-3(2.02)\\
32  & 4.4E-1(0.99) & 6.3E-2(1.00) & 6.7E-4(2.01) & 1.2E-3(2.01)\\
64  & 2.2E-1(1.00) & 3.1E-2(1.00) & 1.7E-4(2.01) & 3.0E-4(2.01)\\
128 & 1.1E-1(1.00) & 1.6E-2(1.00) & 4.1E-5(2.01) & 7.4E-5(2.00)\\
\bottomrule
  \end{tabular}
  \end{center}
  }
  \captionof{table}{Error convergence of \eqref{eq:problem1_simple} and \eqref{eq:problem2_simple}
    on a benchmark problem \eqref{benchmark}. Continuous linear Lagrange
    elements are used.
  }
  \label{tab:error_conform}
  \end{minipage}
\end{table}

Considering inf-sup stable discretization in terms of continuous linear Lagrange
($P_1$) elements (for all the spaces), Table \ref{tab:error_conform}
lists the errors of formulations \eqref{eq:problem1_simple} and \eqref{eq:problem2_simple}
on the benchmark problem. It can be seen the error in $u$ and $\ud$ in $H^1$ norm
converges linearly (as can be expected due to $P_1$ element discretization).
Moreover, the error of the Lagrange multiplier approximation in $H^{-1/2}$ norm
decreases quadratically. In the light of $P_1$ discretization this rate appears
superconvergent. We speculate that the result is due to the fact that the
exact solution is particularly simple, $\lambda=\ld=0$.
%In case of the results for
%\eqref{eq:problem1_simple} the rate can also be due to the fact that the
%error is interpolated into the same finite element space as the approximation $Q_h$.
We remark that for $u$ and $\ud$ the error is interpolated into the finite element space of
piecewise quadratic \emph{discontinous} functions. For \eqref{eq:problem2_simple} we
evaluate the fractional norm and interpolate the error using piecewise continuous
cubic functions. This is due to the fact that evaluating the fractional norm in higher order spaces
for on $\Gamma$ is prohibitively costly. For the sake of comparison with non-conforming formulation of \eqref{eq:problem2} from
\S\ref{sec:unfit2} Table \ref{tab:error_conform} also
lists the error of the Lagrange multiplier in the $L^2$ norm. Here, quadratic convergence is observed
for \eqref{eq:problem2_simple}. For \eqref{eq:problem1_simple} the rate between 1.5 and 2.

We plot the numerical solution of problem \eqref{eq:problem1_simple} and \eqref{eq:problem2_simple} in
Figure \ref{fig:sol_benchm1} and \ref{fig:sol_benchm2}, respectively.

\begin{figure}
\centering
\includegraphics[width = 0.9\textwidth]{./graphics/mfs_LM2d}
\caption{Numerical solution of problem \eqref{eq:problem1_simple}: functions $u_h$ and $\udh$ on the left and the Lagrance multiplier $\lambda_h$ on the right.}\label{fig:sol_benchm1}
\end{figure}


\begin{figure}
\centering
\includegraphics[width = 0.9\textwidth]{./graphics/mfs_LM1d}
\caption{Numerical solution of problem \eqref{eq:problem2_simple}: functions $u_h$ and $\udh$ on the left and the Lagrance multiplier ${\ld}_h$ on the right.}\label{fig:sol_benchm2}
\end{figure}


\subsection{Numerical experiments. $\mathcal{T}^{\Omega}_h$ non-conforming to $\Gamma$}\label{sec:experiment_nonconform}
Using benchmark problem \eqref{benchmark} we consider \eqref{eq:problem2} in the
setting of \S \ref{sec:unfit2}. To this end we let $\mathcal{T}^{\Omega}_h$ be a uniform
tessilation of $\Omega$ such that no cell $\mathcal{T}^{\Omega}_h$ has any edge
lying on $\Lambda$. Further we let $h^{\prime}=h/3$ in $\mathcal{T}^{\Lambda}_{h^{\prime}}$,
cf. Figure \ref{fig:unfit}.

Using discretization in terms of $P_1$-$P_1$-$P_0$ element Table \ref{tab:error_unfit}
lists the error of the stabilized formulation of \eqref{eq:problem2}. A linear
convergence in the $H^1$ norm can be observed in the error of $u$ and $\ud$. We
remark that the norms were computed as in \S\ref{sec:experiment_conform}. For simplicity
the convergence of the multiplier is measured in the $L^2$ norm rather then the $H^{-1/2}(\Gamma)$
norm used in the analysis. Then, convergence exceeding order 1.5 can be observed, however,
the rates are rather unstable.

\miro{The solution is plotted in ..... I guess there should be some zoom in on the
cut cells.}

\begin{table}
%%   %%%
\begin{minipage}[b]{0.35\linewidth}
 \centering
 \includegraphics[width=\textwidth]{graphics/nonconform_mesh.pdf}
 \vspace{-20pt}
 \captionof{figure}{
   Sample discretization of the benchmark geometry in the non-conforming case. 
 }
\label{fig:unfit}
\end{minipage}
\hspace{2pt}
%%%
    \begin{minipage}[b]{0.62\linewidth}
      %%%
  \scriptsize{
  \begin{center}
    \begin{tabular}{l|lll}
      \toprule
    $h^{-1}$ & $\norm{u-u_h}_{1, \Omega}$ & $\norm{\ud-\udh}_{1, \Lambda}$ & $\norm{\ld-\ldh}_{0,\Lambda}$\\
      \hline
5   & 2.6E0(--)    & 2.3E-1(--)   & 1.7E-1(--) \\ 
9   & 1.5E0(0.84)  & 9.4E-2(1.42) & 7.1E-2(1.36)\\
17  & 8.1E-1(0.94) & 4.3E-2(1.18) & 2.9E-2(1.37)\\
33  & 4.2E-1(0.98) & 2.1E-2(1.06) & 7.9E-3(1.91)\\
65  & 2.1E-1(0.99) & 1.1E-2(1.02) & 2.6E-3(1.64)\\
129 & 1.1E-1(1.00) & 5.2E-3(1.01) & 8.5E-4(1.61)\\
\bottomrule
    \end{tabular}
  \end{center}    
}    
  \captionof{table}{
    Error convergence of \eqref{eq:problem2_simple} on a benchmark problem \eqref{benchmark} in case
    $\mathcal{T}^{\Omega}_h$ does not conform to $\Lambda$.}
  \label{tab:error_unfit}    
  \end{minipage}
\end{table}

\subsection{Comparison}
In Tables \ref{tab:error_conform}, \ref{tab:error_unfit} one can observe that 
all the formulations yield practically identically accurate approximations of $u$.
Futher, compared to the conforming case, the stabilized formulation \eqref{eq:problem2}
results in a greater accuracy of $u_{\cdot, h}$ as the underlying mesh $\mathcal{T}^{\Lambda}_{h}$ is
here finer. Due to the different definitions in the three formulations, comparision of the Lagrange
multiplier convergence is not straightforward. We therefore limit ourselves to a
comment that in the $L^2$ norm all the formulations yield faster than linear convergence.

In order to discuss solution cost of the formulations we consider 
the resulting preconditioned linear systems. In particular, we shall compare
spectral condition numbers and the time to convergence of the preconditioned
minimal residual (MinRes) solver with the with stopping criterion requiring
the relative preconditioned residual norm to be less than $10^{-8}$. We remark
that we shall ignore the setup cost of the preconditioner.

Following operator preconditioning technique \cite{mardal2011preconditioning} we
propose as preconditioners for \eqref{eq:problem1} and \eqref{eq:problem2} in the
conforming case the (approximate) Riesz mapping with respect to the inner products of
the spaces in which the two formulations were proved to be well posed.
In particular, the preconditioner for the Lagrange multiplier relies on
(the inverse of) the fractional Laplacian $-\Delta^{-1/2}$ on $\Gamma$ for
\eqref{eq:problem1_simple} and $\Lambda$ for \eqref{eq:problem2_simple}.
A detailed analysis of the preconditioners will be presented in a separate
work. We remark that in both cases the fractional Laplacian was here realized
by spectral decomposition \cite{kuchta2016preconditioners}.

For the unfitted stabilized \eqref{eq:problem2} the Lagrange multiplier preconditioner
uses a Riesz map with respect to the inner product due to $L^2(\mathcal{G}_h)$ and
the stabilization \eqref{eq:stab}, i.e.
\[
({\ld}_h, {\md}_h) \mapsto \sum _{K\in \mathcal{G}_h}\int_{K}{\ld}_h {\md}_h + \sum _{K\in \mathcal{G}_h} \int_{\partial K \setminus \partial \mathcal{G}_h} h \llbracket {\ld}_h \rrbracket \llbracket {\md}_h \rrbracket.
\]
This simple choice does not yield bounded
iterations. However, establishing a robust precondtioner in this case 
is beyond the scope of the paper and shall be pursued in the future works.


In Table \ref{tab:cost} we compare solution time, number of iterations and
condition numbers of the (linear systems due to the) three formulations.
Let us first note that the proposed preconditioners for \eqref{eq:problem1} and
\eqref{eq:problem2} in the conforming case seem robust with respect to discretization
parameter as the iteration counts and condition numbers are bounded in $h$.
We then see that the solution time for \eqref{eq:problem1_simple} is about 2 times
longer compared to \eqref{eq:problem2_simple} which is about 4 times more expensive
than the solution of the Poisson problem \eqref{benchm_3d}. This is in addition to the
higher setup costs of the preconditioner which in our implementation involve solving
an eigenvalue problem for the fractional Laplacian. Therefore it is advantageous to keep
the multiplier space as small as possible. We remark that the missing
results for \eqref{eq:problem1_simple} in Table \ref{tab:cost} and \ref{tab:error_conform}
are due to the memory limitations which we encounter when solving the eigenvalue problem
for the Laplacian, which for finest mesh involves cca 32 thousand eigenvalues, cf. Appendix \ref{sec:appendix}.
%

Due to the missing proper preconditioner for the Lagrange multiplier block the
number of iterations in the third, unfitted formulation can be seen to approximatly
double on refinement. 

\begin{table}
  \scriptsize{
  \begin{center}
    \begin{tabular}{l|lll|lll|lll|ll}
      \toprule
      \multirow{2}{*}{$l$} & \multicolumn{3}{c|}{\eqref{eq:problem1}} & \multicolumn{3}{c|}{\eqref{eq:problem2}} & \multicolumn{3}{c|}{ Stabilized \eqref{eq:problem2}} & \multicolumn{2}{c}{\eqref{benchm_3d}}\\
      \cline{2-12}
      & \# & $T\left[s\right]$ & $\kappa$ & \# & $T\left[s\right]$ & $\kappa$ & \# & $T\left[s\right]$ & $\kappa$ & \# & $T\left[s\right]$\\
      \hline
      1 & 20 & 0.03  & 15.56 & 9  & 0.02  & 3.04 & 21  & 0.01   & 9.70  &3 & $<0.01$\\ 
      2 & 35 & 0.06  & 16.28 & 17 & 0.03  & 4.67 & 31  & 0.03   & 15.87 &4 & $<0.01$\\ 
      3 & 38 & 0.14  & 16.64 & 22 & 0.06  & 6.25 & 53  & 0.15   & 32.93 &5 & 0.01   \\ 
      4 & 39 & 1.70  & 16.75 & 24 & 0.89  & 7.03 & 110 & 4.54   & 61.48 &5 & 0.12   \\ 
      5 & 38 & 12.04 & 16.78 & 20 & 5.21  & 5.02 & 232 & 59.43  & 94.25 &5 & 0.90  \\ 
      6 & -- & --    & --    & 17 & 28.77 & --   & 507 & 832.90 & --    &6 & 7.75  \\ 
      \bottomrule
    \end{tabular}
    \end{center}
    }
  \caption{Cost comparison of the formulations across refinement levels $l$.
    Number of Krylov (preconditioned conjugate gradient for \eqref{benchm_3d},
    MinRes otherwise) iterations and the conditioned number of the preconditioned
    problem is denoted by $\#$ and $\kappa$ respectively. Time till convergence
    of the iterative solver (excluding the setup) is shown as $T$. 
  }
\label{tab:cost}
\end{table}



%{\color{red}
%\begin{remark} In the following some observations:\\
%
%\begin{itemize}
%\item \textbf{How do we compute the norm $\|\cdot \|_{H^{-\frac 12}}$ in Table \ref{tab:error_conform}?} The norm $\|\cdot \|_{H^{-\frac 12}}$ has been computed using a decomposition based on eigenvalues and eigenfuctions of $\Delta ^{-\frac 12}$.\\
%
%\item \textbf{Why $\lambda _h \neq \lambda$?} Since $\lambda = 0 \in Q_h$, we would expect $\lambda _h = \lambda$. However, first of all we suspect that for the approximation error an estimate similar to \cite[(11.26)]{steinbach2007numerical} holds, reported in the following
%\begin{equation*}
%\|u-u_h\|^2_{H^1} + \|\lambda - \lambda_h\|^2_{H^{-\frac 12}} \leq c_1 h^2|u|^2_{H^2} + c_2 h^3\|\lambda_h\|^2_{H^1_{pw}},
%\end{equation*}
%where $\|\cdot \|_{H^1_{pw}}$ is the broken $H^1$ norm. 
%So the error on the LM depends on the error on the solution. We suspect that if for example $u$ and $\ud$ are $P_2$ functions and we use a $P_2$ approximation (therefore $u = u_h$ and $\ud = \udh$), also $\lambda = \lambda_h$.\\
%
%\item \textbf{Why the error is larger close to edges, especially in Figure \ref{fig:sol_benchm1}?} 
%We suspect that the reasoning for why the error is larger close to edges and corner vertices is that the multiplier is
%related to $\nabla u \cdot \boldsymbol{n}$ and the normal in the corner or on the z-aligned edges in the figure is not well-defined. Btw, the oscillations are smaller if we impose Neumann
%bcs on z min and max surfaces of $\Omega$ (the rest has Dirichlet).
%\end{itemize}
%\end{remark}
%}

%-----------------
\bibliographystyle{siamplain}
\bibliography{3d1d_coupled}
\end{document}
