\section{Unfitted P1-P1-P0 case}
\subsection{Problem 1} All the observations and the proofs are an application of \cite{burman2014} to our case. Let $\mathcal{T}^{\Omega}_h$ and  $\mathcal{T}^{\Lambda}_{h'}$ denote a shape-regular triangulation  of $\Omega$ and an admissible partition  of $\Lambda$ respectively. Let us also introduce the set $\mathcal{G}_h$ of elements of $\mathcal{T}^{\Omega}_h$ intersecting the interface $\Gamma$, namely $\mathcal{G}_h=\left\{ K\in \mathcal{T}^{\Omega}_h : \, K\cap \Gamma \neq \emptyset \right\}$. For the discrete solutions $u_h$ and ${\ud}_h$ of the problem in $\Omega$ and on $\Lambda$ we choose the conforming spaces $X_{h,1}^0(\Omega)\subset H^1_0(\Omega)$ and  $X_{h,1}^0(\Lambda)\subset H^1_0(\Lambda)$ respectively. We extend the Lagrange multiplier $\lambda_h$ to the volume elements of $\mathcal{G}_h$ and we define $Q_h=\left\{\lambda_h: \, {\lambda_h}_{|K}\in P^0(K)\,  \forall K\in \mathcal{G}_h\right\}$. With this choice of the LM space, the problem is not inf-sup stable. We add a stabilization term $-s(\lambda_h, \mu_h)$ and we have
\begin{multline*}
a([u_h, {\ud}_h], [v_h, {\vd}_h]) + b([v_h, {\vd}_h], \lambda_h) + b([u_h, {\ud}_h], \mu_h) - s(\lambda_h, \mu_h)= (f, [v_h, {\vd}_h])\\
 \forall [v_h, {\vd}_h]\in X_h(\Omega)\times X_h(\Lambda) ,\, \forall \mu_h \in Q_h.
\end{multline*}
The stabilization $s(\lambda_h, \mu_h)$ is connected to a projector $\pi_L$ between $Q_h$ and a new space $L_h$ for the LM for which we can prove the inf-sup stability.\\
Therefore we have to build this new space $L_h$, prove that the inf-sup condition is fulfilled, build a projection operator $\pi_L: Q_h \rightarrow L_h$, build $s(\lambda_h, \mu_h)$ and prove that $\forall [u_h, {\ud}_h]$, there exists $\xi_h([u_h, {\ud}_h]) \in Q_h$ s.t.  
\begin{equation}\label{stab_coercivity}
a([u_h, {\ud}_h],[u_h, {\ud}_h] ) + b([u_h, {\ud}_h], \xi_h([u_h, {\ud}_h])) \geq \alpha_\xi \vertiii{[u_h, {\ud}_h]}_{X_h(\Omega)\times X_h(\Lambda) },
\end{equation}
\begin{equation}\label{stab_stability}
(s(\xi_{h}, \xi_{h}))^{\frac 12} \leq c_s \vertiii{[u_h, {\ud}_h]}_{X_h(\Omega)\times X_h(\Lambda) },
\end{equation}
being $\vertiii{\cdot }_{X_h(\Omega)\times X_h(\Lambda) }$ a suitable discrete norm. \\
The space $L_h$ is built as the space of $P^0$ functions defined on macro patches $\left\{ F_j \right\}_j$ of elements of $\mathcal{G}_h$. These patches are such that $diam(F_j)\leq H$ and  $H\leq |F_j\cap \Gamma|\leq H+h$. Moreover, there exist constants $c_h$ and $c_H$ such that $c_hh\leq H \leq c_H^{-1}h$. To each patch $F_j$ we associate two shape regular macro elements $\omega_j$ and $\tilde{\omega}_j$: $\omega_j$ is built adding to $F_j \cap \Omega_{\oplus}$ a sufficient number of elements of $\mathcal{T}_h^{\Omega}$ contained in $\Omega_{\oplus}$, whereas $\tilde{\omega}_j$ is obtained adding to $F_j \cap\Sigma$ a sufficient number of elements of $\mathcal{T}_h^{\Omega}$ contained in $\Sigma$. Thanks to the shape regularity of these macro elements, we have that the following trace and Poincarè inequalites hold. For every function $v\in H^1(\omega_j)$,
\begin{equation}\label{discr_trace_ineq}
\|Tv\|_{\Gamma\cap \omega_j} \lesssim H^{-\frac 12} \|v\|_{L^2(\omega_j)}
\end{equation}
\begin{equation}\label{disc_poincare_ineq}
\|v- \pi_Lv\|_{L^2(\omega_j)} \leq c_P H \|\nabla v\|_{L^2(\omega_j)},
\end{equation}
being $\pi_L$ the projection onto piecewise constant functions on $F_j$.
The same also in $\tilde{\omega _j}$ for $v\in H^1(\tilde{\omega _j})$. This choice leads to the following stabilization 
\begin{equation*}
s(\lambda_h, \mu_h)= \sum _{K\in \mathcal{G}_h} \int_{\partial K \setminus \partial \mathcal{G}_h} h \llbracket \lambda_h \rrbracket \llbracket \mu_h \rrbracket,
\end{equation*}
being $\llbracket \lambda_h \rrbracket$ the jump of $\lambda_h$ across the internal faces of $\mathcal{G}_h$.
\subparagraph{Is $L_h$ inf-sup stable with constants independent of the cuts?} We have to prove that $\forall l_h \in L_h$, $\exists \beta >0$ s.t.
\begin{equation*}
\sup_{\substack{v_h \in X_{h,1}^0(\Omega),\\ {\vd}_h \in X_{h',1}^0(\Lambda)}} \frac{b([v_h, {\vd}_h], l_h)}{\vertiii{[v_h, {\vd}_h]}} \geq \beta \|l_h\|_{H^{-\frac 12}(\Gamma)}.
\end{equation*}
As in the continuous case, we can choose ${\vd}_h=0$ and we prove that
\begin{equation*} 
\sup_{v_h \in X_{h,1}^0(\Omega),\\ {\vd}_h \in X_{h,1}^0(\Lambda)} \frac{b([v_h, {\vd}_h], l_h)}{\vertiii{[v_h, {\vd}_h]}} \geq  \sup_{v_h \in X_{h,1}^0(\Omega)} \frac{b(v_h, l_h)}{\|v_h\|_{H^1(\Omega)}} \geq \beta \|l_h\|_{H^{-\frac 12}(\Gamma)}, 
\end{equation*}
where $b(v_h, l_h)=(Tv_h,l_h)_{\Gamma} $ with a little abuse of notation. 
Proving the last inequality it is equivalent to find the Fortin operator $\pi_F: H^1_0(\Omega) \rightarrow X_{h,1}^0(\Omega)$, such that 
\begin{equation*}
b(v-\pi_F v, l_h)=0, \quad \forall v\in H^1_0(\Omega), \, l_h \in L_h
\end{equation*} 
and
\begin{equation*}
\|\pi_F v\|_{H^1(\Omega)}\lesssim \|v\|_{H^1(\Omega)}.
\end{equation*} 
We define
\begin{equation*}
\pi_F v = I_h v + \sum _j \alpha _j \varphi _j \qquad \text{with }\alpha_j =\frac{\int_{F_j \cap \Gamma}T(v-I_hv)}{\int_{F_j \cap \Gamma}T\varphi _j}
\end{equation*}
and $\varphi_j \in X_{h,1}^0(\Omega)$ s.t. supp$(\varphi_j)\subset \bar{\omega}_j$, $\varphi_j =0$ on $\partial \omega _j$ and 
\begin{equation*}
 \int_{F_j\cap \Gamma}T\varphi_j=O(H) \text{ and } \|\nabla \varphi\|_{L^2(\omega _j)}=O(1). 
\end{equation*}
This construction is always possible provided $H$ is sufficiently larger that $h$ (usually $H=3h$).
Then $b(v-\pi_F v, l_h)=0 \, \forall l_h$ follows by construction. Indeed:
\begin{multline*}
b(v-\pi_F v, l_h)= \int_{\Gamma} T(v-\pi_Fv)l_h = \sum _j \int_{F_j\cap \Gamma} \left[ T(v-I_hv)-\sum _i \alpha_i T\varphi _i \right]l_h \\
=(\text{supp} \varphi \subset \omega_j) \sum _j \int{F_j\cap \Gamma} \left[ T(v-I_hv)-\alpha_j T\varphi _j \right]l_h\\
=\sum _j \int_{F_j\cap \Gamma} T(v-I_hv) l_h - \frac{\int_{F_j\cap \Gamma} T(v-I_hv)}{\int_{F_j\cap \Gamma}T\varphi_j} \int_{F_j\cap \Gamma}T\varphi _jl_h\\ 
=(\text{using $l_h$ constant on $F_j\cap \Gamma$})\,0.
\end{multline*}
Concerning the continuity of $\pi_F$, we have
\begin{multline*}
\|\nabla \pi_F v \|_{L^2(\Omega)} \leq \|\nabla I_h v\|_{L^2(\Omega)} + \sum_j|\alpha_j|\|\nabla \varphi _j\|_{L^2(\bar{\omega}_j)}
\\
(\text{stability of }I_h)\lesssim   \|\nabla  v\|_{L^2(\Omega)} + \sum_j|\alpha_j|\|\nabla \varphi _j\|_{L^2(\bar{\omega}_j)}
\\
\left(\text{using }\|\nabla \varphi\|=O(1)\right) \lesssim \|\nabla  v\|_{L^2(\Omega)} + \sum_j \frac{\left|\int_{F_j\cap \Gamma} T(v-I_h v)\right|}{\left|\int_{F_j\cap \Gamma}T\varphi _j\right|}
\\
\left(\text{since }\left|\int_{F_j\cap \Gamma}T\varphi _j\right|=O(H)\right) \lesssim  \|\nabla  v\|_{L^2(\Omega)} + \frac 1H \sum_j \left|\int_{F_j\cap \Gamma} T(v-I_h v)\right| 
\\
(\text{H\"older}) \lesssim  \|\nabla  v\|_{L^2(\Omega)} + \frac 1H \sum_j |F_j\cap \Gamma|^{\frac 12} \left(\int_{F_j\cap \Gamma} (T(v-I_h v))^2\right)^{\frac 12}
\\
(\text{being }|F_j\cap \Gamma| \leq H+h)\lesssim   \|\nabla  v\|_{L^2(\Omega)} + \frac {1}{H^{\frac 12}} \sum_j  \| v-I_h v\|_{F_j\cap \Gamma} 
\\
\left(\text{trace inequality} \right)\lesssim   \|\nabla  v\|_{L^2(\Omega)} + \frac {1}{H} \sum_j  \| v-I_h v\|_{L^2(\omega_j)} \lesssim \|\nabla  v\|_{L^2(\Omega)} + \frac {1}{H}  \| v-I_h v\|_{L^2(\Omega)} 
\\
(\text{approximation properties of }I_h)\lesssim \|\nabla  v\|_{L^2(\Omega)}.
\end{multline*}

\subparagraph{Satisfaction of the assumptions of the 
abstract analysis} 
We have to prove \eqref{stab_coercivity} and \eqref{stab_stability}.
We choose the following discrete norm
\begin{equation*}
\vertiii{[u_h, {\ud}_h]}^2_{X_h(\Omega)\times X_h(\Lambda) }
= \|u_h\|^2_{H^1(\Omega)}+|D|\|{\ud}_h\|^2_{H^1(\Lambda)} + \|u_h - {\ud}_h\|^2_{-\frac 12, h, \Gamma},
\end{equation*}
where $\|u_h - {\ud}_h\|^2_{-\frac 12, h, \Gamma} = \|h^{\frac 12} (u_h - {\ud}_h)\|^2_{L^2(\Gamma)} $.
Concerning the coercivity property \eqref{stab_coercivity}, we have to show that $\forall [u_h, {\ud}_h]$, there esists $\xi_h$ s.t.
\begin{equation*}
(u_h,u_h)_{H^1(\Omega)}+ |D|({\ud}_h, {\ud}_h)_{H^1(\Lambda)} +  (Tu_h -  \mathcal{U}_E {\ud}_h, \xi_h)_{\Gamma} \geq \alpha_{\xi}(\|u_h\|^2_{H^1(\Omega)}+|D|\|{\ud}_h\|^2_{H^1(\Lambda)}+ \|u_h - {\ud}_h\|^2_{-\frac 12, h, \Gamma}).
\end{equation*}
We choose 
\begin{equation*}
{\xi_h}_{|F_j\cap \Gamma}=\delta \frac 1H \pi_L(u_h-{\ud}_h) \qquad \text{with } \pi_L(u_h-{\ud}_h) =\frac{1}{|F_j\cap \Gamma|}\int_{F_j\cap \Gamma} Tu_h-\mathcal{U}_E {\ud}_h.
\end{equation*}
Actually, $\xi_h\in L_h \subset Q_h$. Then,
\begin{multline*}
\left( Tu_h - \mathcal{U}_E{\ud}_h, \xi _h \right)_{\Gamma} 
= \sum_j \int_{F_j\cap \Gamma} ( Tu_h - \mathcal{U}_E{\ud}_h)\xi_h
\\
= \delta \frac{1}{H} \sum_j \int_{F_j\cap \Gamma} ( Tu_h - \mathcal{U}_E{\ud}_h) \pi_L(u_h - {\ud}_h)
\\
=\delta \frac{1}{H} \sum_j \int_{F_j\cap \Gamma}  (\pi_L(u_h - {\ud}_h))^2
\\
=\delta \frac{1}{H} \sum_j \left(\|(\pi_L - \mathcal{I})(u_h-{\ud}_h)\|^2_{L^2(F_j\cap \Gamma)}  + \|u_h-{\ud}_h\|^2_{L^2(F_j\cap \Gamma)} \right)
\\
\geq -\delta \frac 1H \sum_j \|(\pi_L - \mathcal{I})u_h\|^2_{L^2(F_j\cap \Gamma)}
- \delta \frac 1H \sum_j \|(\pi_L - \mathcal{I}){\ud}_h\|^2_{L^2(F_j\cap \Gamma)} 
+ \delta \frac 1H \sum_j \|u_h-{\ud}_h\|^2_{L^2(F_j\cap \Gamma)}
\\
\geq -\delta \sum_j c_P^2 \|\nabla u_h\|^2_{L^2(\omega_j)}
- \delta \sum_j c_P^2 \|\nabla  \mathcal{U}_E u_h\|^2_{L^2(\tilde{\omega}_j)}
+ \delta \frac 1H \sum_j \|u_h-{\ud}_h\|^2_{L^2(F_j\cap \Gamma)}
\\
(\text{since }c_H H \leq h ) \leq -\delta c_P^2 \|\nabla u_h\|^2_{L^2(\Omega)} - \delta c_P^2 |\D|\|\nabla {\ud}_h\|^2_{L^2(\Lambda)} + \delta c_H \|u_h-{\ud}_h\|^2_{--\frac 12,h,\Gamma}. 
\end{multline*}
With a little abuse of notation we are denoting with $\mathcal{U}_E$ both the uniform extension to $\Gamma$ and $\tilde{\omega}_j$. Therefore, we obtain
\begin{multline*}
a([u_h, {\ud}_h],[u_h, {\ud}_h] ) + b([u_h, {\ud}_h], \xi_h([u_h, {\ud}_h]))
\geq \\
(1-\delta c_P^2) \|\nabla u_h\|^2_{L^2(\Omega)} + (1- \delta c_P^2) |\D|\|\nabla {\ud}_h\|^2_{L^2(\Lambda)}
+\delta c_H \|u_h-{\ud}_h\|^2_{-\frac 12,h,\Gamma}
\end{multline*}
and choosing $\delta=\frac{1}{2c_P^2}$ we obtain the coercivity inequality.\\
Concerning the stability inequality, the proof is analogous to the one in \cite{burman2014}.

\subsection{Problem 2} In the case of Problem 2 we consider $Q_h=\{\lambda_{h} : \lambda_h \in P^0(K) \forall K \in \mathcal{T}^{\Lambda}_{h'}\}$, namely the multiplier lives on the same mesh used for the 1D solution ${\ud}_h$. Notice that in this case we suppose that the mesh sizes of the 3D mesh $	\mathcal{T}^{\Omega}_h$ and the 1D mesh $\mathcal{T}^{\Lambda}_{h'}$ are different, in particular we suppose the 1D mesh is finer. With this choice the problem is not inf-sup stable, therefore we add a stabilization term as in the case of Problem 1. Again, we have to build a new space $L_h$, prove that the inf-sup condition is fulfilled, build a projection operator $\pi_L: Q_h \rightarrow L_h$, build $s(\lambda_h, \mu_h)$ and prove that $\forall [u_h, {\ud}_h]$, there exists $\xi_h([u_h, {\ud}_h]) \in Q_h$ s.t. \eqref{stab_coercivity} and \eqref{stab_stability} holds. We recall that in the case of problem 2, 
\begin{equation*}
b([u_h, {\ud}_h], \lambda _h) = |\DD|\left(\avrc{T u_h} - {\ud}_h, \lambda _h\right)_{\Lambda}.
\end{equation*}
Let us consider macro patches $\left\{ F_j \right\}_j$ of elements of the 3D mesh $\mathcal{T}_h^{\Omega}$ intersecting the 1D manifold $\Lambda$. These patches are such that and  $H\leq |F_j\cap \Lambda|\leq H+h$, where $H$ is sufficiently larger than $h$. Moreover, there exist constants $c_h$ and $c_H$ such that $c_hh\leq H \leq c_H^{-1}h$. We define the space $L_h$ as the space of functions which are $P^0$ on each intersection $F_j\cap \Lambda$. Moreover, we associate to each patch $F_j$ a shape regular macro elements $\omega_j$, which is built adding to $F_j$ a sufficient number of elements of $\mathcal{T}_h^{\Omega}$.We make the following technical assumption: $\Gamma \subset \bigcup _{j} \omega_j$. Thanks to the shape regularity of these macro elements, again we have that the discrete trace and Poincarè inequalites hold. Moreover $\forall u_h \in X_h^\Omega$ we have the following average inequality 
\begin{equation*}
\sum _j |\DD| \|\avrc{T u_h}\|^2_{L^2(F_j \cap \Lambda)} \leq \sum _j \|T u_h\|^2_{L^2(\omega_j\cap \Gamma)}.
\end{equation*}
{\color{red} I think this inequality is valid but only globally. Indeed locally it is not guaranteed that the portion of $\Gamma$ corresponding to $F_j \cap \Lambda$ is contained in $\omega _j \cap \Gamma$}. \\
We define $\pi_L$ as the projection onto piecewise constant functions on $F_j\cap \Lambda$. This choice leads to the following stabilization 
\begin{equation*}
s(\lambda_h, \mu_h)= \sum _{K\in \mathcal{T}_{h'}^{\Lambda}} \int_{\partial K} h \llbracket \lambda_h \rrbracket \llbracket \mu_h \rrbracket,
\end{equation*}
being $\llbracket \lambda_h \rrbracket$ the jump of $\lambda_h$ across the internal faces of $\mathcal{T}_{h'}^{\Lambda}$.

\subparagraph{Is $L_h$ inf-sup stable with constants independent of the cuts?} We have to prove that $\forall l_h \in L_h$, $\exists \beta >0$ s.t.
\begin{equation*}
\sup_{\substack{v_h \in X_{h,1}^0(\Omega),\\ {\vd}_h \in X_{h',1}^0(\Lambda)}} \frac{|\DD|\left(\avrc{T v_h} - {\vd}_h, l _h\right)_{\Lambda}}{\vertiii{[v_h, {\vd}_h]}} \geq \beta \|l_h\|_{H^{-\frac 12}(\Lambda)}.
\end{equation*}
As in the continuous case, we can choose ${\vd}_h=0$ and we prove that
\begin{equation*} 
\sup_{v_h \in X_{h,1}^0(\Omega)} \frac{|\DD|\left(\avrc{T v_h} , l _h\right)_{\Lambda}}{\|v_h\|_{H^1(\Omega)}} \geq \beta \|l_h\|_{H^{-\frac 12}(\Lambda)}.
\end{equation*} 
Proving the last inequality it is equivalent to find the Fortin operator $\pi_F: H^1_0(\Omega) \rightarrow X_{h,1}^0(\Omega)$, such that 
\begin{equation*}
|\DD|\left(\avrc{T v} - \avrc{T \pi _F v}  , l _h\right)_{\Lambda}=0, \quad \forall v\in H^1_0(\Omega), \, l_h \in L_h
\end{equation*} 
and
\begin{equation*}
\|\pi_F v\|_{H^1(\Omega)}\lesssim \|v\|_{H^1(\Omega)}.
\end{equation*} 

We define
\begin{equation*}
\pi_F v = I_h v + \sum _j \alpha _j \varphi _j \qquad \text{with }\alpha_j =\frac{\int_{F_j \cap \Lambda}|\DD| (\avrc{Tv}-\avrc{TI_hv})}{\int_{F_j \cap \Lambda}|\DD|\avrc{T\varphi _j}}
\end{equation*}
and $\varphi_j \in X_{h,1}^0(\Omega)$ s.t. supp$(\varphi_j)\subset \bar{\omega}_j$, $\varphi_j =0$ on $\partial \omega _j$ and 
\begin{equation*}
 \int_{F_j\cap \Lambda}|\DD|\avrc{T\varphi_j}=O(H) \text{ and } \|\nabla \varphi\|_{L^2(\omega _j)}=O(1). 
\end{equation*}
This construction is always possible provided $H$ is sufficiently larger that $h$.
Then we have
\begin{multline*}
|\DD|\left(\avrc{T v} - \avrc{T \pi _F v}  , l _h\right)_{\Lambda} 
= \sum _j \int_{F_j\cap \Lambda} |\DD |\left[ \avrc{Tv}-\avrc{TI_hv}-\sum _i \alpha_i \avrc{T\varphi _i} \right]l_h \\
=(\text{supp} \varphi \subset \omega_j) \sum _j \int_{F_j\cap \Lambda}|\DD| \left[ \avrc{Tv}-\avrc{TI_hv}-\alpha_j \avrc{T\varphi _j} \right]l_h\\
=\sum _j \int_{F_j\cap \Lambda} |\DD| (\avrc{Tv}-\avrc{T I_h v}) l_h - \frac{\int_{F_j\cap \Lambda} |\DD| (\avrc{Tv}-\avrc{TI_hv})}{\int_{F_j\cap \Lambda}|\DD|\avrc{T\varphi_j}} \int_{F_j\cap \Lambda} |\DD|\avrc{T\varphi _j}l_h\\ 
=(\text{using $l_h$ constant on $F_j\cap \Lambda$})\,0.
\end{multline*}
Concerning the continuity of $\pi_F$, we have
\begin{multline*}
\|\nabla \pi_F v \|_{L^2(\Omega)} \leq \|\nabla I_h v\|_{L^2(\Omega)} + \left(\sum_j|\alpha_j|^2\|\nabla \varphi _j\|^2_{L^2(\bar{\omega}_j)}\right)^{\frac 12}\\
(\text{stability of }I_h)\lesssim   \|\nabla  v\|_{L^2(\Omega)} + \left(\sum_j|\alpha_j|^2\|\nabla \varphi _j\|^2_{L^2(\bar{\omega}_j)}\right)^{\frac 12}
\end{multline*}
and for the second term we have
\begin{multline*}
\sum_j|\alpha_j|^2\|\nabla \varphi _j\|^2_{L^2(\bar{\omega}_j)}\leq
\\
\left(\text{using }\|\nabla \varphi _j\|=O(1)\right) \lesssim  \sum_j \frac{\left(\left|\int_{F_j\cap \Lambda} |\DD| (\avrc{Tv}-\avrc{TI_hv})\right|\right)^2}{\left(\int_{F_j\cap \Lambda}|\DD|\avrc{T\varphi_j}\right)^2}
\\
\left(\text{since }\left|\int_{F_j\cap \Lambda}|\DD|\avrc{T\varphi_j}\right|=O(H)\right) \lesssim \frac {1}{H^2} \sum_j \left(\left|\int_{F_j\cap \Lambda} |\DD| (\avrc{Tv}-\avrc{TI_hv})\right| \right)^2
\\
(\text{Jensen}) \lesssim  \frac {1}{H^2} \sum_j |F_j\cap \Lambda| \int_{F_j\cap \Lambda} |\DD|^2(\avrc{Tv}-\avrc{TI_hv})^2
\\
(\text{being }|F_j\cap \Lambda| \leq H+h)\lesssim  \frac {1}{H} \sum_j \| \avrc{Tv}-\avrc{TI_hv}\|^2_{L^2(F_j\cap \Lambda), |\DD|}
\\
(\text{average inequality}) \lesssim  \frac {1}{H} \sum_j \| T(v-I_hv)\|^2_{L^2(\omega _j\cap \Gamma)}  
\\
\left(\text{trace inequality} \right)\lesssim  \frac {1}{H^2} \sum_j  \| v-I_h v\|^2_{L^2(\omega_j)} \lesssim  \frac {1}{H^2}  \| v-I_h v\|^2_{L^2(\Omega)} 
\\
(\text{approximation properties of }I_h)\lesssim \|\nabla  v\|^2_{L^2(\Omega)}
\end{multline*}
and the continuity of $\pi_F$ follows.

\subparagraph{Satisfaction of the assumptions of the 
abstract analysis} 
We have to prove \eqref{stab_coercivity} and \eqref{stab_stability}.
We choose the following discrete norm
\begin{equation*}
\vertiii{[u_h, {\ud}_h]}^2_{X_h(\Omega)\times X_{h'}(\Lambda) }
= \|u_h\|^2_{H^1(\Omega)}+|D|\|{\ud}_h\|^2_{H^1(\Lambda)} + |\DD|\|\avrc{Tu_h} - {\ud}_h\|^2_{-\frac 12, h, \Lambda},
\end{equation*}
where $|\DD|\|\avrc{Tu_h} - {\ud}_h\|^2_{-\frac 12, h, \Lambda} = |\DD|\|h^{\frac 12} (\avrc{Tu_h} - {\ud}_h)\|^2_{L^2(\Lambda)} $.
Concerning the coercivity property \eqref{stab_coercivity}, we have to show that $\forall [u_h, {\ud}_h]$, there esists $\xi_h \in Q_h$ s.t.
\begin{multline*}
(u_h,u_h)_{H^1(\Omega)}+ |D|({\ud}_h, {\ud}_h)_{H^1(\Lambda)} +  |\DD| (\avrc{Tu_h} - {\ud}_h, \xi_h)_{\Lambda} \\
\geq \alpha_{\xi}(\|u_h\|^2_{H^1(\Omega)}+|D|\|{\ud}_h\|^2_{H^1(\Lambda)}+ |\DD|\|\avrc{Tu_h} - {\ud}_h\|^2_{-\frac 12, h, \Lambda}.
\end{multline*}
We choose 
\begin{equation*}
{\xi_h}_{|F_j\cap \Lambda}=\delta \frac 1H \pi_L(\avrc{Tu_h}-{\ud}_h) \qquad \text{with } \pi_L(\avrc{Tu_h}-{\ud}_h) =\frac{1}{|\Gamma_{F_j\cap \Lambda}|}\int_{F_j\cap \Lambda}\|\DD\| (\avrc{Tu_h}- {\ud}_h),
\end{equation*}
being $\Gamma_{F_j\cap \Lambda}$ the portion of $\Gamma$ with centerline $F_j\cap \Lambda$. 
Actually, $\xi_h\in L_h \subset Q_h$. Then,
\begin{multline*}
|\DD|\left( \avrc{Tu_h} - {\ud}_h, \xi _h \right)_{\Lambda} 
= \sum_j \int_{F_j\cap \Lambda} |\DD|( \avrc{Tu_h} - {\ud}_h)\xi_h
\\
= \delta \frac{1}{H} \sum_j \int_{F_j\cap \Lambda}|\DD| (\pi_L( \avrc{Tu_h} - {\ud}_h) )^2
\\
(\text{orthogonality of $\pi_L$}) =  \delta \frac{1}{H} \|(\pi_L-\mathcal{I})(\avrc{Tu_h} - {\ud}_h)\|^2_{L^2(F_j\cap \Lambda),|\DD|} + \delta \frac{1}{H} \|\avrc{Tu_h} - {\ud}_h\|^2_{L^2(F_j\cap \Lambda),|\DD|}
\\ 
\geq -\delta \frac 1H \sum_j \|(\pi_L - \mathcal{I})\avrc{Tu_h}\|^2_{L^2(F_j\cap \Lambda), |\DD|}
- \delta \frac 1H \sum_j |\DD|\|(\pi_L - \mathcal{I}){\ud}_h\|^2_{L^2(F_j\cap \Lambda),|\DD|}
\\ 
+ \delta \frac 1H \sum_j |\DD|\|\avrc{Tu_h}-{\ud}_h\|^2_{L^2(F_j\cap \Lambda),|\DD|}. 
\end{multline*}
For the first term we have
\begin{multline*}
\sum _j  \|(\pi_L - \mathcal{I})\avrc{Tu_h}\|^2_{L^2(F_j\cap \Lambda),|\DD|} =\sum _j \int _{F_j \cap \Lambda} |\DD| (\pi_L \avrc{Tu_h}- \avrc{Tu_h})^2
\\
(\text{Average inequality)} \leq \sum _j  \int_{\omega _j \cap \Gamma} (\pi_L \avrc{Tu_h} -Tu_h)^2 
\\
(\text{trace inequality}) \leq \sum _j  \frac 1H \int_{\omega_j}(\pi_L \avrc{Tu_h} - u_h)^2 
\\ 
(\text{Poincare, see \cite[Corollary B.65]{MR2050138}})\leq \sum _j  H c_P ^2 \|\nabla u_h\|^2_{L^2(\omega _j)}.
\end{multline*}
For the second term we have
\begin{multline*}
\sum _j \|(\pi_L - \mathcal{I}){\ud}_h\|^2_{L^2(F_j\cap \Lambda),|\DD|} = \sum _j \int_{F_j\cap \Lambda} |\DD| (\pi_L {\ud}_h -{\ud}_h)^2
\\
(\text{Poincare, \cite[Corollary B.65]{MR2050138}})\leq \sum _j  H^2 c_P^2 \int_{F_j\cap \Lambda} |\DD|(\nabla {\ud}_h)^2
\\
(\text{since $H$ is fixed, we can find a constant s.t. } H|\DD| \lesssim |\D|) \lesssim \sum _j H c_P^2  \int_{F_j\cap \Lambda} |\D|(\nabla {\ud}_h)^2 
\\
\lesssim \sum _j H c_P^2  \|\nabla {\ud}_h\|^2_{L^2(F_j\cap \Lambda),|\D|}.
\end{multline*}
Therefore, we obtain
\begin{multline*}
a([u_h, {\ud}_h],[u_h, {\ud}_h] ) + b([u_h, {\ud}_h], \xi_h([u_h, {\ud}_h]))
\geq \\
(1-\delta c_P^2) \|\nabla u_h\|^2_{L^2(\Omega)} + (1- \delta c_P^2) |\D|\|\nabla {\ud}_h\|^2_{L^2(\Lambda)}
+\delta c_H |\DD| \|\avrc{Tu_h}-{\ud}_h\|^2_{-\frac 12,h,\Lambda}
\end{multline*}
and choosing $\delta=\frac{1}{2c_P^2}$ we obtain the coercivity inequality.\\
Concerning the stability inequality, the proof is analogous to the one in \cite{burman2014}.

